\documentclass{exam}
\usepackage{xcolor} \pagecolor[rgb]{0,0,0} \color[rgb]{1,1,1}
\begin{document}

\question How many spectral lines are emitted from a hydrogen atom excited to the state	designated by the principal quantum number, n=3?
\begin{oneparchoices}
        \choice 1	\choice 2	\choice 3	\choice 4
\end{oneparchoice}

\question How many moles are there in 159g of alanine, C3H7N02?
\begin{oneparchoices}
        \choice 0.560	\choice 0.992	\choice 1.78	\choice 3.31
\end{oneparchoice}

\question How many chloride ions are in 1.0 mole of CaCl2?
\begin{oneparchoice}
        \choice \(3.01\times10^{23} \space Cl^¯ ions\)	\choice \(1.81\times10^{24} \space Cl^¯ ions\)
        \choice \(6.02\times10^{23} \space Cl^¯ ions\)	\choice \(1.20\times10^{24} \space Cl^¯ ions\)
\end{oneparchoice}

\question Which of the following is the right  order of the steps of a scientific method?
\begin{choices}
        \choice Performing experiments formulating hypothesis Making observations
        \choice Formulating hypothesis Making observations performing experiments
        \choice Making observations formulating hypothesis performing experiments
        \choice Making observations performing experiments –Formulating  hypothesis
\end{choice}

\question The distance between two carbon atoms in a diamond is 154 pm. What is the distance	between the carbon atoms in millimeters?
\begin{oneparchoice}
        \choice \(7.7 \times 10^{-5}\)	\choice \(7.7 \times 10^{-7}\)	\choice \(1.54 \times 10^{-7}\)	 \choice \(1.54 \times 10^{-9}\)
\end{oneparchoice}

\question In which of the following numbers all of the zeros significant?
\begin{oneparchoice}
        \choice 100.090090	\choice 0.143290	\choice 0.1000	\choice \question 0.030020
\end{oneparchoice}

\question The first step of the scientific method involves
\begin{choices}
        \choice forming a hypothesis	        \choice making observations
        \choice performing an experiment	\choice predicting the result of an experimrnt
\end{choice}

\question Which of the following is correct when 34495 is rounded to three significant figures? 
\begin{oneparchoice}
        \choice 345	\choice 34500	\choice 344	\choice 3840
\end{oneparchoice}

\question What is the first step of scientific method?
\begin{choices}
        \choice Making observations	        \choice Forming a hypothesis
        \choice Performing an experiment	\choice Predicting the result of an experiment
\end{choice}

\question Which of the following is correct?
\begin{oneparchoice}
        \choice 1 Pa = \(10Nm^{-2}\) \choice 1 N = \(10\space kg\space ms^{-2}\)	\choice 0.00072 = \(7.2 \times 10^{-3}\)	\choice 1 L = \(1 dm^3\)
\end{oneparchoice}

\question Which of the following represents a tentative explanation of certain scientific law?
\begin{oneparchoice}
        \choice Hypothesis	\choice Observation	\choice Experimentation	        \choice Theory
\end{oneparchoice}

\question In order to advance to the level of theory, a hypothesis should...
\begin{choices}
        \choice be obviously accepted by most people	\choice be repeatedly confirmed by experimentation
        \choice be a fully functional experiment 	\choice report the past experience
\end{choice}

\question What is the equivalent of 500 C in \(^{o}F\) ?
\begin{oneparchoice}
        \choice 100 \(^{o}F\)	\choice 180 \(^{o}F\)	\choice 820 \(^{o}F\)	\choice 1229 \(^{o}F\)
\end{oneparchoice}

\question A student determined the density of an acid to be 3.91, 3.90, and 3.93 g cm- 3 . If the actual	density of the solid is 2.76 g cm- 3, how should the students result be described?
\begin{choices}
        \choice Low accuracy and low precision	\choice Low accuracy and high precision
        \choice High accuracy and low precision	\choice High accuracy and high precision
\end{choice}

\question A pattern or relationship that has been established based on a large amount of experimental      data is a
\begin{choices}
        \choice Theory	       \choice Hypothesis	\choice Law	\choice Scientific method
\end{choice}

\question Which of the following numbers has 4 significant figures?
\begin{choices}
        \choice 0.0430	\choice 0.04309	\choice 0.0431	\choice 0.43980
\end{choice}

\question Which of the following correctly expresses the number 0.0000850 in scientific notation? 
\begin{oneparchoice}
        \choice \(8.50 \times 10^{-5}\)	\choice \(8.50 \times 10^{-4}\)	\choice \(8.5 \times 10^{-5}\)	\choice 8.50  \times 105
\end{oneparchoice}

\question What is the sum of 3.71 x108 and 4.62 x107 to the correct significant figure? 
        \begin{choices}
\choice 4.17 X108	\choice 4.99 X107	\choice 4.17 X108	\choice 4.991 X 107
\end{choice}

\question What is the closeness of the measurement to its true value?
        \begin{choices}
\choice Precision	\choice Reproducibility	\choice Accuracy	\choice Usefulnes
\end{choice}

\question What skill is a scientist using when he/she listens to the sounds that animals make?
        \begin{choices}
\choice Drawing conclusion	\choice Making a hypothesis	\choice Making observation	\choice Interpreting data
\end{choice}

\question relationship between picometer(pm) and nanometer(nm)is:
        \begin{choices}
\choice 1pm=10nm	\choice 1nm=1000pm	\choice 1pm=100nm	\choice 1nm=10pm
\end{choice}

\question To determine the volume of an irregularly shaped glass vessel, the vessel is weighed empty (121.3 g) and when filled with CCl4(283.2g). What is the volume capacity of the vessel, given that the density of CCl4 is 1.59g/cm3? 
        \begin{choices}
\choice 76.29cm3	\choice 257.42cm3	\choice 178.11cm3 \choice 101.82cm3
\end{choice}

\question What is the bases for the scientific method?
        \begin{choices}
\choice To formulate a research problem and disprove the hypothesis.
        \choice To test hypotheses and if they are disproved, they should be abandoned completely.
        \choice To test hypotheses in conditions that are favourable to their success.
        \choice To formulate a research problem, test the hypotheses under carefully controlled conditions that challenge the hypotheses
        \end{choice}

\question Which of the following has the same number of significant figures as the number 1.00310? 
        \begin{choices}
\choice 199.791	\choice 1X106	\choice 100	\choice 5.119
\end{choice}

\question Precision refers to………..
        \begin{choices}
\choice How close a measured number is to the true value	\choice How close a measured number is to the zero
        \choice How close a measured number is to the calculated value	\choice How close a measured number is to other measured numbers
        \end{choice}

\question What is the first step in scientific investigation?
        \begin{choices}
\choice Ask questions	\choice Draw conclusions	\choice Do research	\choice Make observation
\end{choice}

\question Which of the following is the SI units of electric current?
        \begin{choices}
\choice Watt	\choice Volt	\choice Amphere	\choice Columb
\end{choice}



 	11 Chapter- 2

\question Which one the following electronic transition in a hydrogen atom releases the largest energy? 
        \begin{choices}
\choice n =2	    n=1	        \choice n = 6	n = 3
        \choice n=4	    n=2	        \choice n = 7	n = 6
        \end{choice}

\question Which set of quantum numbers (n, L, mℓ, ms) is not possible?
        \begin{choices}
\choice 1,0,0,1/2	\choice 1,1,0,1/2	\choice 1,0,0,- 1/2	\choice 2,1,- 1,1/2
\end{choice}

\question Which of  the following particles contains more electrons than  neutrons?
            I. 1
            II. 35Cl-
            III. 39K+
        \begin{choices}
\choice I only	\choice II only	\choice I and II only	\choice II and III only
\end{choice}

\question In which region of the periodic table would the element with the electronic structure below belocated? 1s2 2s22p6 3s2 3p6 3d10 4s2 4p6 4d6 5s2
        \begin{choices}
\choice Group 6	\choice Noble gases	\choice s block	\choice d block
\end{choice}

\question What is the ionization energy of an iron atom if it requires a radiation of 276nm to completely remove its outer most electrons in the gaseous state?
(planck’s constant, h = 6.626x10- 34Js, speed of light, C=3x108ms- 1)
        \begin{choices}
\choice 7.21x10- 19J	\choice 7.21x10- 19kJ	\choice 7.21x1019J	\choice 7.21x 1019kJ
\end{choice}

\question Which of the electron configurations describes the ground state electron configuration of Ca+2- ? 
        \begin{choices}
\choice 1s2 2s2 2p6 3s2 3p6	\choice 1s2 2s2 2p6 3p1
        \choice 1s2 2s2 2p6 3s1	    \choice 1s2 2s22p61s2 2s2 2p6 3s23px23py1
        \end{choice}

\question Which of the following statement is TRUE?
    \begin{choices}
\choice Ultraviolet light has longer wavelength than visible light
    \choice The energy of radiation decreases as the wave length decreases
    \choice The frequency of radiation increase as the wavelength decrease
    \choice Wave number of an electromagnetic radiation increase as wavelength increase
    \end{choice}

\question An electron has a spin quantum number, s= + ½ and a magnetic quantum number, m1 = +1, In which of the following orbital will it NOT be present?
    \begin{choices}
\choice S- orbital	\choice p- orbital	\choice d- orbital	\choice f- orbital
\end{choice}

\question Which of the following represents the general configuration of the transition elements?
    \begin{choices}
\choice ns2 np6 \choice ns(n- 1)d	\choice ns(n- 2)f	\choice ns2np6(n- 1)d10
\end{choice}

\question The quantum numbers listed below are meant for four different electrons in an atom: I.	n = 4, 1 = 0, m1 = 0, ms = + ½	II. n = 3, 1 = 1, m1 = 1, ms = + ½
II.	n = 4, 1 = 2, m1, = 0, ms = + ½	IVn = 4, 1 = 1, m1, = 0, ms = - ½
When these set as of quantum numbers are arranged in order of increasing energy, one may get:
    \begin{choices}
\choice I < II < III << IV	\choice I < III < II < IV	\choice II < I < III < IV	\choice IV < III < II < I
\end{choice}

\question The compound CuCl emits blue light having a wavelength of 450nm when heated at about 12000 C what is the increment in energy (quantum) that is emitted at 450nm?
    \begin{choices}
\choice 2.25x10- 19J	\choice 4.41x10- 19J	\choice 8.20x10- 19J	\choice 16.20x10- 19J
\end{choice}

\question What is the total number of valence- shell electrons in BrO3-
    \begin{choices}
\choice 20	\choice 26	\choice 32	\choice 36
\end{choice}

\question What is the number of moles of atoms and the number of atoms in a 10.0- g sample of copper?
    \begin{choices}
\choice 0.08 mol cu atoms & 2.16 x 1023 atoms	\choice 0.16 mol Cu atoms & 9.63 x 1022 atoms
    \choice 0.16 mol Cu atoms & 9.63 x 1023 atoms	\choice 0.31 mol Cu atoms & 4.16 x 1023 atoms

\question Which group of elements is characterized with ns 2 np 2 outer- electron configuration?
    \begin{choices}
\choice Group 2A	\choice Group 4A	\choice Group 4B	\choice Group 3B
\end{choice}

\question Which of the following quantum number/s determine the energy of an electron in a hydrogen atom?
    \begin{choices}
\choice n	\choice n and 1	\choice n, 1 and m	\choice n, 1 m and s
\end{choice}

\question For elements in the left- most column of the periodic table. Properties that have increasing values as the atomic number increases include which of the following?
        I. Ionization energy	II. Atomic radius	III. Atomic mass
    \begin{choices}
\choice III Only	B I, II, and III	\choice I and II only	\choice II and III only
\end{choice}

\question What did Rutherford’s particle experiment show?
    \begin{choices}
\choice Electrons have a negative charge
    \choice A proton is a hydrogen atom without electron
    \choice Electrons circle the nucleus of an atom in orbits
    \choice Most of the mass and all of the positive charge of an atom is found in a tiny nucleus.
    \end{choice}

\question Which of the following electron transitions requires the smallest energy to be absorbed by the hydrogen atom?
    \begin{choices}
\choice From n = 1 to n = 2	\choice from n = 3 to n = 4
    \choice from n = 2 to n = 3	\choice from n = 4 to n = 5
    \end{choice}

\question For an electron that has quantum numbers n = 4 and m l = 0, which of the following is true?
    \begin{choices}
\choice It must have the quantum number n = \question     \choice It must have the quantum number l = 0
    \choice It must have the quantum number m, = + 1/    \choice It may have the quantum number l = 0,1,2,3
    \end{choice}

\question For which of the following elements is Hund’s rule used in writing the electron configuration?
    \begin{choices}
\choice C	\choice B	\choice Be	\choice Li
\end{choice}

\question Which set of quantum numbers (n, L, mℓ .,ms) is NOT permitted by the rules of	quantum mechanics?
    \begin{choices}
\choice 1,0,0 1/2	B.2,1,- I,- 1/2	    C.3,3,1,- 1/2	D.4,3,2,1/2

\question What can you conclude from the figure below?
1S	2s	2P
    \begin{choices}
\choice Hund's rule has been violated.	                \choice The Pauli Exclusion Principle has been violated.
    \choice The Pauli Exclusion Principle has been violated.	\choice The Aufbau principle has been violated
    \end{choice}

\question This is a valid orbital diagram
    \begin{choices}
\choice          \choice          \choice          \choice      
\end{choice}
\question Which of the following is true about chlorofluorocarbons?
    \begin{choices}
\choice React directly with stratospheric ozone to destroy it.
    \choice Interact with UV energy and become free radicals which destroy ozone.
    \choice Become free radicals that react with oxygen to create ozone.
    \choice React with free radicals to remove carbon dioxide.
    \end{choice}

\question A monoatomic ion that has 20 protons and a +2 charge
    \begin{choices}
\choice Has 16 protons. \choice Has the symbol Ar2+ \choice has 18 neutrons  \choice is iso electronic with Ar
\end{choice}

\question According to valence bond theory, which orbital's on bromine atoms overlap in the formation of the bond in Br2
    \begin{choices}
\choice 3s	\choice 3p	\choice 4s	\choice 2p
\end{choice}

\question Which one of the following represents an acceptable possibl set of quantum numbers ( in the order n, l,m1,, ms) for an electron in an atom
    \begin{choices}
\choice 2, 1 , 0, 0	\choice 2, 0 ,2, +1/2	\choice 2,1,- 1, 1/2	\choice 2, 0, 1,- 1/2
\end{choice}

\question Of the types of radioactivity characterized by Rutherford, which of the following are particles
    \begin{choices}
\choice ϒ - rays	\choice β- rays	\choice α- rays and β- rays	\choice α - rays, β- rays , and ϒ - rays
\end{choice}

\question Consider the three electromagnetic waves shown below. 
Which of the electromagnetic waves has the highest frequency ?
    \begin{choices}
\choice 1	\choice 2	\choice 3	\choice 4
\end{choice}
\question Which of the following diagrams describes the electron density in the dxy orbital's
    \begin{choices}
\choice         \choice 	       \choice        \choice  
\end{choice}

\question The wave number of an electromagnetic radiation is 1 x 105 cm- 1 . The frequency of the	radiation would be 
    \begin{choices}
\choice 3 X 108 s- 1	    \choice 3 X 106 s- 1	    \choice 3 X 1010 s- 1	    \choice 3 X  1015 s- 1
\end{choice}

\question The maximum number of electron in p- orbital with n = 6, ml = 0 is
    \begin{choices}
\choice 2	\choice 6	\choice 16	\choice 14
\end{choice}

\question Which of the following transition will emit maximum energy in the hydrogen atom? 
    \begin{choices}
\choice n = 4  n = 3	\choice n = 4  n = 2	\choice n = 2  n = 1	\choice n= 3  n = 2
\end{choice}

\question What is the ratio of the energy of a photo of 300nm wavelength radiation to that of 600nm radiation? 
    \begin{choices}
\choice 1:2	\choice 1:1	\choice 2:1	\choice 3:1
\end{choice}

\question Which of the following quantum number(s) is (are) related to the size and energy of an electron in a hydrogen atom?
    \begin{choices}
\choice n	\choice n,l	\choice n,l,m	\choice n,l,m,s
\end{choice}

\question What is the difference between chlorine- 35 and chlorine - 37?
    \begin{choices}
\choice Chlorine- 37 has two more protons than chlorine- 35.
    \choice Chlorine- 37 has two more neutrons than chlorine- 35.
    \choice Chlorine- 35 has two more electrons than chlorine- 35
    \choice Chlorine- 37 has one more proton and one more neutron than chlorine- 35.
    \end{choice}

\question Which one of the following electromagnetic radiation has the shortest wavelength?
    \begin{choices}
\choice X- rays	\choice UV rays	\choice gamma rays	\choice microwaves
\end{choice}

\question How many atoms are present in 22 g CO2?
    \begin{choices}
\choice 3.10x1023	\choice 6.02x1023	\choice 2x6.02x1023	\choice 1.5x6.02x1023
\end{choice}

\question The hybridization of the central atom in the XeF4 molecule is
    \begin{choices}
\choice sp2	\choice sp3	\choice sp3d	\choice sp3d2
\end{choice}

\question Which of the following are NOT electromagnetic waves?
    \begin{choices}
\choice Infrared waves	\choice Gamma waves	\choice Radio waves	\choice Sound waves
\end{choice}

\question What is the distance that a radio wave will travel in 0.250s?
    A 1.2 ╳ 107 m	\choice 12 ╳ 107m	\choice 7.5 ╳ 107m	\choice 12 ╳ 107m
    \end{choice}

\question Which of the following types of rays combine to form atoms of helium?
    \begin{choices}
\choice gamma rays ()	\choice beta( ) rays	\choice alpha(a) rays	\choice X- rays
\end{choice}

\question What is the relationship between frequency (v) , wavelength () and the speed of light (c)?
    \begin{choices}
\choice v= c	\choice vc = h	\choice hc= v	\choice c = v
\end{choice}

\question What is the magnitude of quantum energy and the frequency for an object whose wavelength is 0.6 x 10- 6 m? 
    \begin{choices}
\choice 3.31 x 10- 19 J , 5 x 1014 s- 1	\choice 3.98 x 10 - 40 J , 2 x 10- 15 s- 1
    \choice 1.99 x 10- 25 J , 3.98 x 10- 40 s- 1	\choice 9.94 x 10- 12 J , 1.99 x 10- 25 s1
    \end{choice}

\question What new concept did Bohr adapt and use to formulate his model of the atom?
    A Electromagnetic theory developed by Maxwell	\choice The quantum concept developed by Planck
    \choice Photoelectric theory developed by Thompson	\choice Neutron theory developed by Chadwick
    \end{choice}

\question What is the energy required to excite a hydrogen atom by causing an electronic transition from the energy level with	n = 1 to the level with n = 4? En
- 21.79 x 10-19
    \begin{choices}
\choice 665 x 1026 J	\choice 1.824 x 10- 15 J	\choice 2.024 x 10 - 18 J	\choice 3.649 x 10- 15j
\end{choice}

\question Which statement below is true with regard to Bohrs model of the atom?
        \begin{choices}
\choice The model was based on the wave properties of the electron
        \choice The model accounted for the absorption spectra of atoms but not for the emission spectra
        \choice The model accounted for the emission spectra of atoms, but not for the absorption spectra
        \choice The model could account for the emission spectrum of hydrogen and for the Rydberg	equation
        \end{choice}

\question A radar unit is operating on frequency of 9.527 GHz. What is the wave length of the	radiation? 
        \begin{choices}
\choice 314.7nm	\choice 314.7m	\choice 3.147cm	\choice 314.7cm
\end{choice}

\question What important conclusion was reached through the study of cathode rays?
        \begin{choices}
\choice Cathode rays were shown to be neutral particles with mass
        \choice Cathode rays were proven to be light rays indicating that atoms were indeed indivisible
        \choice Cathode rays were shown to be positively charged particles indicating that atoms contained electric charge
        \choice The ratio of the charge to mass of particles making up cathode rays was constant, indicating they were fundamental particles found in all matter
        \end{choice}
\question If it takes 8.33min for light to travel from the sun to earth , how far away is the sun?
        \begin{choices}
\choice 1.86 x 105 miles	\choice 9.30 x 107 miles	\choice 3.72 x 107 miles	\choice 4.66 x 107 miles
\end{choice}

\question An element M with an atomic number of 25 has an electronic configuration of 1s22s22p63s23p64s23d5 .What will be its period and group, respectively, in the periodic	table? 
        \begin{choices}
\choice 4, 7B	\choice 4, 5B	\choice 5, 5B	\choice 6, 5B
\end{choice}

\question What is the electron configuration of sulfur?
        \begin{choices}
\choice 1s22s23p23s23p4	\choice 1s22s22p4	\choice 1s22s23p63s23p2	\choice 1s22s22p63p4
\end{choice}

\question What values of m1  are  permitted for an electron with   = 3?
        \begin{choices}
\choice 0,1,2,3,	\choice - 3, - 2 ,- 1 ,0 ,1, 2, 3	\choice - 2, - 1, 0, 1, 2	\choice 1, 2, 3
\end{choice}

\question used the cathode Ray Tube to discover the electron and determine its charge to mass ratio?
        \begin{choices}
\choice Robert. \begin{choices}
\choice Millikan	\choice Ernest Rutherford	\choice James Chdwick	\choice J.J Thomson
\end{choice}

\question The maximum kinetic energy of a photo electron emitted from a metal is 1.03 x10- 19Jwhen light  that has a 656nm wavelength shines on the surface s the threshold frequency for this metal?
        \begin{choices}
\choice 4.57 X10- 14 S- 1	\choice 4.57 X1014 S- 1	\choice 3.02 X10- 14 S- 1	D.3.02 X1014 S- 1

\question What is the maximum number of electrons in an atom that can have the principal quantum number n=4?	   
        \begin{choices}
\choice 32	\choice 8	\choice 18	\choice 34
\end{choice}

\question Which quantum number is used to determine sub shell?
        \begin{choices}
\choice Principal quantum no \choice Magnetic quantum number \choice Azimuthal quantum no \choice Spin quantum no
\end{choice}

\question Which of the following is fundamentally different from others?
        \begin{choices}
\choice Radio wave	\choice Sound wave	\choice Light wave	\choice Micro wave
\end{choice}

\question Which of the following equations’ expresses de Broglie hypothesis?
        \begin{choices}
\choice V =c /λ	\choice E =hc/λ	\choice E =c/λ	\choice λ=h/(mv)
\end{choice}

\question What wiil be the wavelength of a radio wave having a frequency of 3MHz? 
        \begin{choices}
\choice 300nm	\choice 300m	\choice 100nm	\choice 100m
\end{choice}

\question Which of the following correctly lists electromagnetic waves in order from shortest to longest wavelength?
        \begin{choices}
\choice Microwaves, ultraviolet, visible light, gamma rays
        \choice Radio waves ,infrared gamma rays ,ultraviolet
        \choice gamma rays ,ultraviolet ,infrared , microwaves
        \choice gamma rays ,infrared ,ultraviolet, microwaves
        \end{choice}

\question When an electron in a hydrogen atom makes the transition from the n=4 state, to the n=2state, blue light with a wavelength of 434nm is emitted. Which of the following expressions gives the energy released by the transition?
        \begin{choices}
\choice (6.63X10- 34)(4.34X10- 7)J	    \choice (6.63X10- 34) (3.00X108)J ( 3.00X108)		(4.34x10- 7)
        \choice (6.63X10- 34)	J	            \choice (4.34X10- 7)	J (3.00X108) (4.34X10- 7)		(6.6.3x10- 34) (3.00x108)
        \end{choice}

\question Which of the following is Not true about the photoelectric effect?
        \begin{choices}
\choice Most metals require ultraviolet light to emit electrons
        \choice A bright light causes less electron to be emitted the a weak light.
        \choice Hight frequency light emits electrons with high kinetic energy.
        \choice A bright light causes more electrons to be emitted than a weak light.
        \end{choice}

\question The sublevel that can be occupied by maximum of 10 electrons is identified by the letter…….?
        \begin{choices}
\choice f	\choice d	\choice p	\choice s
\end{choice}

\question The energy of an electron in the first bohr orbit of hydrogen atom is - 13.6ev. The possible value of the excited state for an electron in Bohr orbit of hydrogen is……..
        \begin{choices}
\choice - 4.21ev	\choice - 6.8ev	\choice - 1.51ev	\choice +6.8ev
\end{choice}

\question Consider the following two possibilities for electron transfer in hydrogen, given below:
First: The electron drops from the Bohr orbit n=3 to the orbit n=2, followed by the transition from n=2 to n=1.
Second: The electron drops from the Bohr orbit n=3 directly to the orbit n=
\question Which of the following is correct about  the energy change of these transitions?
        \begin{choices}
\choice The sum of the energies for the first transitions is less than the energy of transition of the second.
        \choice The energies of transitions of the first and the energy of transition of the second can’t be compared
        \choice The sum of the energies for the first transitions is greater than the energy of transition of the second.
        \choice The sum of the energies for the first transitions is equal to the energy of transition of the second
        \end{choice}

\question Which of the following elements has the highest fifth ionization energy (IE5)?
        \begin{choices}
\choice Si	\choice Al	\choice P	\choice S
\end{choice}

\question What aspects of the modern view of atomic structure was proved by Rutherford’s gold foil experiment?
        \begin{choices}
\choice The charge on an electron	\choice The charge on an alpha particle
        \choice The existence of the nucleus	\choice The existence of the electron
        \end{choice}

\question In the electromagnetic spectrum with wavelengths shown(in micrometers,μm), w/c bracketed section of the spectrum represents visible light?
        \begin{choices}
\choice O	\choice Y	\choice X	\choice Z
\end{choice}

\question Which of the orbitals in the figure below has (have) an angular momentum number of l=2?
                I
                II
                III
                IV
        \begin{choices}
\choice II   \choice I and III    \choice I    \choice I and Iv
\end{choice}


















\question Chapter- 3

\question The unit cell in a certain lattice consists of a cube formed by an anion at each corner, an anion in the center, and a cation at the center of each face. How many cations and how many anions does the unit cell have?
        \begin{choices}
\choice 5 anions and 6 cations	\choice 5anions and 3 cations	\choice 2 anions and 3 actions	\choice 3anions and 4 cations
\end{choice}

\question Which one of the following atoms in its ground state has the greatest number of unpaired electrons?
        \begin{choices}
\choice 13Al	\choice 14Si	\choice 15P	\choice 16S
\end{choice}

\question Which compound contains both covalent and ionic bonds?
        \begin{choices}
\choice Sodium carbonate, Na2CO3	\choice Dichloromethane, CH2Cl2
        \choice Magnesium bromide,MgBr2	\choice Ethanoic acid,CH3COOH
        \end{choice}

\question Which molecule or ion does NOT have a tetrahedral shape?
        \begin{choices}
\choice XeF4	\choice SiCl4    \choice BF-	\choice NH+
\end{choice}

\question Why are metals soft and malleable?
        \begin{choices}
\choice Because they are very shiny	\choice Because of the presence of mobile electrons
        \choice 	B/c they experience electrostatic repulsion	\choice Because the metal cations can slip over each other fairly easily
        \end{choice}

\question How many π bonds are present in CO2?
        \begin{choices}
\choice One	\choice Two \choice Three	\choice Four
\end{choice}

\question What is the correct molecular electronic configuration for the mfolecular ion, B2+?
        \begin{choices}
\choice σ1 22	21s     2s2σ2p2	    \choice σ1 22	21s     2s2	12s	2px1
        \choice σ1 22	21s     2s2π2py2	\choice σ1 22	21s     2s2π2pxπ2py
        \end{choice}

\question Which of the followazing molecules or ions will exhibit delocalized bonding? NO2-, NH4+, N -
        \begin{choices}
\choice NO2- and N3-	\choice NH + and N -	\choice NO - \choice NO2- and NH +
\end{choice}

\question Based on molecular orbital theory, the bond orders of H2, H2+ and H - are	respectively.
        \begin{choices}
\choice 1, 0 and 0 \choice 1, ½, and 0	\choice 1, 0, and ½	\choice 1, ½, and 1/2
\end{choice}
    
\question 	How many 3d electrons are present in the ground state of chromium atom? 
        \begin{choices}
\choice 4	\choice 5	\choice 6	\choice 1
\end{choice}

\question Which of the following ionic compounds is formed from the reaction between magnesium and nitrogen?
        \begin{choices}
\choice MgN2	\choice Mg2N2	\choice Mg2N2	\choice Mg2N3
\end{choice}

\question Which of the following molecules represents a non- polar covalent bond?
        \begin{choices}
\choice B- Cl	\choice C- Cl	\choice Cl- Cl	\choice Mg- Cl
\end{choice}

\question Which one of the following groups in the periodic table has paramagnetic atoms?
        \begin{choices}
\choice Group zero	\choice Group IIA	\choice Group IIB	\choice Group IVA
\end{choice}

\question How many types of cubic unit cells are known?
        \begin{choices}
\choice 2	\choice 3	\choice 4	\choice 5
\end{choice}

\question The total number of electrons participating in the bond formation of carbonate anion, CO32- , in the molecule of carbonic acid are:
        \begin{choices}
\choice 16	\choice 10	\choice 8	\choice 5
\end{choice}

\question Which of the following crystals possess high electrical and thermal conductivities?
        \begin{choices}
\choice Ionic crystals \choice Metallic crystals \choice Molecular crystals \choice Covalent network crystals
\end{choice}

\question Which of the following molecules has a trigonalbipyramidal structure?
        \begin{choices}
\choice SF4	\choice IF 5    \choice ICl4	\choice BrF5
\end{choice}

\question Which of the following hybrid orbitals is favoring the formation of trigonalbipyramidal?
        \begin{choices}
\choice Sp3d	\choice sp3	\choice sp3d2	\choice sp3d3
\end{choice}

\question Which one of the following molecules/molecular ions is paramagnetic according to the molecular orbital theory?
        \begin{choices}
\choice O 2-	\choice O2	\choice F2	\choice O22+
\end{choice}
    
\question Which of the following molecules has a dipole moment?
        \begin{choices}
\choice XeF4	\choice H2S	\choice SO3	\choice CH4
\end{choice}

\question Which of the following element has the highest melting point?
        \begin{choices}
\choice Iodine	\choice Tungsten	\choice mercury	\choice Bromine
\end{choice}

\question Which of the following is a chemical formula that represents and amino acid?
        \begin{choices}
\choice CH4	\choice CH3NH2	\choice CH3COOH	\choice NH2CH2COOH
\end{choice}

\question Which term describes the units that make up compounds with covalent bonds?
        \begin{choices}
\choice Ions	\choice Acids	\choice Salts	\choice Molecules
\end{choice}

\question There is a strong covalent bond between the N atoms in nitrogen gas, N\question Why, then, does nitrogen have such a low boiling point of - 1960C?
        \begin{choices}
\choice The bond between the N- atoms is triple
        \choice N is very electronegative, only next to F and O
        \choice The strong bond, and intermolecular one, determines the boiling point of the substance
        \choice Boiling point is determined by intermolecular force, which in this case is weak as the molecule is non- polar
        \end{choice}

\question Which of the statement below best explains why atoms react chemically with each other?
        \begin{choices}
\choice When atoms react, they gain protons and are more stable
        \choice When atoms react, they lose all their electrons and become more stable
        \choice When atoms react, they lose, gain, or share electrons and are then less stable
        \choice When atoms react, they lose, gain, or share electrons to attain a full outer energy level and are then more stable.
        \end{choice}

\question Which of the following species has the smallest H- X- H and angle where X is the central atom?
        \begin{choices}
\choice H2O	\choice NH3	\choice CH4	\choice BH3
\end{choice}

\question What is the hybridization of phosphorus atom in PCl5
        \begin{choices}
\choice Sp3d	\choice sp3d2	\choice sp3	\choice sp2
\end{choice}

\question Which molecule has a Lewis structure that does NOT obey the octet rule
        \begin{choices}
\choice NO	\choice CS2	\choice PF3	\choice HCN
\end{choice}

\question Which of the following explains why, at room temperature, 12 is a solid, Br2 is a liquid and CL2 is a gas?
        \begin{choices}
\choice Ionic bonding	\choice Hybridization	\choice Hydrogen bonding	\choice London dispersion forces
\end{choice}

\question Which molecule listed below has two sigma (π) bonds?
        \begin{choices}
\choice N2	\choice C2H4	\choice N2F2	\choice HCN
\end{choice}

\question What is the hybridization of the carbon atom attached to nitrogen in acetonitrile shown?
        \begin{choices}
\choice Sp	\choice sp2	\choice sp3	\choice sp4
\end{choice}

\question Which one of the following is NOT true of metallic bonding?
        \begin{choices}
\choice It gives rise to excellent electrical conductivity
        \choice Electrons are free to move throughout the structure
        \choice The strength of metallic bonds increases down a group.
        \choice The strength of metallic bonding affects the boiling point of metals.
        \end{choice}

\question All of these are characteristics of most ionic compounds in the solid phases EXCEPT,
        \begin{choices}
\choice High melting point	\choice high electrical conductivity
        \choice Solubility in water	\choice insolubility in organic solvents
        \end{choice}

\question Which one of the following does NOT form hydroxide ions when placed in water?
        \begin{choices}
\choice Ionic hydrides	\choice Ionic metal oxide	\choice nonmetal oxides	\choice ionic nitrides
\end{choice}

\question Which set contains only covalently bonded molecules?
        \begin{choices}
\choice BCl3,SiCl4, PCl5	\choice Br2, N2, HBr	\choice 12, H2S, NaI	\choice AI, O3, As4
\end{choice}

\question Which of the following compounds would be expected to have the highest melting point?
        \begin{choices}
\choice BaF2     \choice BaCl2	\choice BaBr2	\choice BaI2
\end{choice}

\question Which one of the compounds below is most likely to be ionic?
        \begin{choices}
\choice CCl4	    \choice NO2	\choice SCCl3	\choice ClO2
\end{choice}

\question When the following substances are arranged in order of increasing melting point ( lowest melting point first), the correct order is:
        \begin{choices}
\choice CH3 CH2 CH3, CH3 COCH3, CH3 CH2CH2 OH
        \choice CH3 CH2 CH3, CH3 CH2CH2 OH, CH3 COCH3
        \choice CH3 COCH3, CH3 CH2CH2 OH, CH3 CH2 CH3
        \choice CH3 CH2 CH2CH2OH, CH3 CH2 CH3, CH3 COCH3
        \end{choice}

\question The type of compound that is MOST likely to contain a covalent bond is one that is
        \begin{choices}
\choice a solid metal	\choice composed of only nonmetals
        \choice composed of a metal from the far left and a non metal from far right of the periodic	table \choice held together by the electrostatic forces between appositively charged ions
        \end{choice}
    
\question How many sigma and pi bonds are present in the following molecule ?	H3C- CH = CH- CH3
        \begin{choices}
\choice 8  bonds and 1  bond	\choice 8  bonds and 2  bond
        \choice 10  bonds and 2  bond	\choice 11  bonds and 1  bond
        \end{choice}

\question How many orbital's are there in an atom with n = 4? 
        \begin{choices}
\choice 2	\choice 8	\choice 16	\choice 25
\end{choice}

\question What hybridization change does the carbon atom undergo in the combustion of methane? CH4(g) + 2O2 (g)		CO2(g) +	2H2O (g)
        \begin{choices}
\choice sp  sp2	    \choice sp2		sp3	    \choice sp3 	sp	    \choice sp2  sp
\end{choice}
    
\question Which of the following ionic compounds has the greatest lattice energy?
        \begin{choices}
\choice LiF	\choice LiCL	\choice LiBr	\choice LiI
\end{choice}

\question How many unpaired electrons are there in the Lewis structure of a N3 - ion? 
        \begin{choices}
\choice 0	\choice 1	\choice 2	\choice 3
\end{choice}

\question Which one following compound does NOT follow the octet rule?
        \begin{choices}
\choice CS2	\choice PBr3	\choice IBr	\choice BrF3
\end{choice}

\question The molecular geometry of the H3O+ ion is
        \begin{choices}
\choice Linear	\choice tetrahedral	\choice bent	\choice trigonal pyramidal
\end{choice}

\question What is the hybridization of sulfur atom in SF6?
        \begin{choices}
\choice Sp2	\choice Sp3	\choice sp3 d	\choice sp3d2
\end{choice}

\question Which of the following electron transition required the smallest energy to be absorbed by the hydrogen atom?
        \begin{choices}
\choice From n=4 to n=5	\choice From n=3 to n=4	\choice From n=2 to n=3	\choice From n=1 to n=2
\end{choice}

\question Which of the following molecules has a dipole moment?
        \begin{choices}
\choice XeF2	    B.IF3	\choice BF3	\choice SF5 +
\end{choice}

\question The dissolution of water in octane (C8H18) is prevented by
        \begin{choices}
\choice dipole- dipole attraction between octane molecules
        \choice hydrogen bonding between water molecules
        \choice London dispersion forces between octane molecules
        \choice repulsion between like charged water and octane molecules
        \end{choice}

\question Which one of the following is NOT a form of chemical bonding?
        \begin{choices}
\choice Covalent bonding	\choice Metallic bonding	\choice Ionic bonding	\choice Hydrogen bonding
\end{choice}

\question Which of the following statement is NOT true about covalent bonding?
        \begin{choices}
\choice Covalent bonds are least likely to be formed between atoms of the element.
        \choice Covalent bonds are least likely to be formed between atoms of different elements on the	right side of periodic table
        \choice Covalent bonds are least likely to formed between an element in Group 1 and an element in	Group V11
        \choice Covalent bonds are least likely to be formed by head of the group elements with high	ionization energies
        \end{choice}

\question What values of l arepermittedforanelectronwithn= 4 ?
        \begin{choices}
\choice 1, 2, 3	\choice 1, 2 , 3, 4	\choice 0, 1 , 2, 3, 4	\choice 0, 1, 2, 3
\end{choice}

\question Which of the following electron, identified only by their n and l quantum numbers have the	highest energy?
                n = 3, l = 0
                n = 4, l = 1
                n = 3, 1 = 2
                n= 4, l = 2
        \begin{choices}
\choice n = 3 , l =2	\choice n = 4, l =1	\choice n = 4, 1= 2	\choice n = 3, l = 0
\end{choice}

\question What is the maximum number of unpaired electrons in a d shell? 
        \begin{choices}
\choice 2	\choice 5	\choice 3	\choice 4
\end{choice}

\question The following energy level diagram represents the outermost shell of what ground state	element?
        \begin{choices}
\choice B	\choice He	C, Al	\choice Be
\end{choice}

\question Formic acid , which is released by ants , has a molecular formula of HCOOH. What are the possible hybridiz2ations   3    that exist in the mo3lecule?	2	3 	2
        \begin{choices}
\choice spand sp
        \choice sp and sp
        \choice sp,spand sp	+
        \choice sp and sp
        \end{choice}

\question What would happen to the O2 molecule upon ionization to O2
        \begin{choices}
\choice The bond length will increase and the bond energy will increase
        \choice The bond length will increase and the bond energy will decrease
        \choice The bond length will decrease and the bond energy will increase
        \choice The bond length will decrease and the bond energy will decrease
        \end{choice}

\question How many bonding pairs and lone pairs , respectively does the ion ICl4 have? 
        \begin{choices}
\choice 3, 2	    \choice 4, 2	    \choice 5,1	    \choice 4, 1
\end{choice}
   
\question Which of the following molecules does NOT have a tetradral central atom?
        \begin{choices}
\choice SF4	\choice AlH4	\choice BF4	\choice SiCl4
\end{choice}

\question Acrylontrile has the following Lewis structure with designation of x, y and z for each carbon atom:
x	y	zCH2 = C - C ≡ N| H   What will be the value of the bond angle and geometry of y	z C - C ≡ N	?0	|	0	0	0
        \begin{choices}
\choice 109 ,tetrahedral	\choice 120 ,trigonal pyramidal	\choice 180 ,linear	\choice 90 , T- shaped
\end{choice}

\question Antimony (Sb) is a group V element . What will be the molecular geometry and number of	lone pair electrons, respectively that exist in the ion [SbCl5]2- ?
        \begin{choices}
\choice Seesaw,1	\choice Square planar, 2	\choice Seesaw, 2	\choice Linear, 3
\end{choice}

\question Which of the following molecule does NOT have a trigonal bipyramidal electron- pair	geometry?
        \begin{choices}
\choice SF4	\choice ClF3	\choice XeF2

\question How many atomic orbitals are required for an sp3 
        \begin{choices}
\choice 2	\choice 6	C.4	\choice 8    \choice BrF5 hybridization?
\end{choice}

\question A neutral molecule having the general formula AB , has two unshared pair of electrons on A .	What is the hybridization of A ?
        \begin{choices}
\choice sp	\choice sp2	\choice sp3	\choice sp3
\end{choice}

\question Which of following contains an sp2 hybridized atom?
        \begin{choices}
\choice CH2Cl	\choice H2O	\choice N2	\choice H2CCH2
\end{choice}

\question What is the electron set and molecular geometry of BrO2 ?
    \begin{choices}
\choice Trigonal planner, trigonal planar	\choice Tetrahedral ,trigonal planner
    \choice Trigonal pyramidal , linear      	\choice Tetrahedral, bent
    \end{choice}

\question According to VSEPR theory , what is the geometry of PCl3 molecule?
        \begin{choices}
\choice Linear	\choice Trigonal planner	\choice Trigonal pyramidal	\choice Tetrahedral
\end{choice}

\question What is the geometry of the molecular compound formed by the reaction of sulfur with	hydrogen?
        \begin{choices}
\choice Linear	\choice Trigonal planner	\choice Trigonal pyramidal	\choice Tetrahedral
\end{choice}
    
\question Which combination of atoms is more likely to produce an ionic compound?
        \begin{choices}
\choice Al and F	\choice P and H	\choice SI and  O	\choice S and Br
\end{choice}

\question What are the ions present in KHCO3 ?
        \begin{choices}
\choice KH+ and	CO3	\choice K+ , H+ , C4+ and O +	\choice K+ , HCO +	\choice KH2+ ,CO32-
\end{choice}

\question Which of the following substances contains an atom that obeys the octet rule?
        \begin{choices}
\choice PCl3	\choice AlF3	\choice SF4	\choice NO2
\end{choice}

\question Which of the following has formed coordinate covalent bond?
        \begin{choices}
\choice H2O	\choice NH4	\choice CO 2-	\choice Na2O
\end{choice}

\question Which of the following elements will form an ionic bond with chlorine?
        \begin{choices}
\choice Magnesium	\choice Oxygen	\choice Phosphorous	\choice Silicon
\end{choice}

\question The perchloric acid molecule contains
        \begin{choices}
\choice 8 lone pairs, no  bonds ,and 5 bonds	\choice 9 lone pairs ,2 bonds , and 5 bonds
        \choice 8 lone pairs , 3 bonds, and 5 bonds	\choice 2 lone pairs , 3 bonds , and 4 bonds
        \end{choice}

\question When a student draws a plausible Lewis structure for hydrazine molecule (N2H4), how many lon pairs of electrons are available?	
        \begin{choices}
\choice 2	\choice 1	C.3	\choice 4
\end{choice}

\question The number of resonance structures for CO 2- are: 
        \begin{choices}
\choice 3	\choice 2	\choice 6	\choice 9
\end{choice}

\question In the following equation, what type of hybridization change, if any, occurs at the Xe atom? XeF2(s) + F2(g)	XeF4(s)
        \begin{choices}
\choice Sp3d to sp3	\choice dsp2 to sp3	\choice sp3d to sp3d2	\choice sp3 to sp3d
\end{choice}

\question What is (are)the bond angle(s) in SF6?
        \begin{choices}
\choice 180o	\choice 109.5o	\choice 90o and 109.5o	\choice 90o
\end{choice}

\question Which of the following statements about oxygen and fluorine is NOT correct?
        \begin{choices}
\choice O and F have the same number of core electrons. \choice O has a smaller atomic radius than F.
        C.O has a smaller electron affinity than F.	\choice O2- has a larger ionic radius than F-
        \end{choice}

\question What will be the charges on the ions formed when silicon reacts with nitrogen? 
        \begin{choices}
\choice Si2+, N2-	\choice Si4+, N3-	\choice Si3+, N3+	\choice Si4+, N2-
\end{choice}

\question Which of the following compounds does NOT contain an ionic bond?
        \begin{choices}
\choice K2S	\choice NaOH	\choice HCl	\choice LiH
\end{choice}

\question Which of the following molecular orbital diagram is correct for the carbide ion (C 2- )?
        \begin{choices}
\choice σ1s2 σ* 1s2σ2s2σ* 2s2π2p4	\choice σ1s2 σ* 1s2σ2s2σ* 2s2π2p4σ2p2π* 2p4
        \choice σ1s2 σ* 1s2σ2s2σ* 2s2π2p4σ2p2π* 2p2	\choice σ1s2 σ* 1s2σ2s2σ* 2s2π2p4σ2p2
        \end{choice}

\question Which of the following is not the decomposition product of HNO3?
        \begin{choices}
\choice N2O4	\choice NO2	\choice O2	\choice H2O
\end{choice}

\question From CO2, H2O, BeCl2 and N2O which have the same molecular geometry?
        \begin{choices}
\choice CO2, BeCl2 and N2O	\choice CO2, H2O and N2O
        \choice CO2 and BeCl2 only	\choice H2O and N2Oonly
        \end{choice}

\question How many electrons are present in the σ2p molecular orbital of N +?
        \begin{choices}
\choice 1	\choice 4	\choice 3	\choice 2
\end{choice}

\question Give the following AFn species, BF3, BeF2 ,CF4,NF3, OF2,what is the correct order of F- A- F bond angles?
        \begin{choices}
\choice OF2<BeF2<NF3<BF3<CF4	    \choice OF2<NF3<CF4<BF3<BeF2
        \choice CF4<BF3<NF3<BeF2<OF2	    \choice BeF2<OF2<NF3<BF3<CF4
        \end{choice}

\question Which of the following molecules has the largest dipole moment?
        \begin{choices}
\choice HF	\choice HCN	\choice HCl	\choice CO
\end{choice}

\question Arrange the following molecules in the order of increasing stability.
        \begin{choices}
\choice N +<N2<N - <N22-	\choice N22- <N2- <N2<N2+	\choice N2<N +=N2- <N22-	\choice N22- <N - =N2+<N2
\end{choice}
    
\question Which of the following statements is correct about nitrosyl chloride (NOCl)?
        \begin{choices}
\choice It has a trigonal planar geometry with O a central atom
        \choice It has a bent or angular geometry with O a central atom
        \choice It has a trigonal planar geometry with N a central atom
        \choice It has a bent or angular geometry with N a central atom
        \end{choice}

\question What hybridization change, if any occurs at  the underlined atom in the following reaction? CO2   + H2O	H2CO3
        \begin{choices}
\choice Sp2 to sp3	\choice sp to sp2	\choice sp3 to sp3d	\choice No hybridization change observed
\end{choice}

\question What is the molecular shape of ICl4- ?
        \begin{choices}
\choice Octahedral	\choice T- shaped	\choice Trigonal bipyramidal	\choice Square planar
\end{choice}

\question Which one of the following types of  bonding exists between atoms with  very different electronegativities?
        \begin{choices}
\choice Ionic bonding	\choice Hydrogen bonding	\choice Network covalent bonding	\choice Metallic bonding
\end{choice}

\question Considering only resonance structures that are major contributors to the over all bonding in PF5, which of the following statements is correct?

\begin{choices}
        \choice There are no resonance structures that involve ionic contributions.
        \choice Only three resonance structures can be drawn for PF5
        \choice One resonance structures contains five P- F bonds.
        \choice In each resonance structure, the P atom carries a positive charge.
\end{choice}

\question Which groups in the periodic table form ionic bonds?
        \begin{choices}
\choice Groups IA and VIIB, Groups IIA & VIB	\choice Groups IA and 17(VIIA), Group IIA & 16(VIA)
        \choice Group IA & 18(VIIA), Groups IVB & 14 (IVA)	\choice Groups IIIB & VB, Group IVB & 14 (IVA)
        \end{choice}

\question There is a progressive decrease in the bond angle in the series of molecules CCl4, PCl3 and H\question According to the VSEPR model, this is best explained by:
        \begin{choices}
\choice Increasing electro negativity of the central atom	\choice increasing number of lone pairs electrons
        \choice Decreases the size of the central atom           	\choice decreasing bond strength
        \end{choice}

\question Which of the following compounds does not contain both ionic and covalent bond?
        \begin{choices}
\choice NH4NO3	\choice Na2CO3	\choice NH4Cl	\choice CH3CO2H
\end{choice}

11 Chapter-  4

\question In a reaction, A+B → product, the rate is doubled when the concentration of B is ‘doubled, and the rate increases by a factor of 8 when concentrations of both the reactants (A and B) are doubled, the rate law for the reaction can be written as:
        \begin{choices}
\choice Rate = k[ A] [ B]	\choice Rate = k[ A] [ B] 2	\choice Rate = k [ A] 2 [ B]	\choice Rate = k [ A] 2 [B] 2
\end{choice}

\question Which factor will influence the rate of the reaction shown below? NO 2(g) + CO(g) ⇌ NO(g) + CO2(g)
    I. The number of collisions per second
    II. The energy of the collisions
    III. The geometry with which the molecules collide
        \begin{choices}
\choice I only	\choice II only	\choice I and II only \choice I, II and III
\end{choice}

\question The mechanism of a reaction is shown is shown below. HOOH + I ¯ → HOI + OH - ¯ ( slow)
            HOI + I ¯ → I2 + OH ¯( fast)
            2OH ¯+ 2H3O + → 4H2O ( fast)
            What is the rate law based on this mechanism?
        \begin{choices}
\choice Rate = k [ HOOH] [ I- ]	\choice Rate = k [ HOOH] [ I- ] 2
        \choice Rate = k [ HOOH] 2 [ I- ]	\choice Rate = k [ HOOH]
        \end{choice}

\question The half life for the first order decomposition of nitro methane, CH3NO2, at 500k is 650 seconds. If the initial concentration of CH3NO2 is 0.500M, what will its concentration be(M) after 1300 seconds have elapsed?
        \begin{choices}
\choice 0.125	\choice 0.140	\choice 0.250	\choice 0.425
\end{choice}

\question 	In a zero- order reaction for every 100 rise of temperature, the rate is doubled. If the temperature is increased from 100c to 1000C, the rate of the reaction will become
        \begin{choices}
\choice 64times	\choice 128 times	\choice 256 times	\choice 512 times
\end{choice}

\question The kinetic data below are for the reaction:
            A + B → C
            [ A]	[ B]	Initial Rate ( mol dm+3 sec- 1)
            0.1	0.1	1x10¯ 5
            0.1	0.2	4x10¯ 5
            0.2	0.1	1x10¯ 5
        \begin{choices}
\choice order of A = 1 order of B = 0	\choice order of A = 0 order of B = 4
        \choice order of A = 0 order of B = 2	\choice order of A = 1 order of B = 2
        \end{choice}

\question Which of the following molecules represents a non- polar covalent bond?
        \begin{choices}
\choice B- Cl	\choice C- Cl	\choice Cl- Cl	\choice Mg- Cl
\end{choice}

\question What is a valid rate expression for the following reaction? 2NO + 2H2		N 2 + 2H2O
        \begin{choices}
\choice 1 ∆[ NO]
        \choice - 1 ∆[ H2o]
        \choice - 1 ∆[ NO]
        D.- ∆[ N2]

\question For the reaction: 2A + B	C The following experimental results were obtained:
   Experiment, What is the value of the rate constant?
        \begin{choices}
\choice 0.6mol L- 1s- 1	\choice 0.6Lmol- 1s- 1	\choice 1.2Lmol- 1s- 1	\choice 2.4molL- 1s- 1
\end{choice}
    
\question Increase in temperature of a reaction also increase the rate of a given reaction is due to the increase in the:
        \begin{choices}
\choice Extent of molecular dissociation	\choice Frequency of collision of the reacting species
\end{choice}
        \choice Activation energy of the reaction	Numerical value of the rate constant of the reaction

\question The reaction for the formation of nitrosyl chloride 2NO(g) + Cl2(g) ⇌ 2NOCl(g) Was studied at 250\choice The value of Kp for this reaction at 250C is 1.9 x 103 atm¯ 1 What is the value of Kc at 250C?
        \begin{choices}
\choice 1.9 x 10- 3 L/mol	\choice 3.8 x 10- 3 L/mol	\choice 4.6 x 104 L/mol	\choice 4.6 x 105 L/mol
\end{choice}

\question What is the half- life, t 1/2 for a zero order reaction A  	B, (K is rate constant)?
        \begin{choices}
\choice In2K	    \choice [ A] /2K	    \choice Ink[ A]  	\choice In2[ A] K
\end{choice}
        
\question Consider the following reaction: 2S2O32- (aq)+I2 (aq	S4O62 - (aq)+2I- (aq) If, in an experiment, 0.05 mol S2O32- is consumed in 1.0 L of solution each second, at what rates are S O 2- and 1- produced in this solution?
        \begin{choices}
\choice S O 2- =0.025;1- =0.025	\choice S4O62- =0.025;1- =0.05
        \choice S O 2- =0.05;1- =0.05	\choice S O 2- =0.05;1- =0.025
        \end{choice}

\question The reaction 2X + Y	Z was studied and the following data were obtained
Expt
[ X]
[ Y]
Rate (mole L- 1- s- 1)
1
3.0
3.0
1.8
2
3.0
1.5
0.45
3
1.5
1.5
0.45
What is the proper rate expression?
        \begin{choices}
\choice Rate = K[ X]	\choice Rate = K[ X] [Y]	\choice Rate = K[ Y] 2 D.2Rate = K[ X] 2 [ Y]

\question The reaction between NO and I2 is second order in NO and first- order in I What change occurs in the rate of the reaction if the concentration of NO is doubled and I2 left unchanged?
        \begin{choices}
\choice Double	\choice Quadruple	\choice Eight times	\choice Three times
\end{choice}

\question A reaction is 50% complete in 2 hours and 75% complete in 4 hours. What is the order of this reaction? 
        \begin{choices}
\choice 0	\choice 1	\choice 2	\choice 3
\end{choice}

\question Which are the number of moles and the mass of a copper sample containing 5.00xl020 atoms?
        \begin{choices}
\choice 3.8xlO- 4mol Cu and 5.2xlO- 2g Cu	\choice 5.2xlO- 2 mol Cu and 8.3xlO- 4g Cu
        \choice 8.3xl0- 4 mol Cu and 5.2xlO- 2g Cu	\choice 5.2xlO- 2 mol Cu and 3.8xI0- 4g Cu
        \end{choice}

\question Given the following reaction, what mass of gaseous carbon dioxide can be absorbed by lkg of lithium hydroxide? 2LiOH(s) + CO2 (g)  LiC03(s) + H20 (1)
        \begin{choices}
\choice 920g	\choice 1840g	\choice 2760g	\choice 3680g
\end{choice}

\question If a sample containing 36g NH2 is reacted with 180g of CuO, according to the following	reaction, then what is the limiting reactant and how many grams of N2 will be formed? 2NH3 (g) + 3CuO (s)  N2 (g) + 3Cu(s) + 3H20 (g)
        \begin{choices}
\choice NH3; 10.6gN2	    \choice NH3;22.3gN2	    \choice CuO; 22.3gN2	    \choice CuO; 10.6gN2
\end{choice}

\question If the fermentation of sugar in an enzymatic solution, which is initially 0.2M, the concentration of the sugar is reduced to O.IM in 10hours and to 0.05M in 20 hours. What is the order of the reaction and the rate constant?
        \begin{choices}
\choice First order K= 1.92xlO- ss- 1    \choice  Second order K=1.38xl0- 4M- 1s- 1
        \choice First order K= 3.85xI0- ss- 1    \choice Second order K= 2.72xlO- 4M- 1s- l
        \end{choice}

\question The reaction, 20r- - >302, proceeds through the mechanism given below: O3   O2 +  O,	fast O  +  O3      2O2	slow.
What would be the rate law expression for the reaction?
        \begin{choices}
\choice Rate = K[ 03f[ 02r1	\choice Rate = KfQ:rhE02l	\choice Rate = KL<hll:02]	\choice rate = K[ 03] 2
\end{choice}

\question Consider the following:
2NO(g)+Cl2  - - . 2NOCl(g), H=  - 78.38KJ
Which of the following does NOT affect the rate of a chemical reaction?
        \begin{choices}
\choice Enthalpy of the reaction 	\choice Surface area
        \choice Concentration of reactants	\choice Temperature
        \end{choice}

\question Which of the following is NOT a valid expression for the rate of the reaction given below? 4NH3+7O2 - 4NO2+6H2O
        \begin{choices}
\choice  [ NO2]	\choice lA[ NO2]	\choice 1A[ H20]	\choice 1A [ NH3]
\end{choice}

\question Each of the choices below gives a reaction and the corresponding rate law. Of these	choices, which one could be an elementary process or individual step in a chemical	reaction?
        \begin{choices}
\choice 2A P, rate = K[ A]  	\choice A+B  P, rate = K [ A] [ B]
        \choice A+2B P, rate = K [ A]	\choice A+B+C P, rate = K [ A] [ C]
        \end{choice}

\question Consider the reaction in which nitric oxide is oxidized to nitrogen dioxide: 2NO(g)+O2(g)- 2NO2(g)
For which the rate law is rate = k [ N0] 2[ 02] ' If this reaction takes place in a sealed		vessel and the   partial pressure of nitric oxide is doubled, what effect would this have on the rate	of reaction?
        \begin{choices}
\choice The reaction rate would increase by a factor of four.
        \choice The reaction rate would increase by a factor of three.
        \choice The reaction rate would increase by a factor eight.
        \choice The reaction rate would increase by a factor of two.
        \end{choice}

\question The equilibrium constant for reaction (1) is K what is the equilibrium constant for reaction	(2)? SO2(g) + O2    2SO3(g)	(1)
2SO3lg) 2SO2(g) + O2(g)	(2)
        \begin{choices}
\choice K2	B.2K	\choice IK	D.IK2

\question A homogeneous liquid reaction mixture is often heated to increase the rate of reaction. This is best explained by the fact that v raising the temperature:
        \begin{choices}
\choice Increases the heat of reaction.	    \choice Increases the vapor pressure of the liquid.
        \choice Decrease the energy of activation.	\choice Increases the average kinetic energy of the reactants.
        \end{choice}

\question Considering the reaction below , in which of the following will the effect of concentration and temperature simultaneously cause an increase in the rate at which products are formed?
CaCO3(s)	+	2HCl(aq)		CO2(g)	+	CaCl2(aq)	+	H2O(l)	+ heat
        \begin{choices}
\choice Decrease [ HCl ] and decrease temperature	\choice Increase [ HCl ] and decrease temperature
        \choice Increase [ HCl ] and increase temperature	\choice Grind up the CaCO3 and decrease temperature
        \end{choice}

\question For the gas phase reaction
N2    + O2	⇌	2NO H	=	+ 180KJmol- 1	the value of K changes with the
        \begin{choices}
\choice change in pressure	\choice introduction of NO
        \choice change in concentration of N2	\choice change in temperature
        \end{choice}
    
\question In the reaction	(2SO2	+   O2	⇌	2SO3 ,Keq = 100)	what will be the concentration of O2 , the concentration of SO2    is the same as that of SO3
        \begin{choices}
\choice [O2 ] = [SO2 ]	\choice [O2] = 0.01M	\choice [O2] = 100M	\choice [O2] = 0.1M
\end{choice}

\question The decomposition of nitrosyl chloride was studied as 2NOCl(g)	⇌	2NO(g) + Cl2 (g)
        The following data were obtained where
        Rate = - [NOCl] / t
        [NOCl]0	Initial Rate

        (molecules/cm3 )	( molecules/cm3 .s)

        3.0 x 1016	5.98 x 104
        2.0 x 1016	2.66 x 104
        1.0 x 1016	6.64 x 103
        4.0 x 1016	1.06 x 105
What is the rate law in the above decomposition?
        \begin{choices}
\choice r = k[NOCl]2	\choice r = k[NOCl]	\choice r = k[NOCl] [NO]	\choice r = k[NOCl] [Cl]
\end{choice}

\question Considering the mechanism for a reaction below , which of the following statement is correct? 
            Step 1: HBr + O2    	HOOBr
            Step 2 : HBr + HOOBr		2HOBr
            Step 3 : 2HOBr + 2HBr   	2Br2	+ 2 H2O
        \begin{choices}
\choice Br2 is reactant  	\choice HBr is a product
        \choice HOBr is a catalyst	\choice HOOBr is a reaction intermediate
        \end{choice}

\question The reaction A + 3B = 2C + D is first order with respect to reactant A and second order with respect to reactant B . If the conc of A is doubled and the concentration of B is halved , the rate of the reaction would….by a factor of…
        \begin{choices}
\choice increase ,2	\choice decrease ,2	\choice increase, 4	\choice decrease , 4
\end{choice}

\question What conditions of temperature and pressure will produce the highest yield of NOCI at equilibrium?
        \begin{choices}
\choice High temperature High pressure.	\choice Low temperature high pressure.
        \choice High temperature low pressure.	\choice Low temperature low pressure.
        \end{choice}

\question At 4450C,Ke for the following reaction is 0.020.
            2HI(g)   ⇌ H2(g)    + I2(g)
A mixture of H2, I2, and HI in a vessel at 4450C has the following concentrations: [ HI] = 2.0 M,[ H2] = 0.50M and [ I2] = 0.10M. which one of the following statements concerning the reaction quotient, Qc, is true for the above system?
        \begin{choices}
\choice Qc is less than Kc; more HI will be produced
        \choice Qc is greater than Kc ; more than HI will be produced.
        \choice Qc is less than Kc; more H2 and I2 will be produced.
        \choice Qc is greater than Kc ; more H2 and I2 will be produced.
        \end{choice}

\question The conventional equilibrium constant expression (Kc) for the system 2ICl(s)   ⇌ I2(s)    + Cl2(g) is
        \begin{choices}
\choice [I2] [Cl2]/ [ICl]2	\choice [I2] [Cl2]/ 2[ICl]
        \choice [Cl2]	            \choice [I2] + [Cl2]/2 [ICl]
        \end{choice}

\question How many electrons will appear when the following half- reaction is balanced? S O 2-   	S O 2-
        \begin{choices}
\choice 3	\choice 2	\choice 4	\choice 1
\end{choice}

\question The decomposition of carbon disulfide. CS2 to carbon monosulfide, CS, and sulfur is first order with K=2.8x10- 7S- 1 at 10000\choice What is the half- life of the reaction below at 10000C?
        CS2  CS+S
        \begin{choices}
\choice 5.0x10- 7S	\choice 4.7x10- 6S	\choice 3.8x105S	\choice 2.5x106S
\end{choice}

\question If we increase the concentration of a reactant , what happens to the collisions beteen particles?
        \begin{choices}
\choice There are more collisions            	\choice There are fewer collisions
        \choice There are the same number of collisions	\choice There are same number of collisions , but they have more energy	- 1
        \end{choice}
    
\question A drug decomposes by zero-- 1order kinetics with a rate constant of 2mg mL
month 1, If the initial concentration is 100 mg mL  , how long will it take for the drug to decompose by 10% ?
        \begin{choices}
\choice 2 month	\choice 3 month	\choice 5month	\choice 4 month
\end{choice}

\question For a first - order reaction , a plot of - - - - - - - - - - - versus	is linear.
        \begin{choices}
\choice   1 ,t	\choice Ln   1 ,t	\choice [A] t ,t	\choice Ln[A] t ,t[A]t	[A]t
\end{choice}

\question 	The rate law of the overall reaction	A +	B		C	is:	rate =	K[ A]2  Which of the following will NOT	increase the rate of the reaction?
        \begin{choices}
\choice Increasing the concentration of reactant A	\choice Increasing the temperature of the reaction
        \choice Increasing the concentration of reactant B	\choice Adding a catalyst for the reaction
        \end{choice}

\question Which of the following statement(s) is (are) applicable to a balanced chemical equation of an elementary reaction?
    i. Order is the same as molecularity	    ii. Order is less than the molecularity
    iii. Order is greater than the molecularity	iv. Molecularity can never be zero
        \begin{choices}
\choice i	\choice i , ii	\choice i , iv	\choice i , iii
\end{choice}

\question At high pressure, the following reaction is zero order 11.30 K,Pt
        2NH3(g)	→ N2(g)    +	3H2(g)
    i. Rate of reaction = rate constant
    ii. Rate of reaction depends on the concentration of ammonia
    iii. Rate of decomposition of ammonia remains constant until ammonia decomposes completely
    iv. Further increase in pressure will change the rate of reaction
        \begin{choices}
\choice i	\choice i , iii, iv	\choice i, ii	\choice i ,ii iv
\end{choice}

\question Which of the following expressions is correct  for the rate of the reaction given below? 5Br (aq) +	BrO3 (aq)	+	6H+ (aq)		3Br2 (aq) + 3H2O(l)
        \begin{choices}
\choice ∆[Br] ∆t = 5[∆H]∆t
        \choice ∆[Br]∆t= 5 /6 [∆H]∆t
        \choice ∆[Br]∆t= 6 /5[∆H]∆t
        \choice ∆[Br]∆t= 6[∆H]∆t
        \end{choice}

\question Rate for the reaction   A + 2B	➜ C is found to be Rate = K [A][B]
If the concentration of reactant  B is doubled , keeping the concentration  A constant , what will be the value of the rate constant?	
        \begin{choices}
\choice the same	\choice doubled	\choice halved	\choice quadrupled
\end{choice}

\question The oxidation of chloride by dichromate (Cr O 2- ) in acidic solution can be written as follows:
6Cl- 1(aq) + Cr2O72- (aq)		3Cl2(g)	+	2Cr3+(aq)	+	H2O (l)
The reaction is first order in Cl- 1 ,first order in Cr2O72- and second order in H+ . What is the change in initial rate if the concentration of Cl- 1 and Cr2O72- are halved ? The new rate will be /have
        \begin{choices}
\choice rate = 1 (initial rate)	\choice rate = 1( initial rate)8	2
        \choice rate = 14( initial rate)	\choice no change
        \end{choice}

\question Consider the following gaseous reaction and its rate law given below 2A(g)	+ B(g)		C(g)
Rate = K[A]2 [B]
In this reaction [A] = 2.0 M and the rate was recorded to be 0.048 mole 1- 1s- 1 . What will be the numerical value of the rate constant, K ?
        \begin{choices}
\choice 8.O	\choice 6.0 x 10- 3	\choice 3.0 x 10- 3	\choice 1.5 x 10- 3
\end{choice}

\question Given : A + 3B	 2C +D
This reaction is first order with respect to reaction A and second order with respect to reactant B . If the concentration of A is doubled and the concentration of B is halved , the rate of the reaction would	by a
factor of- - - - - - -
        \begin{choices}
\choice increase, 2	\choice decrease , 2	\choice increase , 4	\choice decrease , 4
\end{choice}
    
\question The graph shown below shows the variation of concentration of a reactant with time as a reaction proceed.What is the average reaction rate , in mol1- 1s- 1 , during the first 20s
        \begin{choices}
\choice 0.0025	\choice 0.0036	\choice 0.75	\choice 0.0090
\end{choice}

\question For zero order reactions ,which one of the following is true ?
        \begin{choices}
\choice The units of the rate constant (k) are time - 1
        \choice The half- life may be represented by the expression t 2 = 0.693/k
        \choice The rate of degradation is independent of the concentration of the reactant(s)
        \choice A plot of the concentration remaining against time is a straight line with a gradient of 1/k
        \end{choice}

\question If the reaction is zero order in A, tripling the concentration of A will cause the reaction rate to:
        \begin{choices}
\choice Increase by a factor of 27	\choice Remain constant
        \choice Increase by a factor of 3	\choice Increase by a factor of 9.
        \end{choice}

\question Which one of the following factor does NOT affect the rate of a chemical reaction?
        \begin{choices}
\choice Humidity	\choice Concentration	\choice Temperature	\choice Nature of reactants
\end{choice}

\question Consider the following equilibrium:
    2CO(g)   + O2(g)	2CO2(g)	Keq= 4.0 x 10- 10
    What is the value of Keq for 2CO2(g)	2CO(g) + O2(g)
        \begin{choices}
\choice 4.0 x 10- 10	\choice 2.5 x10 9	\choice 5.0 x 104	\choice 2.0 x 10- 5
\end{choice}

\question What species of ions are present in a 0.1M solution of HCl and what will be their equilibrium concentration? 
        \begin{choices}
\choice [ H3O] += 01M; [ OH]- = 0.1M, [ Cl] - =0.1M	    \choice [ H3O] + =0.1M; [ OH]- =10- 13M, [ Cl] =0.1M
        \choice [ H3O] + = 0.1M; [ OH]- =0.01M, [ Cl] - =0.1M	\choice [ H3O] + =10- 13; [ OH]- =0.1M,[ Cl] - =0.1M
        \end{choice}

\question The decomposition of a compound at 400oc is first order with the half life of 1570seconds. What fraction of an initial amount of the compound remains after 4710seconds?
        \begin{choices}
\choice 1/12	\choice 1/6	\choice 1/8	\choice 1/3
\end{choice}

\question The diagram below shows the range of energies of collision of a collection of reactants at two temperatures, T1 and T2.
Fraction of molecules	T1	T2
        \begin{choices}
\choice 1/12	\choice 1/6	\choice 1/8	\choice 1/3
\end{choice}

\question Which of the following is true regarding T1 and T2?
        \begin{choices}
\choice T1=T2, fraction of molecules at both temperatures are equal.
        \choice T1<T2, fraction of molecules at T1 is smaller.
        \choice T2<T1, fraction of molecules at T2 is smaller.
        \choice T1<T2, fraction of molecules at T1 is larger.
        \end{choice}

\question In three different experiments the following results were obtained for the reaction A	products: [ A] 0=1.00M, t1/2 = 50min; [ A] 0= 2.00M, t1/2=25min; [ A] 0=0.50M, t1/2= 100min. what is the value of the rate constant for this reaction?
        \begin{choices}
\choice 0.010Lmol- 1min- 1	\choice 0.030Lmol- 1min- 1	\choice 0.020Lmol- 1min- 1	\choice 0.040Lmol- 1min- 1
\end{choice}
    
\question The reaction below takes place with all of the reactants and products in the gaseous phase. Which of the following is true of the relative rates of disappearance the reactants and appearance of the products?
2NOCl	2NO + Cl2
        \begin{choices}
\choice NO appears at twice the rate that NOCl disappears.		\choice NO appears at half the rate that NOCl disappears. C.NO appears at the same rate that NOCl disappears	\choice Cl2 appears at the same rate that NOCl disappears
\end{choice}

\question The proposed reaction mechanism between nitrogen monoxide and bromine is given below. NO + Br2	NOBr2(fast)
NOBr2   + NO	2NOBr(slow)
Which of the following rate equetions is consist with the proposed mechanism?
        \begin{choices}
\choice Rate=K[ NO] 2	\choice Rate=k[ NO][ Br2] 2	\choice Rate=k[ NO] 2[ Br2]	\choice Rate=k[ NO][ Br2]
\end{choice}

\question The minimum energy required for an effective collision is called……?
        \begin{choices}
\choice activation energy	\choice Potential energy	\choice Free energy	\choice Kinetic energy
\end{choice}

\question For the reaction, N2(g) + 3H2(g)	2NH3(g), the rate of disappearance of H2 is 0.01molL- 1min-
    What is the rate of appearance of NH3?
        \begin{choices}
\choice 0.007molL- 1min- 1	\choice 0.02molL- 1min- 1	\choice 0.01molL- 1min- 1	\choice 0.002molL- 1min- 1
\end{choice}

\question The appropriate unit for a first order rate constant is?
        \begin{choices}
\choice 1/S	\choice 1/MS	\choice M/S	\choice 1/M2S
\end{choice}








11 Chapter- 5

\question Answer the following question using the phase diagram below.
At which point can only the solid and liquid phases coexist? 
        \begin{choices}
\choice 1	\choice 2	\choice 3	\choice 4
\end{choice}

\question Which statement is true about chemical reactions at equilibrium?
        \begin{choices}
\choice The forward and back ward reactions proceed at equal rates
        \choice The forward and backward reactions have stopped
        \choice The concentrations of the reactants and products are equal
        \choice The forward reaction is exothermic
        \end{choice}

\question Which changes will increase the amount of SO3(g) at equilibrium? 2SO2(g) + O2(g) ⇆2SO3(g)	∆H0 = - 197kJ
    I. Increasing the temperature
    II. Decreasing the volume
    III. Adding a catalyst
        \begin{choices}
\choice I only	\choice II only	\choice I and II only \choice I,II and III
\end{choice}

\question What is the equilibrium constant expression for the following reaction? 2Hg(g) + O2(g) ⇆2HgO(s)
        \begin{choices}
\choice k = 1/( [ Hg] 2 [ O2] )	\choice k = [ HgO] 2 / ( [ Hg ] 2 [ O2 ] )
        \choice k = [ Hg] 2 [ O2]	\choice k = [ 2HgO] / ( [ 2Hg] [ O2]
        \end{choice}

\question Which of the following mathematical relationships between K, K1 and K 2 correct? CO2(g) + H2 (g) ⇆ CO (g) + H2O(g)	K
Fe (s) + CO2 (g) ⇆ FeO (s) + CO (g)	K1 Fe(s) + H2O(g) ⇆ FeO (s) + H2 (g)	K2
        \begin{choices}
\choice K = K1 + K2	\choice K = K1 /K2
        \choice K = K1 x K2	\choice K = K2/K1
        \end{choice}

\question The value of Keq for the following equilibrium reaction is 4.0 at a temperature of 373K. CH3COOH + C2H5OH ⇌ CH3COOC2H5 + H2O
What mass of ethyl ester ( CH3 COOC2H5) would be present in the equilibrium mixture if 15g of acetic acid and 11.5g of ethanol were mixed and equilibrium was established at this temperature?
        \begin{choices}
\choice 5.2	    \choice 10.1	    \choice 12.6	    \choice 14.1
\end{choice}

\question Which of the following statements is TRUE about equilibrium reaction?
        \begin{choices}
\choice No more reactants are transformed into products
        \choice There are equal amounts of reactants and products
        \choice The rate constant for forward reactions equals that  of the reverse reaction
        \choice The rate for the forward reactions equals that of the reverse reactions
        \end{choice}

\question 	Three gases are in equilibrium in a closed chamber sealed with a piston. The following equilibrium is established :
2NH3(g) ⇌ N2(g) + 3H2(g)
What will happen if the piston is pushed into the chamber?
        \begin{choices}
\choice The mole fraction of N2 increases	\choice The mole fraction of N2 remains the same
        \choice The mole of N2 decreases	\choice The mole fraction of N2 increases and then decreases
        \end{choice}

\question Consider the following phase Diagram for CO2
What happens when in a CO2 sample initially at 1 atm and - 700C the temperature increases from - 700C to
- 100C at a constant pressure of 60 atm?
        \begin{choices}
\choice CO2(g)	CO2(s)	\choice CO2(g)		CO2(g)
        \choice CO2 (s)   	CO2(1)	\choice CO2(g)		CO2(I)
        \end{choice}
\question What will happen if NaOCl is added to this reaction at equilibrium HOCl +	H2O   ⇌	H3O+   +   OCl ¯	?
        \begin{choices}
\choice The concentrations of both HOCl and H3O+ would increase.
        \choice The concentrations of both HOCl and H3O+ would decrease.
        \choice The concentration of HOCl would increase and the concentration of H3O+ would decrease.
        \choice The concentration of HOCl would decrease and the concentration H3O+ would increase.
        \end{choice}

\question Consider the following equilibrium
CaCO3(S)	⇌	CaO(S) + CO2(g)
Which of the following mixtures, each placed in a closed container and allowed to stand is not capable of reaching the equilibrium given above?
        \begin{choices}
\choice Pure CaCO3	\choice Some CaO and a pressure of CO2 greater than the value of Kp
        \choice CaCO3 and CaO	\choice Some CaCO3 and a pressure of CO2 greater than the value of Kp
        \end{choice}

\question Which of the following statement correctly describes a chemical reaction at equilibrium?
        \begin{choices}
\choice The concentrations of the products and reactants are equal
        \choice The change in the concentrations of the products and reactants is constant
        \choice The rate of the forward reaction is less than the rate of the reverse reaction
        \choice The rate of the forward reaction is greater than the rate of the reverse reaction
        \end{choice}

\question If the following reaction is at equilibrium, which one of the following changes will shift the equilibrium to the left?
            N2 + 3H2	2NH3 + heat
        \begin{choices}
\choice Increasing pressure	\choice Adding more N2 and H2
        \choice Decreasing temperature	\choice Increasing the volume of the reaction container.
        \end{choice}

\question Suppose reactions A	B and B	A are both elementary processes with rate constants of 8 x 102s- 1 and 4 X 104s- 1, respectively. What is the value of the equilibrium constant for the equilibrium?
A ⇌ B
        \begin{choices}
\choice 2 x 102	\choice 0.5 x 102	\choice 4 x 102	\choice 4 x 102
\end{choice}

\question Which one of the following will change the value of an equilibrium constant?
        \begin{choices}
\choice Changing the temperature.
        \choice Adding other substances that do not react with any of the species involved in the equilibrium.
        \choice Varying the initial concentration of reactants.
        \choice Varying the initial concentration of products
        \end{choice}

\question The conventional. equilibrium constant expression (Kc) for the system as described by the equation: 2ICI(s) ⇌ I(s)+Cl2(g) is:
        \begin{choices}
\choice [ Cl]	\choice [ Cl2] / [ ICI] 2	C.[ l2] [ Cl2] / [ ICl]	\choice [ I2] [ Cl2] / [ ICl]
\end{choice}

\question The value of Keq for the following reaction is 0.5 SO2(g)   +	NO2(g)   ⇆ SO3(g)   +	NO(g)
What is the value of Keq at the same temprature for the reaction below ? 2 SO2(g)   +	2NO2(g)   ⇆ 2 SO3(g)   +	2NO(g)
        \begin{choices}
\choice O.25	\choice 0.026	\choice 0.50	\choice 16
\end{choice}

\question The following equilibrium constants were determined at 3000 C
2N2O(g)		⇆	2N2(g) +   O2(g)		Kc =	4.0 x 1018 N2(g) +   O2 (g)	⇆	2NO(g)	Kc    = 4.0 x 10- 31
What will be the equilibrium constant at 3000 C for the gaseous reaction of N2O(g)   +   O2 (g)	⇆	4NO(g) ?
        \begin{choices}
\choice 3.2 x 10- 12     \choice 2 x 10- 13	\choice 5.O   x 1050	\choice 1.6 x 10- 49
\end{choice}

\question When 0.50 mol of N2O4 is placed in a 4.0 liter reaction vessel and heated to 400K, 80% of the N2O4
    decomposes to	NO2 gas as follows:
    N2O4 (g)	⇆	2NO2 (g)
    What will be the value of Kp , in units of pressure , at 400K for this reaction? 
        \begin{choices}
\choice 2.62	    \choice 13.12	\choice 50.48	\choice 16.20
\end{choice}

\question Consider the following graph ,which rela- tes to the equ+ ilibrium system:
CH3COOH(aq) +   H2O	⇆	CH3COO (aq) +	H3O (aq)	∆H < 0
        \begin{choices}
\choice 2.62	    \choice 13.12	\choice 50.48	\choice 16.20
\end{choice}

\question Which of the following actions caused the change in the concentration of [ H3O+ (aq)] at time t ?
        \begin{choices}
\choice Addition of CH3COO- (aq)	\choice Addition of HCl
        \choice Decreasing of temperature	\choice Increasing the volume of the container
        \end{choice}

\question In which of the following systems will the position of equilibrium shift to the left upon an increase in pressure, but to	the right upon an increase in temperature ?
        \begin{choices}
\choice CO2 (g) +   H2(g)	⇌	CO (g) +   H2O(g)	∆H > 0
        \choice C2H4(g)	+	H2O(g)	⇌	C2H5OH(g)	∆H < 0
        \choice C2H6	⇌	C2H4(g)	+	H2(g)	∆H > 0
        \choice 2SO2(g)	+	O2(g)	⇌	2SO3(g)	∆H < 0
        \end{choice}

\question The hydrogen used in the Haber process is made by the follo0 wing reaction:
CH4(g) +   H2O(g)	⇌	CO(g)	+	3H2(g)	∆H	= +206 kJ
Which of the following sets of conditions will favor the formation of H2?
        \begin{choices}
\choice Low pressure and high temperature	\choice Low pressure and low temperature
        \choice High pressure and low temperature	\choice High pressure and high temperature
        \end{choice}

\question Why does the rate of the reaction increase when powdered calcium carbonate is used instead of marble chips?
        \begin{choices}
\choice The powdered calcium carbonate acts as a catalyst
        \choice There is an increase of the concentration of the calcium carbonate
        \choice There is an increase of yhe particles size of the calcium carbonate
        \choice There is an increase of the suface area of the calcium carbonate
        \end{choice}

\question In the Haber process for the synthesis of ammonia, the expected reaction is N2 (g)	+	3H2 (g)	2NH3 (g) + 92.4KJmol- 1
Which of the following is true about this process at equilibrium?
        \begin{choices}
\choice Concentration of reactant and product are equal
        \choice The forward and backward reaction rate are equal
        \choice The formation of ammonia is more dominant at equilibrium
        \choice Formation and dissociation of ammonia at equilibrium is static.
        \end{choice}

\question In the coal - gasification process, carbon monoxide is converted to carbon dioxide via the following reaction: CO(g) +	H2O (g)	⇆	CO2(g)	+	H2(g)
In an experiment , 0.35 mol of CO and 0.40 mol of H2O were placed in a 1.00- L reaction vessel. At equilibrium
, there were 0.19 mol of CO remaining . What is Keq at the temperature of the experiment?
        \begin{choices}
\choice 0.25 	\choice 0.36	    \choice 0.56	    \choice 0.78
\end{choice}

\question A pure substance is heated as indicated in the diagram below . Which section of the graph indicates the
boiling point?
        \begin{choices}
\choice A	\choice B	\choice C	\choice D
\end{choice}

\question The value of Keq for the equilibrium
H2 (g)	+	1/2 I2(g)	⇆	2HI(g)
is 794 at 25 0C . At this temperature. what is the value of Keq for the equilibrium below?
HI(g)	⇆	1/2H2(g)	+ 1/2I2(g)
        \begin{choices}
\choice 0.0013	\choice 0.035	\choice 28	\choice 397
\end{choice}

\question Which one of the following will change the value of equilibrium constant?
        \begin{choices}
\choice Adding other substances that do net react with any of the specis involved in the equilibrium.
        \choice Varying the initial concentration of reactants
        \choice Varying the initial concentration of products
        \choice Changing temperature
        \end{choice}
\question Which of the following statements is true about equilibrium involving a chemical reaction?
        \begin{choices}
\choice The rate constants of the forward and reverse reactions are equal
        \choice The rate of the forward and reverse reactions are equal
        \choice The value of the equilibrium constant is 1
        \choice All chemical reactions have ceased
        \end{choice}

\question The rate equation for the decomposition of nitramide, H2NNO2	N2O + H2O, is Rate=k[ H2NNO2][ H+] - 1 . Which of the following mechanisms is consistent with this rate equation?
        \begin{choices}
\choice H2NNO2	N2O + H2O	slow
        \choice H2NNO2 + H+			H3NNO2+     fast equilibrium H3NNO +		N2O + H3O+	slow
H3O+	 		H+ H2O	fast equilibrium
        \choice H2NNO2			OH- + NH +	slow NH +		NH3 + H+	fast equilibrium
H2O		H+ + OH-	fast equilibrium
        \choice H2NNO2		H+ + HNNO - fast eqilibrum HNNO2-			N2O + OH-	slow
        \end{choice}
H+ + OH-	H2O	fast

\question Given the equilibrium constant values:
N2(g) + 1/2O2(g)			N2O(g) Kc= 3.4 x10- 18 N2O4(g)	2NO2(g)	Kc= 4.6 x10- 3 1/2N2(g) + O2		NO2(g) Kc=4.1 x10- 9
What is the value of Kc for the following reqaction? 2NO2   + 3O2(g)		2N2O4(g)
        \begin{choices}
\choice 2.4 x 10- 6  	\choice 1.2 x 106	    \choice 1,2 X10- 6	    \choice 4.8 x 106
\end{choice}

\question The equilibrium constant for the ionization of HCN is 4.9 x10- 10
HCN		H+ + CN- K=4.9 x10- 10
Which of the following statements is true regarding this equilibrium?
            I. The reaction is product favored	III. Equilibrium lies far to the right
            II. The reaction is reactant favored	VI. Equilibrium lies far to the left
        \begin{choices}
\choice II and III	\choice I and III	\choice II and IV	\choice I and IV
\end{choice}

\question For the certain gas phase reaction
    2A(g)	B(g) + C(g)	H=+45KJ/mol, K=4.5x 10- 2
    Which of the following would be true if the temperature was increasesd from 25oC to 200oC ?
        I. The value of K would be smaller
        II. The concentration of A (g) would be increased.
        III. The concentration of B(g) would increase.
        \begin{choices}
\choice III	\choice II	\choice I	\choice I and III
\end{choice}

\question For the reaction C6H14(g)	C6H6(g) + 4H2(g) , P(H2)/t was found to be 2.5 x10- 2 atm/s, where  P(H2) is change in the pressure of hydrogen. Determine P(H2)/t(in units of atm/s) for this reaction at the same
time.
        \begin{choices}
\choice - 6.2x10- 3	\choice 1.6x10- 3	\choice 2.5x10- 2	\choice 6.2x10- 3
\end{choice}

\question Consider the following equilibrium: Cl2(g) + 2NO(g)	2NOCl(g) Keq=5.0
At equilibrium , [ Cl2] =1.0M and [ NO] =2.0M. What is the [ NOCl] at equilibrium? 
        \begin{choices}
\choice 4.5M	    \choice 0.89M	\choice 0.80M	\choice 10M
\end{choice}

\question Which of the following statement is NOT true in relation to the triple point on a single component phase diagram?
    \begin{choices}
\choice The point at which the solid, liquid and gaseous phases for a substance coexist.
    \choice The system must be enclosed so that no vapour can escape.
    \choice The triple point exists for a substance occurs at a specific temperature and pressure.
    \choice The triple point exists at a single temperature and is independent of pressure.
    \end{choice}

\question In the figure shown below, what does O denote?
    \begin{choices}
\choice Melting point	\choice vaporization	\choice Boiling point	\choice Triple point
\end{choice}

\question Which one of the following statements regarding a dynamic equilibrium is false?
    \begin{choices}
\choice At equilibrium, the forward and reverse reaction ceases to occur.
    \choice At equilibrium, there is no net change in the system
    \choice At equilibrium, the rate of the forwared and backwared reactons is identical.
    \choice At equilibrium, the concentration of reactants and products saty the same.
    \end{choice}

\question A sample of solid ammonium carbamate is heated in a closed container at 298K and allowed to reach equilibrium.	NH4CO2NH2(s)	2NH3(g) + CO2(g)
If the total pressure of the system is 0.114atm, what is the value of equilibrium constant, KP? 
    \begin{choices}
\choice 1.29X10- 3	\choice 3.80X10- 4	\choice 2.19X10- 4	\choice 7.60X10- 3
\end{choice}
    
\question Which one of the following reaction at equilibrium would be unaffected by an increase in pressure? I. N2(g) + 3H2(g)	2NH3(g)	II. 2H2(g) + O2(g)	2H2O(g)
III. N2(g) +N2(g)	2NO(g)	IV. 2CO(g) + O2(g)	2CO2(g)
    \begin{choices}
\choice I	\choice II	\choice III	\choice IV
\end{choice}

\question A sealed isothermal container initially contained 2mole of CO gas and 3moles of H2 gas. The following reversible reaction occurred: CO(g) + 2H2(g)	CH3OH(g) at equilibrium, there was one mole of CH3OH in the container at equlibreium?
        2X(g)	3Y(g) + Z(g)H(forward rxn)>0
    \begin{choices}
\choice 1	\choice 3	\choice 2	\choice 4
\end{choice}

\question The molar equilibrium concentrations for the reaction mixture represented above at 298K are
[ X] =4.0M,[ Y] =5.0M, and [ Z] =2.0M.What is the value of the equilibrium constant, Keq, for the reaction at 298K?
    \begin{choices}
\choice 16.0	\choice 2.50	\choice 0.06	\choice 62.5
\end{choice}













































Chemistry grade- 11 Entrance Chapter- 6

\question Commercially, liquid vegetable oils are converted to solid fats such as margarine by:
        \begin{choices}
\choice Hydrogenation	\choice Hydration	\choice Saponification	\choice Oxidation
\end{choice}

\question What is the chemical name for Aspirin?
        \begin{choices}
\choice Acetyl salicylic acid	\choice Salicylic acid	\choice Methyl salicylate	\choice Sodium salicylate
\end{choice}

\question Which compound is a carboxylic acid?
        \begin{choices}
\choice CH3,COOH	\choice (CH3CO)2O \choice (CH3)2CHOOCH3	\choice (CH3)2O
\end{choice}

\question A triacylglyceol that is solid at room temperature is called :
        \begin{choices}
\choice Lecithin	\choice Fat	\choice Wax \choice Oil
\end{choice}

\question Which compound is an ester?
        \begin{choices}
\choice CH3 COOH	\choice CH3OC2H5     \choice C2H5CHO	\choice HCOOCH3
\end{choice}

\question Which of the following gives the correct order of decreasing acidity of carboxylic acids?
        \begin{choices}
\choice Cl3 CCOOH, Cl2CHCOOH, FCH2COOH, CH3COOH
        \choice FCH2COOH, CH3  COOH, Cl2 CHCOOH, Cl3 CCOOH
        \choice CH3 COOH, FCH2 COOH, Cl2 CHCOOH, Cl3 CCOOH
        \choice Cl2 CHCOOH, CH3 COOH, FCH2COOH, Cl3 CCOOH
        \end{choice}

\question Which of these compounds is the ester formed from the reaction of acetic acid and 1- propanol?
OH
        \begin{choices}
\choice |CH3COH|OCH2CH2CH3OH|
        \choice CH3CH2COHOCH2CH3O||
        \choice CH3CH2CH2OCH2COH O||
        \choice CH3COCH2CH2CH3
        \end{choice}

\question What is the name of a base- promoted ester hydrolysis reaction?
        \begin{choices}
\choice Acylation	\choice Esterification	\choice Condensation	\choice Saponification
\end{choice}

\question What is the name of the following compound?
        \begin{choices}
\choice Benzoate ester	\choice Ethyl benzoate   \choice Phenyl butyrate	\choice Ethyl benzyl ketone
\end{choice}
\question Which of the following statements is true about esters?
        \begin{choices}
\choice Esters can form intermolecular hydrogen bonds
        \choice Ester molecules can form intermolecular hydrogen bonds
        \choice Ester molecules cannot form intermolecular hydrogen bonds
        \choice Esters have higher boiling points than alcohols of comparable molecular weight
        \end{choice}

\question The organic compound CH3C(O)CH3 is
        \begin{choices}
\choice Aldehyde	\choice Ester	\choice Carbonyl	\choice Ketone
\end{choice}

\question Consider the following reaction
    CH3CH2  C  O  CH3	+	NaOH 
    What are the products of this reaction?
        \begin{choices}
\choice Sodium acetate and ethanol	\choice Sodium propionate and methanol
        \choice Sodium acetate and methanol	\choice Methyl propionic acid and methanol
        \end{choice}

\question Which catalyst is used in the hydrogenation of vegetables?
        \begin{choices}
\choice Iron	\choice Nickel	\choice Platinum	\choice Molybdenum
\end{choice}

\question What is the name of a base- promoted ester hydrolysis reaction?
        \begin{choices}
\choice Acylation	\choice Esterification	\choice Condensation	\choice Saponification
\end{choice}

\question Which organic functional group does the following molecular representation, i.e., R1R2CHCOH belong? (R1 and R2 represent different alkyl chains)
        \begin{choices}
\choice Amides	\choice Aldehyde	\choice Ethers	\choice Organic acids
\end{choice}

\question What is the IUPAC name for the compound ( CH3)2 CHCH2CHCOH
        \begin{choices}
\choice 2,4- dimethypentanoic acid	    \choice 1- hydroxy- 2,4- dimethylpentanone
        \choice 1,1,3- trimethylbutanoic acid	\choice 2- carboxyisohexane
        \end{choice}

\question Given the following reaction
    What is the major product of the reaction
        \begin{choices}
\choice 2,4- dimethypentanoic acid	    \choice 1- hydroxy- 2,4- dimethylpentanone
        \choice 1,1,3- trimethylbutanoic acid	\choice 2- carboxyisohexane
        \end{choice}

\question What is the process that  converts liquid vegetable oils to solid fats?
        \begin{choices}
\choice Hydration	\choice Hydrogenation	\choice Hydrolysis	\choice Saponification
\end{choice}

\question What would be the solubility of HOCH2 (CH2)6 CH2OH compared to CH3(CH2)6CH2OH?
        \begin{choices}
\choice Less soluble in water	\choice The same solubility in water
        \choice More soluble in water	\choice more soluble in a non- polar solvent such as dichloroethane
        \end{choice}
    
\question Which of the following reactions will produce an akyl carboxylic acid?
        \begin{choices}
\choice Heating a methyl ketone with acid and iodine
        \choice Reacting an alky halide with hydrogen gas and platinum
        \choice Reacting an alcohol with ozone
        \choice Oxidation of a primary alcohol with hot permanganate or chromate
        \end{choice}

\question Which of the following statements is NOT TRUE?
        \begin{choices}
\choice Naturally derived soaps consist of a soluble salt of a long chain fatty acid
        \choice Triacylglyverols are esters of glycerol and long chain carboxylic acids
        \choice Long chain carboxylic acids are also known as fatty acids
        \choice The major acidic components of vinegar is formic acid
        \end{choice}

\question What is the product of the hydrolysis of easters in the presence of a mineral acid catalyst?
        \begin{choices}
\choice alcohol	\choice carbon dioxide	\choice ether	\choice ketones
\end{choice}

\question To which organic functional group does the following molecular representation, i.e., R1R2CHCOH belong? ( R1 and R2 represent different alkyl chains)
        \begin{choices}
\choice Amides	\choice Aldehyde	\choice Ethers	\choice Ketones
\end{choice}

\question Chemically , fats and oils are
        \begin{choices}
\choice acids	\choice alcohols	\choice esters	\choice alkene
\end{choice}

\question Which of the following would react to form pentylethanoate?
        \begin{choices}
\choice 1 - prppanol and pentanoic acid	\choice Ethanol  and pentanoic acid
        \choice 1 - pentanol and ethanoic acid	\choice Ethanol and ethanoic acid
        \end{choice}

\question The difference between fats and oils is that
        \begin{choices}
\choice oils are liquid at room temperature.	\choice oils have more calories
        \choice oils are solid at room temperature	\choice fats are liquid at room temperature
        \end{choice}

\question Which of the following is NOT true about carbonyl compounds?
        \begin{choices}
\choice Carbonyl compounds contain 3a - bond and 1 - bond
        \choice The carbon oxygen bond is both longer a0nd weaker
        \choice The bond angle in carbonyl is about 120
        \choice Carbonyl compounds may  be hydrolyzed
        \end{choice}

\question Consider the following reaction;Which of the following types of compounds are expected products from saponification of a fat?
A Glycerol and fatty acid salts	\choice Glycerol and fatty acids
\choice Fatty acid salts and fatty acids	\choice Glycerol, fatty acid salts and fatty acids
\end{choice}

\question Which of the following statements concerning the carbonyl group in aldehydes and ketones is NOT true?
        \begin{choices}
\choice The bond is polar , with a slight negative charge on the oxygen atom
        \choice The bond angles about the central carbon atom are 1200
        \choice The bond is polar . Therefore , carbonyl groups readily form hydrogen bonds with each other
        \choice In condensed form, the carbonyl group can be written as CHO
        \end{choice}
\question Which of the following statements concerning fats and oils is INCORRECT?
        \begin{choices}
\choice They are also called triacylglycerols	\choice They are also called triglycerides
        \choice They are fatty acids salts	\choice They are glycerol triesters
        \end{choice}

\question Which of the following statements concerning petroleum is INCORRECT?
        \begin{choices}
\choice It is a renewable energy source
        \choice It is a fossil fuel
        \choice It is a mixture consisting mainly of hydrocarbons
        \choice It was formed from marine organisms, which died millions of years ago
        \end{choice}

\question Triglycerides ( fats and oils ) are made up of
        \begin{choices}
\choice sugars and water	\choice glycerol and amino acids
        \choice fatty acids and glycerol	\choice water , glycerol and salt
        \end{choice}

\question The compound shown below is derived from- - - - - - - and - - - - - - - It is called- - - - - - - - - - - - - -
        \begin{choices}
\choice propanol, benzoic , propyl benzoate	\choice ethanol, benzoic acid , ethybenzoate
        \choice ethanol, benzol, phenyl butyrate	    \choice ethanol, benzol, ethybenzoate
        \end{choice}

\question What is the correct name of the following compound?
        \begin{choices}
\choice 2- aminopropanoic acid	\choice 3- aminobutanoic acid
        \choice 2- aminobutanoic acid	\choice 3- aminopropanoic acid
        \end{choice}

\question When reacts with NaOH, the product  is sodium benzoate.
        \begin{choices}
\choice Benzoic acid	\choice Benzaldehde	\choice Benzene	\choice Benzoic hydroxide
\end{choice}

\question The reaction between alcohol and acyl chlorides produce……………
        \begin{choices}
\choice Ether	\choice Carboxilic acids	\choice Aromatic salts	\choice Eester
\end{choice}

\question Fats and oils are:
        \begin{choices}
\choice Esters	\choice Alcohols	\choice Acids	\choice Alkanes
\end{choice}

\question Compounds that contain the carboxyl and hydroxyl) group are said to be:
        \begin{choices}
\choice Ester	\choice Ketones	\choice Organic acids	\choice Aldehydes
\end{choice}

\question An ester has the structural formula
O
CH3CH2CH2C
OCH2CH3
On hydrolysis, the ester would produce:
        \begin{choices}
\choice Propanoic acid and propan- 1- ol	\choice Butanoic acid and ethanol
        \choice Ethanoic acid butan- 1- ol	\choice Propanoic acid and ethanol
        \end{choice}
\question Which of the following compounds would be the most stable in H2O?
        \begin{choices}
\choice Ethane	\choice Pentane	\choice Octhanoic acid	\choice Ethanoic acid
\end{choice}

\question Which acid is produced when toluene is subjected to KMnO4 oxidation?
        \begin{choices}
\choice Toluic acid	\choice Benzoic acid	\choice Phenyl acetic acid	\choice Phthalic acid
\end{choice}

\question Which of the following is an organic acid?
        \begin{choices}
\choice CH3CO2H	\choice CH3CH2OH	\choice CH2=CH2	\choice CH3CH3
\end{choice}

\question During esterfication of carboxylic acid with alcohol which bond of carboxylic acid undergoes cleavage?
        \begin{choices}
\choice C- C	\choice C= O	\choice O- H	\choice C- O
\end{choice}

\question Hydrolysis of ester leads to the formation of which of the following products in basic medium?
        \begin{choices}
\choice Alcohol and sodium carbonate	\choice Ether and alcohol
        \choice Aldehyde and alcohol	\choice Sodium carboxylate
        \end{choice}

\question Fats and oils can be classified as	- - .
        \begin{choices}
\choice carbohydrates	\choice Acids	\choice alcohols	    \choice esters
\end{choice}

\question What is the IUPAC name for the following carboxylic acid?	CH3- C- CH2- C- OHCH3
        \begin{choices}
\choice 2- dimethylbutanoic acid	\choice 3,3- Dimethylbutanoic acid
        \choice 2- methylpentanoic acid	\choice 3- methylpentanoic acid
        \end{choice}

\question Acetylsalicylic acid (aspirin) has the structural formula:	 Which functional group (groups)is (are) present in aspirin?
        \begin{choices}
\choice Carboxyl and ester	\choice hydroxyl and carbonyl	\choice carboxyl and acetyl	\choice Hydroxyl
\end{choice}

\question Which of these compounds is propanoic acid ?
        \begin{choices}
\choice CH3 CH2COOCH3	\choice CH3CH2COH	\choice CH3CH2COOH	\choice CH3CH2CH2OH
\end{choice}








Chemistry grade- 12 Entrance exam Chapter- 1

\question Which of the following statement(s) is (are) true of an ideal liquid- liquid solution?
        I. It obeys pV=nRT
        II. It obeys Raoult’s law
        III. Solute- solute, solvent- solvent, and solute –solvent interactions are very similar
        IV. Solute- solute, solvent- solvent, and solute- solvent interactions are quite different
    \begin{choices}
\choice I, II and III	\choice I, II and IV	\choice II and III	\choice II and IV
\end{choice}

\question Butane burns in oxygen according to the equation below. 2C4H10(g) + 13O2(g)	8CO2(g) + 10H2O(1)
If 11.6g of butane is burned in 11.6g of oxygen, which is the limiting reagent?
    \begin{choices}
\choice Butane	\choice Oxygen	\choice Neither	\choice Both oxygen and butane
\end{choice}

\question A beaker filled to the 100mL mark with salt ( the salt has a mass of 100g) and another beaker to the 100mL mark with water ( the water has a mass of 100g) are mixed together in a bigger beaker unit the salt is completely dissolved. What will be the mass of the solution?
    \begin{choices}
\choice It will be much more than 200g	\choice It will be much smaller than 200g
    \choice It will be exactly 200g	\choice It will be slightly more than 200g

\question A solution is made by dissolving 250.0g of potassium chromate crystals ( k2CrO4, molar mass, 194.2g) in 1.00kg of water. What will be the freezing point of the solution? ( kf for water is 1.860c/molal).
    \begin{choices}
\choice - 8.87 0C	\choice - 7.180c	\choice - 5.730c	\choice - 1.860c
\end{choice}

\question How many moles of sodium hydroxide are present in 2.5L of 0.5 M aqueous solution? 
    \begin{choices}
\choice 0.2	\choice 0.5	\choice 1.25	\choice 12.5
\end{choice}

\question If the solute- solvent interactions are greater than the solute- solute and solvent- solvent interactions, what will be the total vapor pressure of the solution?
    \begin{choices}
\choice Greater than that calculated from Raoult’s law
    \choice Less than that calculated from Raoult’s law
    \choice The same as calculated from Raoult’s law
    \choice Raoult’s law cannot be applied for such interactions
    \end{choice}

\question What volume of 0.5000M NaOH is required to neutralize 25.0mL of 1.2 M H2SO4? ( assume complete ionization of the acid).
    \begin{choices}
\choice 60mL	\choice 90mL	\choice 100mL	\choice 120mL
\end{choice}

\question An aqueous solution is 70.0% nitric acid (HNO3) by mass. What is the concentration of HNO3 expressed in molality?
    \begin{choices}
\choice 0.559m	\choice 8.62m	\choice 11.1m	\choice 37.0m
\end{choice}

\question A lab instructor is preparing 5.0 liters of a 0.10 M Pb(NO3)2 ( Molecular mass 331) solution. What is the mass required?
    \begin{choices}
\choice 165.5g of Pb (NO3)2 and add 5.0kg of H2O
    \choice 165.5g of Pb(NO3)2 and add H2O until the solution has a volume of 5.0liters
    \choice 33.1g of Pb(NO3)2 and add H2O until the solution has a volume of 5.0 liters
    \choice 33.1g of Pb(NO3)2 and add 5.0 liters of H2O
    \end{choice}
\question What would be the solubility of HOCH2 (CH2)6 CH2OH compared to CH3(CH2)6 CH2OH?
    \begin{choices}
\choice Less soluble in water	\choice The same solubility in water
    \choice More soluble in water \choice More soluble in a non- polar solvent such as dichloromethane
    \end{choice}

\question What is the mass of one molecule of water?
    \begin{choices}
\choice 3.0 x 10- 23g	\choice 0.0003g	\choice 1.8 x 10 −22g	\choice 18.0g
\end{choice}

\question Which of the following is the most important type of solute- solvent interaction in a solution of n- butanol in water?
    \begin{choices}
\choice Dispersion	\choice Ion – dipole	\choice Dipole – dipole	\choice Hydrogen bonding
\end{choice}

\question Which of the following statements is TRUE about colligative properties?
    \begin{choices}
\choice Both vapor pressure & freezing point increase when a non- volatile solute is added to a solvent
    \choice Both freezing point and boiling point increase when a non- volatile solute is added to a solvent
    \choice Both vapor pressure and boiling point decrease when a non- volatile solute is added to a solvent
    \choice Colligative properties depend only upon the number of solute particles in a solution and not upon their identity
    \end{choice}

\question What is the equivalent weight of HNO3, as an oxidizing agent, in the following balanced reaction? 3Fe2+ + 4H+ + NO3 −		3Fe3+ + NO + 2H2O
    \begin{choices}
\choice 10.50	\choice 15.75	\choice 21.00	\choice 31.50
\end{choice}

\question What is the number of chloride ions (Cl- ) present in 1.0 x 10- 5 mol of AlCl3?
    \begin{choices}
\choice 1.80 x 1019	\choice 6.02 x 1018	\choice 6.02 x 1023	\choice 6.02 x 1028
\end{choice}

\question A solution was prepared by adding 48g of methanol (CH3OH) into 81g of water (H2O). What is the mole fraction of methanol in this solution?
    \begin{choices}
\choice 0.25	\choice 0.75	\choice 1.5	\choice 4.
\end{choice}

\question A solution was prepared by dissolving 3.75g of pure hydrocarbon in 95.0g of cyclohexane. The boiling point of pure cyclohexane was observed to be 80.700c and that of the solution was 81.450c. What is the approximate molecular weight of the hydrocarbon?
(Kb for cyclohexane = 2.790c/m)
    \begin{choices}
\choice 71.0g/mol	\choice 105 g/mol	\choice 147 g/mol \choice 312 g/mol
\end{choice}

\question How many mL conc. HNO3 and how many mL of water are required to prepare 500mL of 0.1 M HNO3 from a conc.13M HNO3?
    \begin{choices}
\choice 1mL HNO3 and 496.15mL H2O	\choice 3.85mL HNO3 and 500mL H2O
    \choice 3mL HNO3 and 500mL H2O	\choice 3.85mL HNO3 and 496.15mL H2O
    \end{choice}

\question Which one of the following organic molecules has the highest water solubility?
    \begin{choices}
\choice HOCH2CH2CH2OH	\choice HOCH2CH2CH2CH2OH
    \choice CH2CH2CH2CH2OH	\choice CH3CH2CH2CH2OH
    \end{choice}

\question Which one of the following substances is a non- conductor of electricity?
    \begin{choices}
\choice   Graphite	\choice MgCl2(s)	\choice Silver (s)	\choice H2SO4 (aq)
\end{choice}

\question Which of the following is Not a solution?
    \begin{choices}
\choice Milk	\choice Brass	\choice Whisky	\choice Coca cola drink
\end{choice}

\question How much water has to be evaporated from 250 ml of 1 M Ca (OH)2 to make it 3M 
    \begin{choices}
\choice 100 ml	\choice 150 ml	\choice 167 ml	\choice 200 ml
\end{choice}

\question How many ml of water is required to dilute 50 ml of 3.5 M H2SO4 to 2.00 M H2SO4? 
    \begin{choices}
\choice 37.5	\choice 45	\choice 75	\choice 87.5
\end{choice}

\question The solubility of sodium selenite, Na2SeO4, is 84g/100g of water at 350\choice If a solution is obtained by dissolving 92 g of Na2SeO4 in 200g of water at 350C, what do you call this solution?
    \begin{choices}
\choice Diluted	\choice Saturated	\choice Unsaturated	\choice Supersaturated
\end{choice}

\question Which law relates the concentration of a dissolved gas, Cg, to its partial pressure?
    \begin{choices}
\choice Henry’s law	\choice Raoult’s	\choice Boyle’s law	\choice Ideal gas law
\end{choice}

\question Which of the following compounds would give the lowest freezing point depression when 100 g of each are dissolved in 1 kg of water (K, for water = 1.860C/m)? Assume complete dissociation.
    \begin{choices}
\choice NaCl	\choice NH4NO3	\choice (NH4)2SO4	\choice glucose, C6H12O6
\end{choice}

\question Which of the following is most likely to deviate from ideal gas behavior?
    \begin{choices}
\choice He	\choice Ar	\choice Cl2	\choice CCL2F2
\end{choice}

\question What is the molarity of a solution containing 10g of sulfuric acid in 500ml of solution?
    \begin{choices}
\choice 0.02	\choice 0.03	\choice 0.12	D.0.2

\question Which of the following types of solutions are possible?
        I. Solid dissolved in a liquid	III. Gas dissolved in a liquid
        II. Gas dissolved in a gas	IV. Solid dissolved in a gas
    \begin{choices}
\choice I and II	\choice I, II,III and IV	\choice I,ll and IV	D.I
    \question What is the normality of l.0M solution ofNa2C03?
    \begin{choices}
\choice 1N	\choice 0.5N	C.2N	D.3N

\question What type of solute- solvent interaction should be the most important in a solution of iodine in carbon tetrachloride?
    \begin{choices}
\choice London forces	\choice Ionic bond	\choice Ion- dipole forces	\choice Dipole - dipole forces
\end{choice}

\question A liquid is any substance of biochemical orgin that is
    \begin{choices}
\choice soluble in both water and non polar solvents
    \choice insoluble in both water and non- polar solvents
    \choice soluble in water but insoluble in non- polar solvents
    \choice soluble in non- polar solvents and insoluble in water
    \end{choice}

\question What is the molarity of a 5 g hydrogen peroxide ( H2O2) in 100 ml. solution that is used for their bleaching? 
    \begin{choices}
\choice 0.015M	\choice \question 15M	\choice 1.5M	\choice 3M
\end{choice}

\question If a student wishes to prepare approximately 100 milliliters of an aqueous solution of 6M HCl using 12 M HCl, which proceure is correct?
    \begin{choices}
\choice Adding 50 ml. of 12 HCl to 50 ml. of water while stirring the mixture steadily.
    \choice Adding 25 ml .of 12M HCl to 50 ml. water while stirring the mixture steadily
    \choice Adding 50ml. of water to 50ml. of 12 M HCl while stirring the mixture steadily
    \choice Adding 25 ml. of water to 50ml. of 12 M HCl while stirring the mixture steadily
    \end{choice}

\question What kind of solution forms when gasoline evaporates in air?
    \begin{choices}
\choice Gas in gas so/n \choice Gas in liquid so/n	\choice Liquid in liquid so/n	\choice Liquid in gas so/n
\end{choice}

\question What is the solvent in 70% alcohol solution
    \begin{choices}
\choice Water	\choice Alcohol	\choice Sugar	\choice Kerosene
\end{choice}

\question How many moles of H2SO4 are needed to prepare 5.0 liters of a 2.0 M of H2SO4
    \begin{choices}
\choice 2.5	\choice 5.0	\choice 20	\choice 10
\end{choice}

\question What is the balanced NET IONIC EQUATION for the reaction of CaCl2(eq) and AgNO3?
    \begin{choices}
\choice CaCl2(aq)	+	2AgNO3(aq)	a(NO3)2(aq)	+	2AgCl(s)
    \choice Ca2+ (aq) + 2Cl- (aq) + 2Ag+ (aq) + 2NO - (aq)	 Ca2+ (aq) + 2NO3- (aq) + 2AgCl(s)
    \choice Cl(aq)	+ Ag+(aq)		2AgCl(s)
    \choice 2Cl (aq)	+	2Ag+ (aq)		2AgCl(s)
    \end{choice}

\question When a small amount of crystal solute is added to the supernatural solution , the solute crystal will
    \begin{choices}
\choice grow bigger	\choice slightly dissolve	\choice remain unchanged	\choice dissolve completely
\end{choice}

\question What is the molality of a solution that contains 51.2 g of naphthalene, C10H8 , in 500 mL of carbon tetrachloride ? The density of CCl4 is 1.60 g/mL.
    \begin{choices}
\choice 0.750m	\choice 0.500m	\choice 0.840m	\choice 1.69 m
\end{choice}

\question Which of the following does NOT affect the solubility of a gas dissolved in a liquid?
    \begin{choices}
\choice Nature of solute and solvent	\choice Pressure
    \choice Temperature	\choice Rate at which the gas dissolves
    \end{choice}

\question 	Equal masses of He and Ne are placed in a sealed container . What is the partial pressure of Ne, if the total pressure is 6 atm?
    \begin{choices}
\choice 2	\choice 3	\choice 1	\choice 5
\end{choice}

\question What is the morality of a solution made by dissolving 10 g of glucose (C6H12O6) in sufficient water to form 300
mL solution?	
    \begin{choices}
\choice 0.18	\choice 0.251	\choice 0.362	\choice 0.278
\end{choice}

\question What is the molar solubility of Fe(OH)3  in a solution that is buffered at pH = 3.50 at 25
0C?
(Ksp (Fe(OH)3 = 4 x 10- 38)
    \begin{choices}
\choice 1x10- 5	\choice 1.1 x 10- 6	\choice 2.0 x 10- 6	\choice 1.26 x 10- 6
\end{choice}

\question Dissolve each of NaI, CuSO4, KMnO4 . KNO3 in different 200 mL measuring cylinders. Which one of the
following forms more concentrated  molar solution?
    \begin{choices}
\choice KNO3	\choice NaI	\choice KMnO4	\choice CuSO4
\end{choice}

\question Consider the following compounds having lattice energy of
Compound
NaOH
Mg(OH)2
MgO
Al(OH)3
Lattice energy (KJ/mol0
900
3006
3791
5627
Which one is insoluble in water?
    \begin{choices}
\choice Al(OH)3	\choice MgO	\choice Mg(OH)2	\choice NaOH
\end{choice}

\question 	At 70 0C , the vapour pressure of pure water is 39 kPa. Which one of the following is the most likely vapour pressure for a 1.5 M solution of sucrose (aq) at the temperature?
    \begin{choices}
\choice 37kPa	\choice 39kPa	\choice 41 kPa	\choice 45 kPa
\end{choice}

\question What is the concentration of nitrate ion (NO3)- in a solution that contains 0.5 M Al (NO3)3 ? \begin{choices}
\choice 0.5 M	\choice 1 M	\choice 1.5 M	\choice 2.5 M
\end{choice}

\question A 500 mL of 0.1 M nitric acid solution (HNO3) is to be prepared from a 13 M concentrated nitric acid (HNO3) . How many mL of concentrated HNO3 and how many mL of water are needed?
    \begin{choices}
\choice 3.85 mL conc. HNO3 and 496.15 mL HNO3	\choice 15 mL conc. HNO3 and 485 mL H2O
    \choice 30 mL conc. HNO3 and 470 mL H2O	        \choice 13 mL conc. HNO3 and 487mL H2O
    \end{choice}
    
\question The figure below shows the solubilities of several ionic solids as a function of temperature.
    \begin{choices}
\choice 3.85 mL conc. HNO3 and 496.15 mL HNO3	\choice 15 mL conc. HNO3 and 485 mL H2O
    \choice 30 mL conc. HNO3 and 470 mL H2O	        \choice 13 mL conc. HNO3 and 487mL H2O
    \end{choice}

\question 	A sample of potassium nitrate (49.0g) is dissolved in 100 g of water at 100 0C with precautions taken to avoid evaporation of any water. The solution is cooled to 30.0 0C and no precipitate is observed . This solution is- - - -
    \begin{choices}
\choice Supersaturated	\choice Saturated	\choice Unsaturated	\choice Hydrated
\end{choice}

\question What is the molarity of sodium chloride in solution that is 13.0 % by mass sodium chloride and that has a density of 1.10g/ml?
    \begin{choices}
\choice 1.43 x 10 - 2	\choice 2.23	\choice 1.22	\choice 2.45
\end{choice}

\question Which opposing processes occur in a saturated solution?
    \begin{choices}
\choice Vaporization and condensation	\choice Dissociation and crystallization
    \choice Dissociation and reduction	    \choice Oxidation and reduction
    \end{choice}

\question Compounds A and B are combined in a mole ratio of 0.30 to 0.70 respectively . At a given temperature, the pure vapor pressure of compound	A is given to be 100 torr and the pure vapor pressure of B is 50 torr.
What will be the total pressure above the solution?
    \begin{choices}
\choice 85 torr	\choice 70 torr	\choice 65 torr	\choice 55 torr
\end{choice}

\question Ammonium sulphate ( NH4)2 SO4 is manufactured by reacting sulphuric acid with ammonia as follows
H2SO4(aq) +	2NH3(aq)	(NH4)2SO4(aq)
What volume of 0.80 M H2SO4 is needed to react with 200 mL of 1.2 M ammonia solution to prepare the required salt, (NH4)2 SO4) ?
    \begin{choices}
\choice 0.40 L	\choice 0.30 L	\choice 0.150 L	\choice 0.0244 L
\end{choice}

\question Which of the following aqueous solutions will have the Lowest freezing point?
    \begin{choices}
\choice Pur H2O	\choice aq.0.50m KF	\choice aq.0.24m FeI3	\choice aq. 0.60m glucose
\end{choice}

\question A……………H corresponds to an	process.
    \begin{choices}
\choice Positive, endothermic	\choice Negative, endothermic
    \choice Positive exothermic	\choice Zero, exothermic
    \end{choice}

\question You are given a bottle of solid X three aqueous solutions of Y , the first saturated,the second unsaturated and
the third supersaturated. Which of the following is correct, if you add the small amount of the solid solute to each solution?
    \begin{choices}
\choice The solution in which the added solid solute dissolves is the saturated solution.
    \choice The supersaturated solution is unstable and addition of additional solute causes the excess solute to crystallize.
    \choice The solution in which the added solid solute remains undissolved is the unsaturated solution.
    \choice In all the three solutions ; saturated,unsaturated and supersaturated the added solidsolute will dissolve.
    \end{choice}

\question The phrase like dissolve like refers to the fact that:
    \begin{choices}
\choice Polar solvents dissolve polar solutes and non polar solvents dissolve non polar solutes
    \choice Polar solvents dissolve nonpolar solutes and vice versa
    \choice Solvents can only dissolve solutes of similar molar mass
    \choice Gases can only dissolve other gases.
    \end{choice}
\question 2.3g of ethanol(CH3 CH2OH) is added to 500 g of water. What is the molality of the resulting solution? \begin{choices}
\choice 0.01m	\choice 1.0m	\choice 0.1m	\choice 10.0m
\end{choice}

\question A 0.5L and 0.1MHNO3 solution is to be prepared by dilution process from a 13M nitric acid. How many Ml con. HNO3 and how many mL of water are required to prepare the dilute solution?
    \begin{choices}
\choice 3.85mL HNO3 and 496.15mLwater	\choice 10ml HNO3 and 490mLwater
    \choice 2mL HNO3 and 498mLwater	\choice 20ml HNO3 and 490ml water
    \end{choice}

\question Commercial concentration sulphuric acid (density=1.831g/cm3) is 94.0% H2SO4, by mass. What is the normality of H2SO4 solution?(Molar mass of H2SO4=98g/mol)
    \begin{choices}
\choice 16.8M	\choice 28.2	\choice 40.4 M	\choice 35.0N
\end{choice}

\question What is the final temperature when 150.0ml of water at 90.0oc is added to 100.0ml of water at 30.0oc? 
    \begin{choices}
\choice 33.0oc	\choice 45.0oc	\choice 66.0oc	\choice 60.0oc
\end{choice}

\question What is the PH of a mixture of 15.0ml of 0.26M NaOH and 21.0M H2SO4? 
    \begin{choices}
\choice 1.70	\choice 13.60	\choice 11.81	\choice 2.15
\end{choice}

\question Which of the following is true regarding the solution formation process?
    \begin{choices}
\choice Intermolecular force between the solute particles must weaken in which the enthalpy change is exothermic (H<0).
    \choice Intermolecular forces between the solvent molecules must weaken in which the enthalpy change is exothermic (H<0).
    \choice Covalent bonds within the solute and solvent molecules must be broken.
    \choice New columbic attractions between the solute and the solvents from in which the enthalpy change is exothermic (H<0)
    \end{choice}

\question What volume of 1.40MH2SO4 solution is needed to react exactly with 10.0 g of aluminum according to the following reaction?	2Al(s) + 2H2SO4(aq)	Al2(SO4)3(aq) + 3H2(g)
    \begin{choices}
\choice 2.643ml	\choice 26.43ml	\choice 2643ml	\choice 264.3ml
\end{choice}

\question A solution of NH4Cl made by dissolving 3.16g NH4Cl in 30.14 g H2O has a density of 1.0272g/cm
    \question What is the mole fraction of NH4Cl?
    \begin{choices}
\choice 0.0341	\choice 0.9659	\choice 0.6500	\choice 0.2100
\end{choice}

\question Is the standard cell potential for the oxidation of ammonia, given below? 4NH3 +   3O2	2N2    +   6H2O ,	G=- 1356KJ
    \begin{choices}
\choice 3.51V	\choice 1.17V	\choice 7.02V	\choice 14.04V
\end{choice}

\question .   What is the freezing point of the solution of 250g of CaCl2 in 1.0kg of water?( Kf for H2O =1.860c/m)	\begin{choices}
\choice - 1.30c	\begin{choices}
\choice - 130c	\choice - 9.00c	\choice - 6.50c
\end{choice}
    \question Which of the following compounds is least soluble in water?
    \begin{choices}
\choice (NH4)2CO3	\choice (Na3(PO4)	\choice (Fe(NO3)3	\choice BaCO3
\end{choice}

\question 	Given the	following unbalanced equation KMnO4 + KI + H2SO4	K2SO4 + MnSO4 + I2 + H2O
How many grams of KMnO4 are needed to make 250ml of 0.20N solution?
    \begin{choices}
\choice 3.95g	\choice 1.58g	\choice 2.98g	\choice 3.16g
\end{choice}

\question If a solution of acetic acid (CH3COOH) has a PH of 3.00, what is its concentration? Ka of acetic acid=1.74x10- 5
    \begin{choices}
\choice 0.0057M	\choice 0.57M	\choice 0.057M	\choice 5.70M
\end{choice}

\question The indicator methyl red in a solution of NaH2PO
    \question Which of the following eqution is consistent with this observation?
    \begin{choices}
\choice H2PO4- + H2O	H3PO4 + OH-	\choice HPO42- + H2O	PO 3- + H3O+
    \choice HPO42- +H2O	H2PO4- + OH-	\choice H2PO4-   + H2O	HPO42- + H3O+
    \end{choice}

\question A chemist creates a buffer solution by mixing equal volume of a 0.2 M HOCl solution and a 0.2M KOCl solution. Which of the following will occur when a small amount of KOH is added to the solution ?
        I. The concentration of undissociated HOCl will increase
        II. The concentration of OCl- ions will increases
        III. The concentration of H+ ions will increase.
    \begin{choices}
\choice I only	\choice I and II only	\choice III only	\choice II only
\end{choice}

\question In which of the following cases will the dissolution of sugar be the most rapid?
    \begin{choices}
\choice Powdered sugar in hot water	\choice Sugar crystals in cold water
    \choice Sugar crystals in hot water	\choice Powdered Sugar in cold water
    \end{choice}

\question How many grams of iodine, I2, must be dissolved in 225.0ml of carbon disulfide, CS2 (density= 1.261g/cm3), to produce a 0.116m solution?
    \begin{choices}
\choice 4.84g	\choice 6.32g	\choice 11.71g	\choice 4.17g
\end{choice}

\question 4 L of 0.02M of aqueous solution of NaCl is diluted with 1L of water. What is the molarity of the solution?
    \begin{choices}
\choice 0.004M	\choice 0.016M	\choice 0.012M	D.0.008

\question How much water, in liters, must be added to 0.50liter of 6.0MHCl to make the solution 2.0M? 
    \begin{choices}
\choice 0.50	\choice 2.0	\choice 3.0	\choice 2.0
\end{choice}

\question Which of the four colligative properties arises in systems where there is equilibrium between a liquid solution phase and a second liquid phase?
    \begin{choices}
\choice Vapour pressure lowering	\choice Boiling point elevation
    \choice Osmotic pressure     	\choice Freezing point depression
    \end{choice}


Chemistry grade 12 Entrance
Chapter- 2

\question Which of the following is NOT a conjugate acid- base pair?
        \begin{choices}
\choice HNO3/NO -	\choice H2SO4/HSO4-	\choice NH3/NH -	\choice H3O+/OH-
\end{choice}

\question During the titration of a know volume of a strong acid with a strong base, there is
        \begin{choices}
\choice A steady increases in PH	\choice steady decrease in pH
        \choice A sharp increase in pH around the end point	\choice A sharp decrease in pH around the end point
        \end{choice}

\question A solution with pH of 7.5 would be described as:
        \begin{choices}
\choice Very basic \choice Slightly basic	\choice Slightly acidic	\choice Very acidic
\end{choice}

\question Which species CANNOT act as a Lewis acid?
        \begin{choices}
\choice NH3	\choice BF3	\choice Fe2+ \choice AlCl3
\end{choice}

\question Which of the following statements is true?
        \begin{choices}
\choice A universal indicator is a mixture of indicators that will give a different colour for a different pH
        \choice Phenolphthalein is a universal indicator
        \choice A universal indicator can only be used in either strongly acidic or basic solution
        \choice The colour of a universal indicator is red in a weak acid
        \end{choice}

\question Three acids, HA, HB, HC have the following ka values.
ka (HA) = 1x10−5 ka(HB) = 2x10- 5 ka (HC) = 1x10- 6
What is the correct order of increasing acid strength (weakest first)?
        \begin{choices}
\choice HA, HB, HC	\choice HC, HB, HA	\choice HC, HA, HB	\choice HB, HA, HC
\end{choice}

\question Which of the following procedures will produce a buffered solution?
    I. Equal volumes of 0.5M NaOH and 1 M HCl solutions are mixed
    II. Equal volumes of 0.5 M NaOH , 1.0M CH3 COOH solutions
    III. Equal volumes of 1 M NaCH3 CO2 and 1 M CH3 COOH
    IV. Equal volumes of 1.0 M NaOH and 1,0M HCl solutions are mixed
        \begin{choices}
\choice I	\choice III	\choice I and II	\choice II and III
\end{choice}

\question Which of the following statements is true about the percent ionization of a weak acid?
        \begin{choices}
\choice Its decrease with increasing concentration	\choice It increases with increasing concentration
        \choice It increases with decreasing concentration	\choice It decreases with decreasing concentration
        \end{choice}

\question Which one of the mixture of the following pairs will NOT give a buffer solution?
        \begin{choices}
\choice HCN and NaCN	\choice NH3 and NH4Cl	\choice H3PO4 and KH2 PO \choice HNO3 and NaNO3
\end{choice}

\question Which one of the following is TRUE for salts formed from strong acids and strong bases?
        \begin{choices}
\choice No hydrolysis takes place	\choice Produces ions which are proton acceptors
        \choice Produces ions that are proton donors	\choice Depends on the pKa and pKb of the parent acids and bases
        \end{choice}

\question What is the quantity of water, in mL, required to prepare 0.5M of HCl from a concentrated solution of 3.5 M in 50mL is?
        \begin{choices}
\choice 50mL	\choice 100mL	\choice 300mL	\choice 350mL
\end{choice}

\question What is the pH of 0.005 M solution of Ca(OH)2?
        \begin{choices}
\choice 2.3 \choice 10	\choice 12	\choice 14
\end{choice}

\question Given the following equilibria and equilibrium constant:
I.	HC2H3O2 + H2O ⇄ H3O+ + C2H3O2- ; Ka = 1.80x10- 5
II.	H2CO3 + H2O ⇄ H3O+ + HCO3-; Ka = 4.20x10- 7
III.	NH + + H2O ⇄ H3O+ + NH3; Ka = 5.6x10- 10
IV.	HCOOH + H2O ⇄ H3O+ H3O+ + HCOO − Ka = 1.80x10- 4
What is the strength of the acids in DECREASING order?
        \begin{choices}
\choice I, IV, II and III	\choice II, III, IV and I   \choice III, II, I, and IV	\choice IV, I, II and III
\end{choice}

\question Given the reaction:
H2PO - + H2O	H3O+ + HPO42-
Which of the following represents a conjugate acid- base pair?
        \begin{choices}
\choice H2PO - and H2O \choice H2PO - and HPO42   \choice H2PO - and H3O-	\choice H2O and HPO42
\end{choice}

\question The indicator Bromthymol Blue (HBb) is a weak acid with Ka = 1.0 x10- 7 ionizes as follows:-	HBb(aq, yellow) ⇌ H+(aq, colourless) + In –(aq, blue)
Which way will the equilibrium shift when NaOH is added and what will the colour of the NaOH solution be containing this indicator?
        \begin{choices}
\choice Equilibrium will shift to the right and the color of NaOH solution will turn green
        \choice Equilibrium will shift to the right and the color of NaOH solution will turn blue
        \choice Equilibrium will shift to the left and the color of NaOH solution will turn yellow
        \choice Equilibrium will shift to the left and the color of NaOH solution will turn blue
        \end{choice}

\question 	A 50 mL solution of H2SO4 of 0.205 M is titrated with NaOH solution of unknown concentration. The endpoint against phenolphthalein indicator was signed when 41.0mL of NaOH was added. What is the concentration of NaOH solution?
        \begin{choices}
\choice 0.10M	\choice 0.25M	\choice 0.41M	\choice 0.50M
\end{choice}

\question To 0.2M solution of a weak monoprotic acid, HA, enough quantity of its sodium salt, NaA, was dissolved to give a concentration of 0.2M of the salt. What will be the acid concentration, [ H3O+] , in the final solution? ( Ka of HA =1.80 x 10- 5)	
        \begin{choices}
\choice 3.60 x 10- 6 M  \choice 1.00 x 10- 5M   \choice 1.80 x 10- 5M	\choice 1.90 x 10- 3M
\end{choice}

\question What is the pH of an aqueous solution prepared to contain 1.3 x10- 3 M sodium nitrite (NaNO2) if the acid dissociation equilibrium constant, Ka, for nitrous acid (HNO2) is
5.1 x 10- 4?Kw = 1.0 x 10- 14
        \begin{choices}
\choice 3.1	\choice 5.1	\choice 6.0	\choice 7.3
\end{choice}

\question A solution is labeled 0.500 M HCl. What is PH?
\begin{choices}
\choice 0.5	\choice 1.5	\choice 1.6	\choice 1.69
\end{choice}
    
\question Which of the following compounds would be the most basic?
        \begin{choices}
\choice 0.1 M acetic acid	\choice 0.1 M hydrochloric acid
        \choice 0.1 M sodium acetate	\choice 0.1 M ammonium chloride
        \end{choice}

\question How is a buffer solution prepared?
        \begin{choices}
\choice By mixing a weak acid and a strong acid
        \choice By mixing a weak acid and its conjugate base
        \choice By mixing a strong acid and its conjugate base
        \choice By mixing a strong base and its conjugate acid.
        \end{choice}

\question The dye bromothymol blue (HBb) is a weak acid whose ionization can be represented as follows, HBb(aq) ⇌ H+ (aq) +Bb- (aq)	Which way will the equilibrium shift when NaOH is added?
        \begin{choices}
\choice To the left	\choice nitially to the left and after a while to the right
        \choice To the right	\choice Initially to the right and after a while to the left
        \end{choice}

\question Which one of the following statements is NOT true about acids
        \begin{choices}
\choice An acid is a proton donor	\choice An acid is an electron pair acceptor
        \choice An acid is an electron pair donor	\choice An acid is a substance that is ionized in water to produce H+(aq)
        \end{choice}

\question Consider the following acids:
I.	CH3COOH, Ka=1.8x10- 5	III. HCO2H, Ka=1.8x10- 4
II.	HOBr,	Ka=2.4x10- 9	V. C6H5OH, Ka=1.0x101- 10
Which of the following aqueous solutions will have the highest pH?
        \begin{choices}
\choice 0.10 M NaOBr	\choice   0.10 M HCO2Na	\choice 0.10 M C6H5ONa	\choice 0.10 M CH3COONa
\end{choice}

\question What is the hydroxide ion concentration for a solution with a pH of 10 at 250C? 
        \begin{choices}
\choice 10- 14M	\choice 10- 10M	\choice 10- 7 M	\choice 10- 4M
\end{choice}

\question Which of the following titrations will have an equivalence point at a pH< 7
        \begin{choices}
\choice Strong acid with weak base	\choice Strong acid with strong base
        \choice Weak acid with weak base	\choice Weak acid with strong base
        \end{choice}

\question All of the following can act as Bronsted- Lowry bases EXCEPT
        \begin{choices}
\choice 1-	\choice NH3	\choice HCO- 3	\choice H2CO3
\end{choice}

\question Which of the following is a conjugate acid/base pair?
        \begin{choices}
\choice HCl/OCl	\choice H3O+/OH-	\choice NH +/NH3	\choice H2SO4/SO 2
\end{choice}

\question The pH at room temperature of a 0.1 M solution of formic acid (HCHO2) was measured to be	4.\question What is the hydrogen ion concentration?
        \begin{choices}
\choice 3.16x10- 5M	\choice 3.16x10- 12M	\choice 6.32x10- 3M	\choice 6.32x10- 4M
\end{choice}
    
\question In which of the following period of the periodic Table is an element with atomic number 20 placed? 
        \begin{choices}
\choice 1	\choice 2	\choice 3	\choice 4
\end{choice}

\question Which of the following substances undergoes hydrolysis in aqueous solution?
        \begin{choices}
\choice HCl	\choice HC2H302	\choice ~OH	\choice NH4Cl
\end{choice}

\question The pH of 0.1M solution of a weak acid is 3.\question What is the value of the ionization constant for the acid? 
        \begin{choices}
\choice 10- 7	\choice 10- 5	\choice 10- 3	\choice 0.1
\end{choice}

\question An amphiprotic species is a molecule or ion that can?
        \begin{choices}
\choice Accept protons	\choice both accept and donate protons
        \choice Donate more than one protons	\choice be formed into a double ion.
        \end{choice}

\question Which one is true for a triprotic acid, such as phosphoric acid, H3PO4?
        \begin{choices}
\choice Kal>Ka2>Ka3	\choice Ka3>Ka2>Kal	\choice Kal>Ka2 = Ka3	\choice Ka, = Ka2 = Ka3
\end{choice}

\question Which of the following is true for a 0.10M solution of a weak base HB? 
        \begin{choices}
\choice [ H+] =0.10M	\choice [ B- ] = 0.10M	\choice [ HB] >[ H+] \choice pH= 1.0
\end{choice}

\question What is the molarity of a solution obtained by dissolving 0.01moles of NaCl is 500ml	of solution?
        \begin{choices}
\choice 0.01M	B.0.005M	C.0.02M	D.0.10M

\question Which of the following is true concerning this acid- base reaction?
HS- + CH3Cl  CH3SH+Cl-
        \begin{choices}
\choice CH3CI is a Bronsted Lowry acid	\choice HS- is a Lewis acid
        \choice HS· is a Bronsted - Lowery acid	\choice HS- is a Lewis Base.
        \end{choice}

\question For the stepwise dissociation of aquesous H3P04, which of the following is NOT a conjugate acid base pair?
        \begin{choices}
\choice HPO2- 4 and PO3-4	\choice H2PO- 4and HPO 2-
        \choice H3P04 an H2PO -	\choice H2P04 and PO3-4
        \end{choice}

\question Which solution below has the highest concentration of hydroxide ions?
        \begin{choices}
\choice pH =3.21	\choice pH =12.59	\choice pH =7.93	\choice PH= 9.82
\end{choice}
    
\question How many grams of Ca (OH)2 are contained in 1500ml of 0.0250M Ca(OH)2 solution? 
        \begin{choices}
\choice 1.85g	\choice 2.78g	C.3.17g	D.4.25g

\question What is the conjugate base of HSO- ·4?
        \begin{choices}
\choice H+	\choice H2S04	\choice OH¯	\choice S042¯
\end{choice}

\question What is the molality of a solution that contains 51.2g of naphthalene,( C10H8  ) in 500ml	of carbon tetrachloride? Given the density of CC14 = 1.60glml.
        \begin{choices}
\choice 0.25m	B.0.50m 	C.0.75m	    D.0.84m

\question Which of the following titrations will have an equivalence point at a pH> 7.00?
        \begin{choices}
\choice Weak acid with strong base.	\choice Weak acid with a weak base
        \choice Strong acid with a strong base  \choice Strong acid withweak base
        \end{choice}

\question A 0.10M solution of a weak acid, HX, is 0.059% ionized. What is the dissociation	constant for the acid?
        \begin{choices}
\choice 3.8xl0- 9	\choice 3.5xl0- 8	C.6.5xl0- 7	D.4.2xl0- 6

\question How many moles are there in  159g  of  alanine,  C6H5NH2? 
        \begin{choices}
\choice 0.560	\choice 0.992	C.1.78	D.3.31

\question What is the normality of l.0M solution ofNa2CO3?
        \begin{choices}
\choice 1N	\choice 0.5N	C.2N	D.3N .

\question Which of the following substances could N0T be amphiprotic?
        \begin{choices}
\choice NO ¯	B H2O	\choice NH3	\choice HCO ¯
\end{choice}

\question For the acid - base equilibrium, HCO - + H2O ⇌H2CO3 +OH¯ , the Bronsted - Lowry acid are	
    \begin{choices}
\choice H2Oand OH-      	\choice HCO3- and OH-
    \choice H2O and H2CO3	\choice HCO3- and H2CO3
    \end{choice}

\question The acide- base indicator bromocresol green is aweak acid . The yellow acid and blue base forms of the indicator are present in equal concentration in a solution when the PH is 4.
    \question What is the pK a of bromocresol green
    \begin{choices}
\choice 4.48	    \choice 4.68	    \choice 5.68	    \choice 6.68
\end{choice}
    
\question If NaNO2 is added to a solution of HNO2 , which of the following statement is true?
        \begin{choices}
\choice The pH of the solution increases	\choice The pH of the solution will decrease
        \choice The pH will remain the same	\choice The equilibrium will not be affected
        \end{choice}

\question If a solute dissolves in an endothermic process,
        \begin{choices}
\choice H bonds must exist between solvent and solute
        \choice Strong ion- dipole forces must exist in the solution
        \choice The entropy of the solution must be greater than that of its pure components
        \choice The solute must be a gas
        \end{choice}

\question The following data was collected at the end point of a titration performed to find molarity of an HCl solution. Volume of acid (HCl) used = 1.4 mL
Volume of base (NaOH) = 22.4mL
Molarity of standard base ( NaOH) = 0.20 M
On the basis of the above data ,what is the molarity of the acid solution? 
        \begin{choices}
\choice 1.6 M	B.0.64 M	\choice 0.31 M	\choice 0.13 M
\end{choice}

\question An aqueous solution contains 0.100 M NaOH at 25.0 o\choice The pH of the solution is? 
        \begin{choices}
\choice 0.100	\choice 1.00	\choice 7.00	\choice 13.0
\end{choice}

\question The pKa  of a weak monoprotic acid is 4.8 . What should be the ratio of [Acid]/[Salt] of buffer	, if pH = 5.8 is required?
        \begin{choices}
\choice 0.1  	\choice 1.0	\choice 2.0	\choice 10
\end{choice}

\question PKa value of three acids X, Y and Z are 4.5 , 3.5 and 6.5, respectively . Which of the following represents the correct order of acid strength ?
        \begin{choices}
\choice X > Y > Z	\choice Z	> X	> Y	\choice Y > X > Z	\choice Z	> Y	> X
\end{choice}

\question What is the concentration of sodium chloride in water needed in order to produce an aqueous	solution that has an identical osmotic pressure (isotonic) with blood (7.70) atm at 25 oC ?
    \begin{choices}
\choice 0.003mol L- - 1	\choice 0.006mol L- 1	\choice 0.1575 mol L- 1	\choice 0.315 mol L- 1
\end{choice}

\question For the reaction N2 + 3H2	2NH3 the rate of disappearance of H2 is 0.01mol L   min .
The rare of appearance of NH3 would be
    \begin{choices}
\choice 0.01mol L- 1min- 1	\choice 0.02mol L- 1min- 1	\choice 0.007molL- 1min- 1	\choice 0.002 mol L- 1 min- 1
\end{choice}

\question 	The solubility of oxygen gas in water at 250C and 1.0 atm pressure of oxygen is 0.041 g/L. The solubility of oxygen in water at 3.0 atm and 250C is	g/L.
    \begin{choices}
\choice 0.014	\choice 0.31	\choice 0.041	\choice 0.123
\end{choice}

\question According to the Arrhenius concept ,an acid is a substance that
        \begin{choices}
\choice is capable of donating one or more H + to any solvent
        \choice causes an increase in the concentration of H + in aqueous solutions
        \choice can accept a pair of electrons to form a coordinate covalent bond
        \choice reacts with the solvent to the cation formed by autoionazation of that solvent
        \end{choice}
\question The Ka of hypochlorous acid ( HClO) is 3.0 ╳	10- 8 at 250\choice What is the % ionization of hypochlorous acid in a	0.015M aqueous solution of HClO at 250C ?
        \begin{choices}
\choice 2.1 ╳ 10- 5	\choice 0.14	\choice 1.4 ╳	10- 2	\choice 3.3 ╳	10- 1
\end{choice}

\question Which one of the following combinations CANNOT produce a buffer solution ?
        \begin{choices}
\choice HCl O4 and NaClO4	\choice HCN and NaCN
        \choice HNO2 and NaNO2	\choice NH3 and (NH4)2SO4
        \end{choice}

\question How many moles of NH4Cl must be added to 1.5L of 0.2M solution of NH3 to form abuffer whose pH is 9.00 (Kb = 1.8  ╳ 10- 5) ?
        \begin{choices}
\choice 0.36	\choice 0.65	\choice 0.54	\choice 0.8
\end{choice}

\question What is the ionization constant for a weak acid, HA, that is .60% ionized in 0.0950 M solution? 
        \begin{choices}
\choice 2.69 ╳ 10- 3	\choice 3.77 ╳ 10- 2	\choice 2.47 ╳ 10- 5	\choice 9.91 ╳ 10- 6
\end{choice}

\question In which direction will the following eq- uilibrium s+ hift ,if a solution of CH3CO2Na is added ?
CH3COOH(aq)	⇌	CHH3CO2 (aq) + H (aq)
        \begin{choices}
\choice The equilibrium shift to the right (more products)	\choice No change
        \choice The eq-u4ilibrium shifts to the left (more reactant)	\choice Cannot be predicted
        \end{choice}

\question \begin{choices}
\choice 1.0 x 10 M solution of a weak acid is found to dissociate by only 1.37% . Which of the following acid is
likely to be?	- 4	- 5
        \begin{choices}
\choice HF	Ka = 7.2 x 10 - 4	\choice CH3COOH	Ka = 1.8 x 10	- 4
        \choice HNO2	Ka   = 6.9 x 10	\choice HCOOH	Ka = 1.9 x 10
        \end{choice}

\question A 1.0 x 10- 4 M solution has a pH of 10.00 The solute is a
        \begin{choices}
\choice weak acid	\choice weak base	\choice strong base	\choice strong acid
\end{choice}

\question Which of the following salt will be hydrolyzed to give acidic solution?
        A.CH3COONa	\choice CH3COONH4	\choice NH4Cl	\choice Na2CO3
        \end{choice}

\question Which of the following would NOT make a good buffer solution? 
        A.CH3COO- and CH3COOH	\choice HCO - and H2CO3
        \choice SO42- and HSO4-	    \choice NH3 and NH4+
        \end{choice}

\question Which of the following will give the strongest conjugate base?
        \begin{choices}
\choice HNO3	\choice H2O	C.HSO -	\choice CN-
\end{choice}
    
\question The dye bromthymol blue (HBb) is a weak acid whose ionization can be represented as
HBb(aq) (yellow)	⇆	H+ (aq)	+ Bb(aq) (blue)
Which way ill the equilibrium shift when NaOH is added and what is the color of NaOH solution containing the dye
        \begin{choices}
\choice There will be no shift
        \choice The equilibrium will shift to the right and the color of the solution will be blue
        \choice The equilibrium will shift to the left and the color of the solution will be blue
        \choice The equilibrium will shift to the right , and thecolor of the solution will be yellow
        \end{choice}

\question Consider an indicator that ionized as shown below for which its Ka =1.0 x 10- 4
HIn(yellow)	+	H2O	⇆	H2O+   +	In- 1 (red)
Which of the following statements is NOT true about the above equilibrium ?
        \begin{choices}
\choice The predominant color in its acid range is yellow
        \choice The pH at which the indicator changes color is pH =4
        \choice At pH =7 most of the indicator is in the un- ionized form
        \choice In the middle of the pH range of its color change , a solution containing the indicator will	probably be change
        \end{choice}

\question In the following acid - base equilibrium, HCO - + H2O ⇆ H2CO3  + OH- , the Bronsted- Lowry acids are
        \begin{choices}
\choice H2O and OH-	\choice HCO3-   and OH-	\choice H2O and H2CO3	\choice HCO- - - and H2CO3
\end{choice}

\question Which of the following is NOT both a Bronsted- Lowry acid and a Bronsted - Lowry base?
        \begin{choices}
\choice HSO4-	\choice H2PO4-	\choice HCO	\choice OH-
\end{choice}

\question The pH of a 0.10 M solution of a weak base is 9.82 . What is the Kb for this base? 
        \begin{choices}
\choice 4.4 x 10- 8	\choice 3 x 10- 7	\choice 3 x 10- 6	\choice 3.4 x 10- 5
\end{choice}

\question Which of the following is NOT both a Bronsted- Lowery acid and a Bronsted- Lowery base?
        \begin{choices}
\choice HS-	\choice HSO4-	\choice OH-	\choice HCO3
\end{choice}

\question A solution in which propanoic acid is 0.94% ionized has a PH of 2.85.what is the value of the acid ionization constant (Ka) for propionic acid?
        \begin{choices}
\choice 4.2x10- 4	\choice 6.8 x10- 4	\choice 1.3x10- 5	\choice 8.7x10- 5
\end{choice}

\question A 0.2M solution of a weak acid HA is 1% ionized at 25o\choice Ka for the acid is equal to:
        \begin{choices}
\choice 0.19		\choice 0.02 x 0.02	\choice 0.01 x 0.01	    \choice 0.002 x 0.002
\end{choice}

\question Which one of the following statement is true?
        \begin{choices}
\choice Addition of (CH3NH3)Cl to a solution of CH3NH2 will decrease the PH.
        \choice Addition of NaNO2 to a solution of HNO2 will decrease the PH.
        \choice Addition of HCl to a solution NaC2H3O2 will increase the PH.
        \choice Addition of KBr to a solution of HBr will increase the PH.
        \end{choice}

\question During titration, what volume of 0.500M KOH is necessary to completely neutralize 10.0ml of 2.00M CH3 COOH?
        \begin{choices}
\choice 10.0ml	\choice 20.0ml	\choice 40.0ml	\choice 25.0mL
\end{choice}

\question If 49 grams of H2SO4 reacts with 80.0grams of NaOH, how much reactant will be left over after the reaction is complete?
        \begin{choices}
\choice 24.5g H2SO4	\choice 40.0g NaOH	\choice 20.0g NaOH	\choice 60.0g NaOH
\end{choice}

\question A solution of 10.0ml of 0.050M nitrous acid ( HNO2; PKa=3.34) is titrated with 0.050M KOH. What is the PH of the solution after 10ml KOH has been added?
        \begin{choices}
\choice 10.02	\choice 6.68	\choice 3.34	\choice 13.36
\end{choice}

\question Consider the following equilibria;
    I. HCO - + H2O	H2CO3 + OH-
    II. NH4+ + H2O	H3O+ +NH3
    III.HSO3- + H3O+		H2O + H2SO3 In each of the following equilibrium(equilibria) does water act as a bronsted- lowery base?	
        \begin{choices}
\choice II only	\choice Ionly	\choice II and III	\choice I, II,III
\end{choice}

\question When a saturated solution of NaCl is heated, it be becomes?
        \begin{choices}
\choice \begin{choices}
\choice unsaturated	\choice Supersaturated	\choice Remains saturated	\choice Attains equilibrium conditions
\end{choice}

\question What is the PH of a 0.10M solution of NH3,a weak base, whose Kb=1.8x10- 5
        \begin{choices}
\choice 11.11	\choice 4.76	\choice 9.24	\choice 2.89
\end{choice}

\question The PH of a solution prepared by the addition of 100ml 0.002M HCl to 100ml distilled water is closest to: 
        \begin{choices}
\choice 1.0	    \choice 1.5	    \choice 3.0  	\choice 2.0
\end{choice}

\question Which of the following can function as both a bronster- Lowery acids and bases?
        \begin{choices}
\choice HSO4-	\choice H2SO4	\choice SO4- 2	\choice H+
\end{choice}


Chemistry grade- 12 Entrance Chapter- 3

\question A gas is confined to a cylinder under constant atmospheric pressure. When the gas undergoes a particular chemical reaction, it releases 135kJof heat to its surroundings and does 63 kJof p- V work on its surroundings. What are the values of ∆H and ∆E for the process?
        \begin{choices}
\choice ∆H= 135kJ, ∆E= 63kJ	    \choice ∆H= 135kJ, ∆E = 198kJ
        \choice H= - 135kJ, ∆E= - 63Kj	\choice ∆H= - 135kJ, ∆E= - 198kJ
        \end{choice}

\question Which statement about the following reaction is correct? 2Fe(s) + 3CO2(g) → Fe2O3(s) + 3CO(g) ∆H0 = +26.6kJ
        \begin{choices}
\choice 26.6kJof energy are released for every mole of Fe reacted
        \choice 26.6kJof energy are absorbed for every mole of Fe reacted
        \choice 53.2kJof energy are released for every mole of Fe reacted
        \choice 13.3kJof energy are absorbed for every mole of Fe reacted
        \end{choice}

\question Which of the following is true about an open system?
        \begin{choices}
\choice A system that can not exchanges both energy and matter with its surroundings
        \choice A system that can not exchange both matter and energy with its surroundings
        \choice A system that only exchanges matter with its surroundings
        \choice A system that only exchanges energy with its surroundings
        \end{choice}

\question What is the value of ∆H for the reaction S (s) →S(g)? S(s) + O2 (g) →SO2(g)	∆H = - 395kJ
S(g) + O2 (g) → SO2(g)	∆H = - 618kJ
        \begin{choices}
\choice - 1013kJ	\choice - 223kJ	\choice +223kJ	\choice +1013kJ
\end{choice}

\question Enthalpy is defined as the heat content of the system at constant:
        \begin{choices}
\choice Heat	\choice Moles	\choice Pressure	\choice Volume
\end{choice}

\question How many kilojoules of heat are absorbed when 20g of NaCl(s) is decomposed into Na(s) and Cl2(g) at constant pressure according to the following reaction?
2Na(s) + Cl2(g)	2NaCl(s) ∆H = - 802.8kJ
        \begin{choices}
\choice - 281.0	\choice - 140.5	\choice +140.5	\choice +281.0
\end{choice}

\question What is the quantity of heat required to raise the temperature of 80g of ethanol from 250c to 75 0c? ( specific heat of ethanol = 2.46JgK- 1)
        \begin{choices}
\choice 2.46kJ	\choice 4.0kJ	\choice 9.84kJ	\choice 18.68kJ
\end{choice}

\question Consider the following gaseous reaction at 250c:
CH4(g) + 2O2(g)	CO2(g) + 2H2O(g) ∆H = - 802kJ
Which energy change would occur if 3.2 moles of CH4 is completely combusted?
        \begin{choices}
\choice 2.57 x 102 kJwill be released	\choice 6.43 x 103 kJwill be released
        \choice 2.57 x 102kJwill be absorbed 	\choice 6.43 x 103 kJwill be absorbed
        \end{choice}

\question Which of the following reactions is expected to have negative value of entropy change (∆S)? 
        \begin{choices}
\choice C6H6(s) + 6O2(g)	6CO2(g) + 6H2O(g)
        \choice CaCO3(s)	CaO(s) + CO2(g)
        \choice N2O4(g) + Cl2(g)	2NOCl(g) + O2(g)
        \choice 2SO2 (g) + O2(g)	2SO3(g)
        \end{choice}
    
\question Gaseous petrol in a combustion system has done 375 kJ of work during its expansion in the piston.
Simultaneously, it absorbed 586kJ of heat from the engine. What is the internal energy change during the process?
        \begin{choices}
\choice +222kJ	\choice +961kJ	\choice - 211kJ	\choice - 961kJ
\end{choice}

\question Under what conditions will the enthalpy change of a process equals the amount of heat transferred into or out of the system?
        \begin{choices}
\choice Under constant volume	\choice Under constant pressure and volume
        \choice Under constant pressure	\choice Under constant pressure and temperature.
        \end{choice}

\question For the reaction A + B →C + D, ∆H0=+40kj and ∆S0 = + 50J/K Therefore, the reaction under standard conditions is
        \begin{choices}
\choice Spontaneous at all temperatures	\choice Nonspontaneous at all temperatures.
\end{choice}
        \choice Spontaneous at temperatures greater than 800K.	\choice Spontaneous only at temperatures between 10K and 800K.

\question What is the change in internal energy, ∆E of a system if it absorbs 300 KJof heat from the surroundings and does 500 KJof work on the surroundings?
        \begin{choices}
\choice 100KJ	\choice - 200 KJ	\choice 400 KJ	\choice 500 KJ
\end{choice}

\question Why does a pressure cooker reduce cooking time?
        \begin{choices}
\choice The heat is more evenly distributed	\choice The high pressure tenderizes the food
        \choice A stronger flame is used for cooking	\choice The boiling point of water inside the cooker is increased
        \end{choice}

\question Which of the hydrogen halide has the highest enthalpy of vaporization?
        \begin{choices}
\choice HI	\choice HBr	\choice HCI	\choice HF
\end{choice}

\question Which of the following statement is NOT true according to 1st law of    thermodynamics?
        \begin{choices}
\choice Energy can neither be created nor destroyed although it can be change from one form to another.
        \choice The total energy of the universe is constant.
        \choice The change in the internal energy of a closed system is equal to the sum of heat absorbed by the system and work done by the system.
        \choice The change in internal energy of a system is zero in all the processes.
        \end{choice}

\question What is the bond order of O2+?
        \begin{choices}
\choice 2	. \choice 2.5	\choice 1.5	\choice 3
\end{choice}

\question Which of the following is characteristic of an anti- bonding molecular orbital?
        \begin{choices}
\choice It is a molecular orbital with a high probability of finding e- parallel to the region between bonded atoms.
        \choice It has no electrons
        \choice It is a molecular orbital with a high probability of finding e- away from the region between bonded atoms.
        \choice It is a molecular orbital with a high probability of finding electrons in the region between bonded atoms.
        \end{choice}

\question The enthalpy change for which reaction represents the standard It. enthalpy of formation for hydrogen cyanide, HCN?
        \begin{choices}
\choice H(g)+C(graphite) +N(g) - +HCN(g)	\choice HCN(g)  1/2 H2(g) +C(graphite) + 1/2 N2(g)
        \choice 1/2 H2(g)+C(graphite) + 1/2N2(g) HCN(g)	\choice H 2(g) + 2C (graphite) +N2(g) 2HCN(g)
        \end{choice}
    
\question What is the standard entropy change for the reaction:	2S02(g)+02(g) 2S03(g) Given the following data: (So of SO2 (g) =248.IJ/(mol.K); S0 of
02 (g) = 205.0J/(mol.K); S0 of S02(g) =256.6J/(mol.K)
        \begin{choices}
\choice - 188.0JIK	\choice +l7:0JIK	\choice - 1965.5JIK \choice +60.IJIK
\end{choice}

\question A system which can change both matter and energy with its surrounding is said to be a/an
        \begin{choices}
\choice isolated system    \choice open system	\choice ideal system      \choice closed system
\end{choice}

\question 	The enthalpy of combustion of solid carbon to form carbon dioxide is - 393.5KJ/mol carbon	and the enthalpy of combustion of carbon monoxide to form carbon dioxide is -    283.3KJ/mol CO. What will be the enthalpy change ,  H for the reaction?
2C(s) + O2(g)          2CO(g)
        \begin{choices}
\choice - 110.4KJ	\choice - 220.8KJ	\choice + 172.9KJ	\choice + 1354.0KJ
\end{choice}

\question For the conversion of C (diamond)	C (Graphic), ∆H =- 3KJ. What does this mean?
        \begin{choices}
\choice Both are equally stable	\choice Diamond is more stable than graphite
        \choice Graphite is more stable than diamond	\choice Graphite has more stable energy than diamond
        \end{choice}

\question If the enthalpy change for a certain reaction A  B is 2kJat 300K, what would be the entropy	change in the surrounding?
        \begin{choices}
\choice - 40J/K	\choice 40J/K	\choice - 3.6 X 106 J/K	\choice 3.6 X 106J/K
\end{choice}

\question The enthalpies of formation of gaseous N2 O and NO at 298 K are 82 and 90KJ/mol. respectively. The enthalpy change for the reactionN O (g) + 1/2 O2(g) 2NO(g) is
        \begin{choices}
\choice - 8KJ	\choice 98KJ	\choice - 74KJ	\choice 8KJ
\end{choice}

\question Sodium acetate spontaneously crystallizes out of a supersaturated solution on standing .Which of the following is true	for for thermodynamic quantities of this system for such a process?
        \begin{choices}
\choice ∆S < 0 ,∆H < 0	\choice ∆G < 0 , ∆H > 0	\choice ∆S > 0, ∆G < 0	\choice ∆S > 0 , ∆H > 0
\end{choice}

\question Which one of the follo0wing statements best describes the standard enthalpy of  formation of any element?
        \begin{choices}
\choice The value of ∆H 0 f (element) depends on temperature
        \choice The value of ∆H f   (element) is zero for any element in the standard state
        \choice The value of ∆H 0 f   (element) is zero only for element in the solid state
        \choice The value of ∆H 0 f (element) is zero only at absolute zero temperature
        \end{choice}

\question A drug used to treat hypertension undergoes a decomposition reaction to give an insoluble product. What is
the temperature at which this reaction becomes spontaneous if the enthalpy of the reaction at 298 K is 51 kJ mol- and the entropy of the reaction at this temperature is 118.74 JK- 1 mol- 1 ?
        \begin{choices}
\choice 375 K	\choice 430 K	\choice 525 K	G. 530 K

\question What happens to the value of ∆H for a thermochemical reaction if the reaction is reversed ?
        \begin{choices}
\choice ∆H has the same numerical value , and the sign changes
        \choice ∆H has the same numerical value , and the sign reactions the same
        \choice ∆H is the reciprocal of the original value, and the sign changes.
        \choice ∆H is the reciprocal of the original value , and the sign remains the same.
        \end{choice}
    
\question The first law of thermodynamics is based on which of the following principles?
        \begin{choices}
\choice Action and reaction	\choice Conservation of mass
        \choice Conservation of energy	\choice The entropy - temperature relationship
        \end{choice}

\question A system that does NOT exchange matter or energy with its surrounding is called an	system.
        \begin{choices}
\choice adiabatic	\choice isolated	\choice isothermal	\choice isotonic
\end{choice}

\question Calculate H for the following reaction using the bond energies given below: H- H(g) + I- I(g)	2H- I(g)
Bond energies: H- H=436KJ/mol, I- I=151KJ/mol, H- I=297KJ/mol 
        \begin{choices}
\choice +290KJ	\choice - 7KJ	\choice +7KJ	\choice - 290KJ
\end{choice}

\question is the standard cell potential for the oxidation of ammonia, given below? 4NH3 +   3O2	2N2    +   6H2O ,	G=- 1356KJ
        \begin{choices}
\choice 3.51V	\choice 1.17V	\choice 7.02V	\choice 14.04V
\end{choice}

\question Which of the following statement is true?
        \begin{choices}
\choice If the entropy of the system increases during a reversible process, the entropy change of the surroundings decrease by the same amount.
        \choice If the entropy of the system increases during a reversible process, the entropy change of the surroundings increase by the same amount.
        \choice 	If the entropy of the system increases during a reversible process, the entropy change of the surroundings will remain the same.
        \choice If the entropy of the system decreases during a reversible process, the entropy change of the surroundings decrease by the same amount.
        \end{choice}

\question What is the change in internal energy of a system that releases 12.4Jof heat and does 4.2Jof work on the surrounding
        \begin{choices}
\choice - 16.6J	\choice 8,2J	\choice 16.6J	\choice - 8.2J
\end{choice}

\question Based on the information given in the table below, what is the enthalpy change for the reaction: 2H2O2(g) +	S(s)	SO2(g) + 2H2O(g)
        \begin{choices}
\choice - 16.6J	\choice 8,2J	\choice 16.6J	\choice - 8.2J
\end{choice}

\question What does it
        \question The is
        \question The process is Adiabatic
    A.200KJ	    \choice - 200KJ	    \choice - 500KJ	    \choice 400KJ
    \end{choice}





Chemistry grade- 12 Entrance Chapter- 4

\question The electrolysis of molten NaCl is an industrial process. what does the electrolysis product?
        \begin{choices}
\choice Na and Cl2	\choice H2 and O2	\choice Na+andCl- \choice NaOH and Cl2
\end{choice}

\question For which conversation is an oxidizing agent required?
        \begin{choices}
\choice   2H+ (aq) → H2(g)	\choice SO3(g) → SO42- (aq)
        \choice 2Br- (aq) → Br2(aq)	\choice MnO2(s) → Mn +(aq)
        \end{choice}

\question The oxidation numbers of nitrogen in NH3, HNO3 and NO2 are	respectively.
        \begin{choices}
\choice - 3, - 5, +4	\choice +3, +5, +4	\choice - 3, +5, - 4	\choice - 3, +5, +4
\end{choice}

\question The two standard electrode potentials involved in the nickel- cadmium rechargeable cell are given below. Calculate the ∆G0 in kJof the cell.
NiO2(s) + 2H2S(1) + 2e−	→ Ni(OH)2 (s) + 2OH- E0 = + 0.49v Cd(OH)2(s) + 2e − → Cd(s) + 2OH- (aq) E0 = - 0.76v
        \begin{choices}
\choice - 184	\choice - 153	\choice - 241	\choice - 206
\end{choice}

\question Which one of the following reactions is NOT a redox reaction?
        \begin{choices}
\choice Ag +(aq) + Cl- (aq) → AgCl(s)	\choice Mg(s) + 2HCl(aq) → MgCl2(aq) + H2(g)
        \choice 2Na(s) + Cl2(g) → 2NaCl(s)	\choice Cu2+ (aq) + Zn(s) → Cu (s) + Zn2+ (aq)
        \end{choice}

\question The half – reaction for formation of magnesium metal upon electrolysis of molten MgCl2 is : Mg2+ + 2e-	Mg
What is the mass of magnesium formed upon passage of a current of 60.0 A for a period of 2.00 x 103 s?
        \begin{choices}
\choice 5.0g	\choice 10.0g \choice 15.1g	\choice 30.2g
\end{choice}

\question Which of the following is NOT a characteristic of the electrolytic cell containing aqueous solution of NaCl used in the manufacture of sodium hydroxide?
        \begin{choices}
\choice The sodium hydroxide solution is produced in the electrolytic cell
        \choice The electrolyte must be a dilute solution of NaCl
        \choice Hydrogen is produced at the cathode
        \choice The production of chlorine gas occurs at the anode
        \end{choice}

\question Which of the following metals is NOT obtained by commercial electrolytic process?
        \begin{choices}
\choice Ag	\choice Al	\choice Cu	\choice Na
\end{choice}

\question A 1 M solution of Cu (NO3)2 is placed in a beaker with a strip of Cu metal. A 1M of SnSO4 is placed in a second beaker with a strip of Sn metal. The two beakers are then connected by a salt bridge and the two metal
electrodes are connected by wires to a voltammeter. Which of the following electrode serves as the anode and which electrode gain mass?
Given that E0 Cu2+/Cu = 0.34V and E0sn2+/Sn = - 0.14V
        \begin{choices}
\choice Anode, Sn,Sn electrode gains mass	\choice Anode, Cu, Sn electrode gains mass
        \choice Anode, Sn Cu electrode gains mass	\choice Anode, Cu, Cu electrode gains mass
        \end{choice}

\question 	In the electroplating of nickel from a solution containing Ni2+ion, what will be the weight of the metal deposited on the cathode by a current of 8A flowing for 500minutes?
        \begin{choices}
\choice 73g	\choice 103g	\choice 117.4g	\choice 145g
\end{choice}

\question Consider the following unbalanced redox reaction in acidic solution: Mn4 − + Fe2+	Mn2+ + Fe3+
What is the change in oxidation state for both the substances oxidized and reduced, and the coefficients of Fe2+ and Mn2+ respectively, after balancing?
        \begin{choices}
\choice 2 and 7, and 2 and 5	\choice 3 and 2, and 5 and \choice 1 and 5, and 5 and 1	\choice 2 and 5, and 5 and 2
\end{choice}

\question 	For the following hypothetical equation, in aqueous solution, what is the correct representation of the cell notation?
A(s) + B2 +(aq)	A2+(aq) + B(s)
        \begin{choices}
\choice A(s)|A2+(aq)||B2+(aq)|B(s)	\choice A2+(aq) | A (s) || B (s) | B2+(aq)
        \choice B2+(aq)| B(s)|| A2+(aq) | A (s)	\choice B(s) | B2+(aq) || A2+(aq) | A(s)
        \end{choice}

\question What reactions occur at the anode and cathode when an aqueous solution of Na2SO4 is electrolyzed? E0red
I.	S2O82- + 2e−	2SO42-  2.01V
II.	O2 + 4H+ + 4e−	2H2O	1.23V
III.	2H2O + 2e−	H2 + 2H-	- 0.83V
IV.	Na+ + e−	Na	- 2.71V
    \begin{choices}
\choice H2 at cathode and O2 at anode	\choice H2 at cathode and S2O82- at anode
    \choice Na at cathode and S O 2− at anode	\choice Na at cathode and O2 at
    \end{choice}

\question If ACr2+ (aq) solution is ectrolyzed us· g a current of 15A then what mass of chromium, ted out after 3 days?
    A.400g	\choice 700g	\choice 800g	D.2100g

\question Which of the following has the greatest number of atoms?
        \begin{choices}
\choice 4g of hydrogen	\choice  24gof ozone	\choice 40g of calcium	\choice 127gof iodine
\end{choice}

\question Consider the following half - reactions:
Fe (H0)2 (s) + 2e- ~Fe (s) + 20H- (aq), Eo1 = - 0.88V
Ni02 (s)+2H20 (I) +2e- ~Ni(0H)2 (s) + 20H- (aq), Eo = +0.49V
What is the standard cell potential for a voltaic cell using these electrodes?
        \begin{choices}
\choice - 1.37V	\choice - 0.39V	\choice O.4	\choice 1.37V
\end{choice}

\question What is the standard entropy change for the reaction:	2S02(g)+02(g) 2S03(g) Given the following data: (So of SOlg) =248.IJ/(mol.K); SO of
02(g) = 205.0J/(mol.K); SO ofS03(g) =256.6J/(mol.K)
        \begin{choices}
\choice - 188.0JIK	\choice +l7:0JIK	\choice - 1965.5JIK \choice +60.IJIK
\end{choice}

\question 	The standard reduction potential of Zn2+ / Zn is (- 0.76) V and that of Cu2+ /Cu is +0.34V.	What would be the electromotive force (emf) of the cell constructed between these two electrodes?
        \begin{choices}
\choice 1.1OV	\choice 0.42V	\choice - l.lOV	\choice - 0.42V
\end{choice}

\question At constant T and P, which one of the following statements about the enthalpy (~H) and internal energy (~U) changes is correct for the reaction? CO (g)+ 1/2 02(g) C02g)
        \begin{choices}
\choice H = U	\choice H < U	C.H > U   \choice .H and U are independent of physical states of the reactants
\end{choice}
    
\question Which one of the following conditions regarding a chemical process ensures its spontaneity at all temperatures?
        \begin{choices}
\choice H < O,~S < O	\choice H > O,S <O	\choice H < O,S > O	\choice H > O,S > O
\end{choice}

\question Which equation represents an oxidation - reduction reaction?
        \begin{choices}
\choice  H2SO4+2NH3~(NH4)2SO4	    C.2K2CrO4+ H2S04~K2Cr2O7+K2SO4+H2O
        \choice  H2SO4+Na2CO3~Na2SO4+H2O+CO2	\choice 2 H2SO4+CU~CuSO4 +2H2O+SO2
        \end{choice}

\question An electrochemical cell constructed for the reaction: Cu2+(aq)+M(s)	Cu(s)+M2+(aq)
Has an E0 cell= 0.75V. The standard reduction E0 red potential for
ACu2+(aq) is 0.34V. What is the standard reduction potential for (M+2 (g)/ M?
        \begin{choices}
\choice - 1.09V	\choice - 0.41OV	C.O.4lOV	D.1.09V

\question What is the correct order when the substances 02,H20,OF2, and H202 are arranged in order of increasing oxidation number for Oxygen?
        \begin{choices}
\choice  02,H2O,OF2,H2O2	\choice H2O,H2O2,O2,OF2	\choice H2O2,O2,H2O,OF2	\choice OF2,O2,H2O2,H2O
\end{choice}

\question Consider the standard voltaic (or galvanic) cell: Fe,Fe2+ versus, Au,Au3+. Which	answer identifies the cathode and gives the E0 red for the cell?
Given EO(Au3+/Au)= 1.50 V, E0 (Fe2+/Fe) = - 0.44V .
        \begin{choices}
\choice Fe,- 0.44V  \choice Au, 1.06V	\choice Au, 1.94V \choice Fe, 1.94V
\end{choice}

\question The process of solute particles being surrounded by solvent particles is known as
        \begin{choices}
\choice saturation	\choice agglomeration	\choice solvent	\choice dehydration
\end{choice}

\question What kind of energy is converte in galvanic cell?
        \begin{choices}
\choice Chemical energy is converted into electrical energy
        \choice Chemical energy is converted to heat
        \choice Chemical energy is obtained from heat
        \choice Electrical energy is converted into chemical energy
        \end{choice}

\question The decreasing order of electrochemical characteristics of some metals is given as : Mg > Al	> Zn > Cu >
Ag . What will happen if a cooper spoon is used to stir a solution of aluminum	nitrate (Al (NO3)3) ?
        \begin{choices}
\choice There is no reaction	\choice The spoon will get coated with aluminum
        \choice The solution becomes blue	\choice An alloy of cooper and aluminum is formed
        \end{choice}

\question Which of the following statement is ture ?
        \begin{choices}
\choice All forms of electromagnetic radiation are visible
        \choice Radio waves have shorter wavelength than visible light
        \choice Ultraviolet light has longer wavelengths than visible light
        \choice The frequency of radiation increases as the wavelength decreases
        \end{choice}

\question A solution in an electrolytic cell contains Cu2+ (Eo = 0.34 V), Ag+ (Eo = 0.80 V), and Zn2+	(Eo = - 0.76 V). If the voltage is initially very low and slowly increased , in which order will	the metals be plated out onto the cathode?
        \begin{choices}
\choice Zn2+ > Cu2+ > Ag+	\choice Ag+ > Zn2+	> Cu2+
        \choice Cu2+ > Zn2+ > Ag+	\choice Ag+	> Cu2+ > Zn2+
    
\question 	During the electrolysis of an aqueous solution of copper sulphate using platinum electrodes,	the reaction takes place at the anode is
        \begin{choices}
\choice CU2+ +2e-    CU	\choice CU		CU2++2e-
        \choice 2H2O		4H+ +O2+4e-	\choice 4H+ + O2   +4e-		H2O
        \end{choice}

\question .Standard electrode potential for Sn4+/Sn2+ couple is +0.15v and that  for the Cr3+/Cr couple	is - 0.74v. These two couples in their standard state are connected to make a spontaneous	electrochemical reaction. The cell potential will be
        \begin{choices}
\choice +1.83v	\choice +1.19v	\choice +0.89v	\choice +0.18v
\end{choice}

\question What mass of magnesium is plated out upon elecrolysis from molton MgCl2 using acurrent of 60 A for a period of 4000 seconds?
        \begin{choices}
\choice 30g	\choice 24g	C, 60g

\question For a voltaic (or galvanic) cell using Ag ,Ag statements is INCORRECT?
        \begin{choices}
\choice The zinc electrode is the anode
        \choice Electrons ill flow through the external circuit from the zinc electrode to the silver elecrode.
        \choice The mass of the zinc electrode will increase as the operates .
        \choice Reduction occurs at the zinc electrode as the operates.
        \end{choice}

\question For the galvanic cell shown below , which one of the following statements is correct?
        \begin{choices}
\choice At the zinc electrode , zinc ions are formed
        \choice The electrode potential is measured by the voltmeter
        \choice The following reaction takes place at the magnesium electrode Mg2+ + 2e-  Mg
        \choice Electrons flow from zinc electrode to the magnesium electrode
        \end{choice}

\question What is the purpose of a salt bridge in an electrochemical cell?
        \begin{choices}
\choice To provide a source of ions to react at the anode and cathode.
        \choice To maintain electrical neutrality in a half - cell through migration of ions
        \choice To provide means of electrons to travel from the cathode to the anode
        \choice To provide means of e0lectrons to travel from the anode to the cathode
        \end{choice}

\question The standard cell potential (E ) for the reaction below is 1.10V. What is the cell potential for this reaction
when
[Cu2+] = 1 x 10- 5 M and [Zn2+] = 1M ?
Zn(s) + Cu2+(aq)		Zn2+(aq)	+	Cu (s)
        \begin{choices}
\choice 1.10	    \choice 0.95	    \choice 1.20	    \choice 1.35
\end{choice}

\question Which of the following statements is tr0ue?
        A The more positive the value of  Ered    , th0e greater the driving force for reduction
        \choice The more exothermic the value of Ered 0, the greater the driving force for reduction
        \choice The more endothermic the value of0 Ered   , the greater the driving force for reduction
        \choice The more negative the value of Ered   , the greater the driving force for reduction
        \end{choice}

\question Which of the following types of elements are good oxidizing agents?
        \begin{choices}
\choice Alkali metals	\choice Halogens	\choice Lanthanides	\choice Transition elements.
\end{choice}

\question Which of the following is NOT true about the electrolysis of copper sulphate solution using copper
electrodes?
        \begin{choices}
\choice Copper metal is deposited at the cathode
        \choice The concentration of copper at the cathode decreases
        \choice The copper metal dissolves leaving electrons at cathode
        \choice The impure copper at anode is turned to pure copper at the cathode
        \end{choice}
\question Of the following , which is the WEAKEST oxidizing agent?
        \begin{choices}
\choice HNO3(aq)	\choice H2O2 (aq)	\choice I2 (s)	\choice Mg(s)
\end{choice}

\question Which of the following metals CANNOT be electroplated on to the surface of another metal using an aqueous electrolyte?
        \begin{choices}
\choice Ag	\choice Cu	\choice Ni	\choice Mg
\end{choice}

\question Which of the following statements is true in the electrolytic decomposition of water?
        \begin{choices}
\choice Oxygen is formed at both the anode and the cathode
        \choice Hydrogen is formed at both the anode and the cathode
        \choice Hydrogen is formed at the cathode and oxygen is formed at the anode 
        \choice Hydrogen is formed at the anode and oxygen is formed at the cathode
        \end{choice}

\question When an electric current is passed through a solution of potassium iodide (KI) containing some
phenolphthalein indicator , a brown colour is observed at one of the electrodes . Which one of the following reactions is occurring at this electrode?
        \begin{choices}
\choice 2H2O + 4I- 2I2 +		2OH-   +	H2 + 2e-	\choice KI   +	I-	+	K
        \choice 2H2O +	2e-		H2	+	2OH -	            \choice 2I-		I2	+	2e-
        \end{choice}

\question What is the half- reaction that occurs at the anode during the electrolysis of molten sodium bromide?
        \begin{choices}
\choice 2Br-		Br2  + 2e-	\choice Br2 + 2e-		2Br
        \choice Na+	+	e-		Na	\choice Na		Na+	+ e-
        \end{choice}

\question Chlorine has an oxidation number of +5 in :
        \begin{choices}
\choice NaClO	\choice NaClO3	\choice NaClO2	\choice NaClO4
\end{choice}

\question When a sodium chromate (Na2CrO4) solution is acidified, its is converted to:
        \begin{choices}
\choice Cr(s)	\choice Cr O 2-	\choice CrO3	\choice Cr2O3
\end{choice}

\question During the electrolysis of a concentrated aqueous solution of NaCl, what substance is formed at the cathode?
        \begin{choices}
\choice Hydrogen	\choice Oxigen	\choice Chlorine	\choice Sodium
\end{choice}

\question In a voltaic cell, electrons flow from the…………..to the………..
        \begin{choices}
\choice Salt bridge, anode	\choice Salt bridge, cathode	\choice Anode, salt bridge	\choice Anode , cathode
\end{choice}

\question During the electrolysis of CuSO4(aq)using carbon electrodes, which of the following is the correct half reaction for the anode electrode?
        \begin{choices}
\choice Cu+(aq) + 2e-	Cu(s)
        \choice 4OH- (aq)	2H2O(l) + 4e-
        \choice   2H2O(L)	O2(g) + 4H+(aq) + 4e-
        \choice SO42- (aq) + O2(g) + 4H+(aq)	SO - (aq) + 2H2O(L)
        \end{choice}
\question Two electrolytic cells were placed in series, one of AgNO3 and the other of CuSO
    \question If 1.273g of Ag is deposited,
how much Cu was deposited at the same time?
        \begin{choices}
\choice 0.0118	\choice 0.954	\choice 0.748	\choice 0.374
\end{choice}

\question How long has a current of 3ampere to be applied through a solution of silver nitrate to coat a metal surface of 80 cm2 with 0.005cm thick layer? Density of Ag=10.5g/cm3.
        \begin{choices}
\choice 476 sec	\choice 1028sec	\choice 1252sec	\choice 683sec
\end{choice}

\question Which one of the following is true?
        \begin{choices}
\choice Entropy increases when a liquid freezes at its melting point	\choice For a spontaneous process G>.
        \choice Entropy of the pure crystalline solid is zero at 0c0	\choice For spontaneous process S.
        \end{choice}

\question A certain current produces 0.50g of hydrogen gas in 2.0hrs. what is the amount of copper librated from a solution of copper sulfate by the same current flowing for the same time?
        \begin{choices}
\choice 15.9g	\choice 63.6g	\choice 31.8g	\choice 6.36g
\end{choice}

\question For which of the following half- cell is the reduction potential E independent of the PH of the solution? 
        \begin{choices}
\choice [ Cr2O7] 2- + 14H+ +6e-	2Cr3+ +7H2	\choice [ NO3]- + H2O +2e-	[ NO2]- + 2[ OH]-
        \choice Co2+ +2e-	Co	                    \choice H2O2 + 2H+ + 2e-	2H2O
        \end{choice}

\question Which of the following metals cannot be electroplated on to the surface of another metal using an aqueous electrolyte?
E0(Ni2+/Ni)=- 0.28V; E0(Cu+/Cu)=0.34v and E0 (Mg2+/Mg) = - 2.37V; E0 (Ag+/Ag)=0.80V
        \begin{choices}
\choice Mg	\choice Cu	\choice Ni	\choice Ag
\end{choice}

\question A nickel electrode can undergo oxidation to Ni2+ ion as follows: Ni(s)	Ni2+ (aq) +2e- E0red=- 0.28V
2H2O(l)	4H+(aq) + O2(g) + 4e- E0red= 1.23V

\question What will happen during the electrolysis of concentrated aqueous solution of NiSO4 using Ni electrodes? I Nickel will deposit at the cathode.	\choice Oxygen will be released at the anode
II. Nickel will dissolve at the anode.	\choice Nickel will deposit at the anode.
\end{choice}
        \begin{choices}
\choice I only	\choice I,II	\choice II,IV	\choice I,III
\end{choice}

\question Features common to both galvanic and electrolytic cells include which of the following?
I. Oxidation at the anode	II. Can perform electrolysis	III. Spontaneous
    \begin{choices}
\choice III only	\choice II only	\choice I only	\choice I and II only
\end{choice}

\question Which of the following is the correct reaction taking place at the electrodes during the electrolysis of dilute sodium chloride solution?
        \begin{choices}
\choice Anode:2Cl- (aq)	Cl2(l) + 2e-   & Cathode: 2Na+(aq) + 2e-	2Na(s)
        \choice Anode: 2Cl- (aq)	Cl2(l) + 2e-	& Cathode: 2H2O(l) + 2e-	H2(g) +OH- (aq)
        \choice Anode: 2H2O(l)+ 2e-	O2(g) + 4H+(aq) + 4e- & Cathode: 2H2O(l) + 2e-	H2(g) +OH- (aq)
        \choice Anode: 2H2O(l)+2e-	O2(g) + 4H+(aq) + 4e-	& Cathode: 2Na+(aq) + 2e-	2Na(s)
        \end{choice}
\question The oxidation state of chlorine in HClO4 is:
        \begin{choices}
\choice +1	\choice +3	\choice +7	\choice +5
\end{choice}




















Chemistry grade- 12 Entrance Chapter- 5

\question The simplest formula for a compound containing Mn+ 4 and O2 is 
        \begin{choices}
\choice MnO	\choice MnO2	\choice Mn2)4	\choice Mn4O2
\end{choice}

\question Which one of the following chemicals is used to disinfect water?
        \begin{choices}
\choice Fluorine	\choice Nitrogen	\choice Oxygen	\choice Chlorine
\end{choice}

\question Which of the following metals forms a volatile compound during the extraction process?
        \begin{choices}
\choice Fe	\choice Co	\choice Ni	\choice Cu
\end{choice}

\question Which of the following metals is extracted by thermal reduction process?
        \begin{choices}
\choice Cu	\choice Fe	\choice Al	\choice Mg
\end{choice}

\question Which of the following metals has the highest electrical and termal conductivities of any Metal?	
        \begin{choices}
\choice Ag	\choice Cu	\choice Ni	\choice Co
\end{choice}

\question Which of the following is the most important  source for the extraction of iron?
        \begin{choices}
\choice Hematite	\choice Bauxite	\choice Chalcopyrite	\choice Sphalerite
\end{choice}

\question Which of the following gases is manufactured using the Haber process?
        \begin{choices}
\choice Ammonia	\choice Nitric oxide	\choice Nitrogen	\choice Nitrogen dioxide
\end{choice}

\question Which of the following elements is the second most abundant element in the earth’s crust?
        \begin{choices}
\choice Aluminum	\choice Iron	\choice Oxygen	\choice Silicon
\end{choice}

\question Which of the following metals has the highest electrical and thermal conductivities?
        \begin{choices}
\choice Ag	\choice Co	\choice Cu	\choice Ni
\end{choice}
\question Which of the following metal alloys does NOT contain tin?
        \begin{choices}
\choice Brass	\choice Bronze	\choice Pewter	\choice Plumber’s solder
\end{choice}

\question Which of the following is the most important ingredients used for production of DAP	fertilizer?
        \begin{choices}
\choice Ammonia and phosphoric acid	\choice Phosphoric acid urea and ammonia
        \choice Nitric acid, urea and phosphoric acid	\choice Sulphuric acid, ammonia and urea
        \end{choice}

\question Consider the following reaction:
I2O5(s) + 5CO (g)	I2(s) + 5CO2(g)
What is the magnitude of the change in the oxidation number of the elements? 
        \begin{choices}
\choice   I,+5 to 0,C,+2to+4	\choice I,+0 to 0,C,+2 to+4
        \choice   I,+5to 0,C,+4to+4	\choice I,+0to 0,C,+2to+4
        \end{choice}

\question Consider the following oxidation- reduction equation: 2H2O(l)+Al(s)+MnO4 (aq)+ →AlOH4(aq)+MnO2(s)
What are the reducing and the oxidizing agents in this reaction?
        \begin{choices}
\choice Al(s) is the reducing agent and H2O is the oxidizing agent
        \choice Al(s) is the oxidizing agent and H2O is the reducing agent
        \choice Al(s)is the oxidizing agent and MnO - (aq) is reducing agent
        \choice Al(s) is the reducing agent and MnO4(aq) is the oxidizing agent
        \end{choice}

\question A solution at 250C contains the metal ions N1- 2+, Pt2+ and Pd2+, all at 1.0 M concentrations. Consider the following standard reduction potentials:
N1- 2+ + 2e-		Ni			E0 = - 0.23V Pt+2 + 2e − 	Pt		E0 = - 0.20V Pd2+ + 2e-		Pd		E0 = - 0.99V
Which metal(s) will plate out first when the solution is electrolyzed?
        \begin{choices}
\choice Ni	\choice Pd	\choice Pt	\choice Ni and Pd
\end{choice}

\question Which of the following statements is true?
        \begin{choices}
\choice A battery is rechargeable
        \choice A battery is produces electricity
        \choice A battery has generally no liquid components.
        \choice A battery produces the same amount of electricity,  regardless of composition.
        \end{choice}

\question All of the following when added to water will produce an electrolytic solution EXCPT
        \begin{choices}
\choice N2(g)	\choice Nal(s)	\choice HCl(g)	\choice KOH(s)
\end{choice}

\question What mass of aluminum is produced in one hour by the electrolysis of molten AlCl3 with a current of 10A? 
        \begin{choices}
\choice 1.5 g	\choice 2.5 g	\choice 3.4 g	\choice 4.3 g
\end{choice}

\question Consider the following balanced equation.
3Ba(NO3)2 (aq) → Fe2(SO4)3 (aq) →3BaSO4(s) + 2Fe(NO3)3(aq
The net ionic equation to describe this balanced equation is
        \begin{choices}
\choice 3Ba+ (aq) + SO 2- (aq) → 3BaSO4(s)
        \choice 3Ba2+ (aq) + 3SO42- (aq) → 3BaSO4(s)
        \choice 6NO3- (aq) + 2Fe2+ (aq) → 2Fe(NO3) 3 (aq)
        \choice 3Ba2+(aq) + 2NO3- (aq) + 2Fe3+ (aq) +3SO42- (aq) → 3BaSO4 (s) + 2Fe+ (aq) + 6NO3- (aq)
        \end{choice}

\question How can silver can be plated onto nickel?
        \begin{choices}
\choice Electricity to a nickel anode in a solution of silver ions.
        \choice Electricity to a silver anode in a solution of nickel ions
        \choice A solution of nickel ions to react with a pice of silver
        \choice Electricity to a nickel cathode in a solution silver ions
        \end{choice}
\question Which of the following ions is the most abundant in sea water?
        \begin{choices}
\choice Na-	\choice Ca2+	\choice CT	\choice HCO3-
\end{choice}

\question Which alkaline earth metal shares diagonal relationship with aluminium?
        \begin{choices}
\choice Ba	\choice Be	\choice Ca	\choice Mg
\end{choice}

\question Which is the metal ion in the porphyrin of heme?
        \begin{choices}
\choice Calcium	\choice Iron	\choice Molybdenum	\choice Magnesium
\end{choice}

\question Which group of elements are the most reactive of all the metallic elements?
        \begin{choices}
\choice Alkali metals	\choice transition metals
        \choice Alkaline earth metals	\choice group 2B metals
        \end{choice}

\question In the process known as 'roasting' ,a (n) __ is chemically converted to a (n).
        \begin{choices}
\choice Carbonate /oxide	\choice sulfide /oxide
        \choice Oxide/ sulfate	\choice phosphate /phosfide
        \end{choice}

\question Why are silicate minerals NOT commonly used as sources of metals?
        \begin{choices}
\choice They are rare minerals.
        \choice They usually do not contain important metals.
        \choice They are difficult to reduce and concentrate.
        \choice They are only found at excessive depths in the oceans.
        \end{choice}

\question Which of the following metals is the best conductor of heat and electricity?
        \begin{choices}
\choice Copper	\choice silver	\choice Gold	\choice Tungsten
\end{choice}

\question HN03 is used as:
    I. a hydrogenating agent.	II. An oxidizing agent.
    II. An acid	IV. An ammoniating agent
        \begin{choices}
\choice II and III	\choice I and II	\choice I, II, III and IV	\choice II, III and IV
\end{choice}

\question Which metal can be found as the free element?
        \begin{choices}
\choice Cr	\choice Fe	\choice Mn	\choice Pt
\end{choice}

\question The conversion of nitrogen gas to nitrates by bacteria is called
        \begin{choices}
\choice nitrification	\choice nitrogen fixation	\choice excretion	\choice dentrification
\end{choice}
\question The most abundant metal on the surface of the earth is
        \begin{choices}
\choice Fe	\choice Al	\choice Ca	\choice Na
\end{choice}

\question Which one of the following metals is extracted by thermal reduction process?
        \begin{choices}
\choice Al	\choice Cu	\choice Fe	\choice Mg
\end{choice}

\question Which of the following gases is manufactured using the Haber process?
        \begin{choices}
\choice Ammonia	\choice Nitric oxide	\choice Nitrogen	\choice Nitrogen dioxide
\end{choice}

\question What is the final concentration of Cl- ion when 250 ml. of 0.20 M CaCl2 solution is mixed with	250ml. of
0.40 M KCl solutions?( Assume additive volumes)
\begin{choices}
\choice 1.60M	\choice 0.40 M	\choice 0.20 M	\choice 0.60 M
\end{choice}

\question The concentration of nitrate ion in a solution that contains 0.900 M aluminum nitrate is \begin{choices}
\choice 0.90	\choice 0.49	\choice 0.30	\choice 2.70
\end{choice}

\question 	Electrolysis of dilute aqueous NaCl solution was carried out by passing 10 mill ampere	current . The time required to liberate 0.01 mol of H2 gas at the cathode is
\begin{choices}
\choice 9.65 X 104 s	B.
 \question 3 X 104 s	\choice 28.95 X 104 s	\choice 38.6 X 104 s
 \end{choice}

\question Which of the following metals forms a volatile compound that is taken as an advantage for its	extraction?
        \begin{choices}
\choice Co	\choice Fe	\choice Ni	\choice W
\end{choice}


\question Which of the following metals is NOT obtained by commercial electrolytic process?
        \begin{choices}
\choice Ag	\choice Al	\choice Cu	\choice Na
\end{choice}

\question Which of the following plant nutrient will be produced as a result of nitrogen fixation?
        \begin{choices}
\choice Carbohydrate	\choice Cellulose	\choice Mineral	\choice Protein
\end{choice}

\question Which is the most common ore used for the extraction of copper?
        \begin{choices}
\choice CuO	\choice CuSO4	\choice CuCO3	\choice CuFeS2
\end{choice}

\question Which of the following material has maximum ductility?
        \begin{choices}
\choice Nickel	\choice Aluminum	\choice Mild steel	\choice Copper
\end{choice}

\question The four most abundant metals in the earths crust in decreasing order of abundance are:
        \begin{choices}
\choice Oxygen, Silicon, Aluminum, and Iron
        \choice Aluminum, Iron, Calcium, and Magnesium
        \choice Iron, Aluminum, Silicon. and Oxygen
        \choice Silicon, Aluminum, Magnesium ,and Sodium
        \end{choice}

\question What is galvanized iron?
        \begin{choices}
\choice Iron that is coated with tin	\choice Iron that is coated with zinc
        \choice Iron that is coated with chromium	\choice Iron that is coated with aluminum
        \end{choice}

\question Sulphuric acid is prepared industrially by contact process. In this process, sulphur will be oxidized to SO3 by
excess oxygen . Why is SO3 formed in this process NOT absorbed directly in water to form H2SO4 ? Because
        \begin{choices}
\choice V2O5 catalyst does not remove all impurities
        \choice when SO3 dissolves in water , it produce high heat and is difficult to liquefy
        \choice the reaction that produces SO3  is exothermic and unable to produce H2S2O7
        \choice the direct reaction of SO3 with water prohibits the formation of H2S2O7
        \end{choice}

\question Which oxide of a metal gets reduced only by coke and not by H2 gas or CO gas?
        \begin{choices}
\choice Fe2O3	\choice PbO	\choice ZnO	\choice CuO
\end{choice}

\question Ores are complex mixtures of metal containing minerals and accompanying rocks and soil, called gangue. What will be the correct order of the extraction processes to get a pure metal?
        \begin{choices}
\choice Floatation ,distillation, and reduction	\choice  Electrolysis , refining, and chemical reduction
        \choice Concentration ,reduction, and refining	\choice Pretreatment, distillation ,and electrolytic reduction
        \end{choice}

\question Which of the following elements is used to galvanize metals for corrosion protection?
        \begin{choices}
\choice Sn	\choice Zn	\choice Cu	\choice Na
\end{choice}

\question How do alkaline earth metals exist in nature?
        \begin{choices}
\choice Salts	\choice Liquids	\choice Oxides	\choice Metals
\end{choice}

\question During nitrification, bacteria convert…………into …………..
        \begin{choices}
\choice N2, N2O into nitrous oxide	\choice N2, nitrous oxide into nitrogen
        \choice NO3-,NH4+ or nitrates into ammonium	\choice N2 into nitrates
        \end{choice}

\question Which of the following metals exists in a free state?
        \begin{choices}
\choice Pt	\choice Mg	\choice Zn	\choice Na
\end{choice}
\question The main ore of lead is called :
        \begin{choices}
\choice Cinnabar	\choice Zinc blend	\choice Galena	\choice Chromite
\end{choice}

\question Which of the following reactions is Not involved in the contact process, dring the production of sulfuric acid?
        \begin{choices}
\choice SO2 +O2	2SO2	\choice SO3 + H2SO4	H2S2O7
        \choice H2S2O7 + H2O	H2SO4	\choice SO3 + H2O	H2SO4
        \end{choice}

\question When NaCrO4 solution is acidified, which of the following is formed?
        \begin{choices}
\choice Cr metal	\choice Cr2O72-	\choice CrO 2-	\choice Cr2O3
\end{choice}

\question What makes the phosphorous cycle different from carbon and nitrogen cycles?
        \begin{choices}
\choice Phosphourous is found in the atmosphere in the gaseous state
        \choice Carbon and nitrogen are found in the atmosphere in the gaseous state
        \choice Phosphorus salts are released at the higher rate compared to carbon and nitrogen cycle
        \choice Carbon and nitrogen are released at lower rate compared to phosphatebsalts
        \end{choice}

\question The metal extracted from limestone, chalk and marble is……..
        \begin{choices}
\choice Sodium	\choice Zinc	\choice Calcium	\choice Chromium
\end{choice}





 \question Which of the following is natural polymer?
Chemistry grade- 12 Entrance Chapter- 6
        \begin{choices}
\choice Nylon  \choice PVC \choice Cotton	\choice Dacron
\end{choice}

\question Which pair of monomers forms polyesters?
        \begin{choices}
\choice bifunctional alcohol, bifunctional organic acid	\choice Ethylene, ethylene
        \choice bifunctional alcohol, bifunctional amino acid	\choice Amino acid, amino acid
        \end{choice}

\question Which of the following is a synthetic rubber produced from caprolactam (CPL)?
        \begin{choices}
\choice Nylon 6, 10	\choice Terlyene	\choice Teflon	\choice nylon 6
\end{choice}

\question Which of the following is a natural polymer?
        \begin{choices}
\choice Polythene	\choice polysaccharides	\choice Nylon	\choice Terylene
\end{choice}

\question Which of the following is a linear polymer?
        \begin{choices}
\choice High density polyethene (HDPE)	\choice Bakelite
        \choice Low density polyethene (LDPE)	\choice Vulcanized rubber
        \end{choice}

\question What structural feature is usually needed to present in order for an addition polymer to be produced?
        \begin{choices}
\choice A carbon – carbon sigma bond	\choice A carbon – oxygen sigma bond
        \choice A carbon – oxygen pi bond	\choice A carbon – carbon pi bond
        \end{choice}

\question What are the raw materials required to synthesize nylon 66, a specific kind of nylon?
        \begin{choices}
\choice Formaldehyde	\choice Diamine	\choice Nitrogen	\choice Sulfur
\end{choice}

\question Which of the following substances are added to natural rubber to toughen it?
        \begin{choices}
\choice Calcium	\choice Carbon	\choice Nitrogen	\choice Sulfur
\end{choice}

\question 	Which of the following form of synthetic rubbers can be vulcanized to greatly enhance its mechanical strength?
        \begin{choices}
\choice Neoprene	\choice Isoprene	\choice Styrene- butadiene rubber	\choice Butyl rubber
\end{choice}

\question What is the range in the number of carbon atoms of the monosaccharides that are found in nature? 
        \begin{choices}
\choice 3 to 7	\choice 4 to 10	\choice 4 to 12	d. 5 to 12

\question Which of the following biomolecules forms a zwitterion at higher or lower pH?
        \begin{choices}
\choice Cellulose	\choice Glucose	\choice Protein	\choice Starch
\end{choice}

\question What are the two principal polysaccharide forms of starch?
        \begin{choices}
\choice Aldohexose and ketopentose	\choice maltose and cellobiose
        \choice Amylose and amylopectin	\choice sucrose and lactose
        \end{choice}

\question Which synthetic polymer  is produced from caprolactam?
        \begin{choices}
\choice Nylon - 6	\choice Nylon 6,10	\choice Teflon	\choice Terylene
\end{choice}

\question Which of the following is a chemical formula that represents an amino acid?
        \begin{choices}
\choice CH4	\choice CH3NH2	\choice CH3COOH	\choice NH2CH2COOH
\end{choice}

\question Which of the following is NOT a carbohydrate?
        \begin{choices}
\choice Starch	\choice Glucose	\choice Glycine	\choice Cellulose
\end{choice}

\question Which element is added to natural rubber to make it harder and reduce its susceptibility to oxidation and chemical attacks?
        \begin{choices}
\choice Sulfur	\choice Silicon	\choice Carbon	\choice Nitrogen
\end{choice}

\question Which of the following statements is not correct?
        \begin{choices}
\choice Fats are esters of glycerol and the fatty acids
        \choice Fats produce more .energy .per gram than either proteins or carbohydrates.
        \choice Fats are insoluble in water, with permits their storage in the body.
        \choice True vegetarians, who do not eat meat, fish, eggs, or dairy products, have zero cholesterol in their bodies.
        \end{choice}

\question Synthetic fibers are:
        \begin{choices}
\choice naturally produced.
        \choice Manufacture through the use of chemical substances.
        \choice Produced from animal hair fibers.
        \choice Produced from plant substances.
        \end{choice}

\question What is the type of monomer unit in a natural protein polymer?
        \begin{choices}
\choice Amino acids	\choice Essential amino acids	\choice a- amino acids	\choice nucleic acids
\end{choice}

\question Which of the following are carbohydrates?
    \question Nucleic acids	II. Polyhydroxyketones III. Disaccharides IV. Aldoses V. triglycerides
        \begin{choices}
\choice II, III andHZ	\choice I and V	\choice I, III and V	\choice III and IV
\end{choice}

\question At. which point can only the solid and liquid phases coexist in phase diagram of water given	below ? Diagram
        \begin{choices}
\choice 3	\choice 4	C.5	\choice 8
\end{choice}

\question Which of the following plant nutrient will be produced as a result of nitrogen fixation? Carbohydrate	\choice Cellulose	\choice Mineral	\choice Protein
\end{choice}

    \question Which of the folloing is used in the reaction called saponification?
A .Strong base	\choice Strong acid	\choice Hydrogen	\choice Nickel
\end{choice}

\question Natural rubber is a polymer of
        \begin{choices}
\choice butadiene	\choice isoprene	\choice neoprene	\choice styrene
\end{choice}

\question Which of the following form of synthetic rubbers can be vulcanized to greatly enhance its mechanical strength?
        \begin{choices}
\choice Neoprene   \choice Isoprene	\choice Styrene- butadiene rubber	\choice Butyl rubber
\end{choice}

\question Which of the following reactions will convert carboxylic acids to primary amines?
        \begin{choices}
\choice Decarboxylation with HBr and peroxide, then reaction of the alkyl brominde with ammonia
        \choice Reduction of the acid to the alcohol with NaOH/ formaldehyde, then reaction with ammonium chloride and heat
        \choice A two –step conversion, first to the amide with ammonia and heat, and then by reduction with lithium aluminum hydride or hydrogen plus catalyst
        \choice Lactase fermentation in the presence of ammonia atmosphere
        \end{choice}

\question Which of the following statements about polyvinyl chloride is NOT correct?
        \begin{choices}
\choice PVC can be used in making water pipes	\choice PVC is stiff
        \choice PVC is softened on heating	            \choice The monomer of PVC is CHCl = CHCl
        \end{choice}

\question Bakelite is obtained from phenol by reacting with
        \begin{choices}
\choice HCHO	\choice (CH2OH)2	\choice CH3CHO	\choice CH3COCH3
\end{choice}

\question Which of the following is a natural polymer?
        \begin{choices}
\choice Keratin	\choice Polythene	\choice Cellulose	\choice Polymethacrylate
\end{choice}

\question A polysaccharide is a polymer made up of which kind of monomers?
        \begin{choices}
\choice Amino acids	\choice Nucleotides
        \choice Simple sugars	\choice Alternating sugar and phosphate groups
        \end{choice}
    
\question Which pair of the following compounds will be the source of phenyl acetate,  CH3CO2C6H5?
        \begin{choices}
\choice CH3CH2OH and C6H5COOH	\choice CH3COOH and C6H5COOH
        \choice CH3COOH and	C6H5OH	\choice CH3CH2OH and C6H5OH
        \end{choice}

\question Consider the following organic compounds, What will be its IUPAC name?
        \begin{choices}
\choice 3,4- dihydroxybenzoic acid	\choice 4,5 - - dihydroxybenzoic acid
        \choice m, p- dihydroxybenzoic acid	\choice 3- hydroxy- para- benzoic acid
        \end{choice}

\question Suppose you went to the market to buy cooking utensils coated with a synthetic polymer, which types of polymer do you prefer for such purpose?
        \begin{choices}
\choice Teflori	\choice PMMA	\choice Polystyrene	\choice Polypropylene chloride
\end{choice}

\question Which of the following statements is true?
        \begin{choices}
\choice The hydrolysis of esters should be carried out at pH 7.0 for optimum efficiency
        \choice The hydrolysis of esters can be carried out under acidic or basic conditions
        \choice The hydrolysis of esters is not pH dependent
        \choice The hydrolysis of esters must be acid catalyzed
        \end{choice}

\question Which of the following plastics might Y be?
        \begin{choices}
\choice Perspex	\choice Urea- methanol	\choice Polystyrene	\choice Polyester
\end{choice}

\question The monomer of neoprene is:
        \begin{choices}
\choice Butadiene	\choice 2- methyl- 1,3- buthadiene	\choice Isoprene	\choice Chloroprene
\end{choice}

\question Nylons are:
        \begin{choices}
\choice Polyamides	\choice Peptides	\choice Amides	\choice polyesters
\end{choice}

\question Which of the following is a natural polymer?
        \begin{choices}
\choice Perspex	    \choice PVC	    \choice Rubber	\choice Teflon
\end{choice}

\question The process of vulcanization of rubber makes it…………
        \begin{choices}
\choice Less elastic	    \choice Soft	    \choice Hard	    \choice more soluble in solvents
\end{choice}


\question Two types of polymers are shawn below. Which of the following statements concerning these polymers is correct?
        \begin{choices}
\choice X and Y are thermosetting                        \choice X & Y are thermoplastic
        \choice X is thermosetting and Y is the thermoplastic	\choice X is thermoplastic whereas Yis thermosetting
        \end{choice}

\question Nylon is a/an	?
            \begin{choices}
\choice Polyamide	\choice Peptide	\choice Amide	\choice Polyester
\end{choice}

\question Which of the following is Not a component of polysaccharides?
        \begin{choices}
\choice Sucrose	\choice Glucose	\choice Cellulose	\choice Glycogen
\end{choice}

\question Which substance is used to lower the melting point of aluminum oxide ore in the electrolytic extraction of aluminum?
        \begin{choices}
\choice Bauxite	\choice Cryolite	\choice Hematite	\choice Magnetite
\end{choice}

\question Which one of the following is Not a condensation polymer?
        \begin{choices}
\choice Polyamides	\choice Neoprene	\choice Poyester	\choice Nylon
\end{choice}

\end{document}